\documentclass[12pt]{article}
\usepackage{amsmath}
\usepackage{graphicx}
\usepackage{hyperref}
\usepackage{listings}
\usepackage{color}
\usepackage{pythonhighlight}

\title{Operating System Course Report - First Half of the Semester}
\author{A class}
\date{\today}

\begin{document}

\maketitle
\newpage

\tableofcontents
\newpage

\section{Introduction}
This report summarizes the topics covered during the first half of the Operating System course. It includes theoretical concepts, practical implementations, and assignments. The course focuses on the fundamentals of operating systems, including system architecture, process management, CPU scheduling, and deadlock handling.

\section{Course Overview}
\subsection{Objectives}
The main objectives of this course are:
\begin{itemize}
    \item To understand the basic components and architecture of a computer system.
    \item To learn process management, scheduling, and inter-process communication.
    \item To explore file systems, input/output management, and virtualization.
    \item To study the prevention and handling of deadlocks in operating systems.
\end{itemize}

\subsection{Course Structure}
The course is divided into two halves. This report focuses on the first half, which covers:
\begin{itemize}
    \item Basic Concepts and Components of Computer Systems
    \item System Performance and Metrics
    \item System Architecture of Computer Systems
    \item Process Description and Control
    \item Scheduling Algorithms
    \item Process Creation and Termination
    \item Introduction to Threads
    \item File Systems
    \item Input and Output Management
    \item Deadlock Introduction and Prevention
    \item User Interface Management
    \item Virtualization in Operating Systems
\end{itemize}

\section{Topics Covered}

\subsection{Basic Concepts and Components of Computer Systems}
This section explains the fundamental components that make up a computer system, including the CPU, memory, storage, and input/output devices.

\subsection{System Performance and Metrics}
This section introduces various system performance metrics used to measure the efficiency of a computer system, including throughput, response time, and utilization.

\subsection{System Architecture of Computer Systems}
Describes the architecture of modern computer systems, focusing on the interaction between hardware and the operating system.

\subsection{Process Description and Control}
Processes are a central concept in operating systems. This section covers:
\begin{itemize}
    \item Process states and state transitions
    \item Process control block (PCB)
    \item Context switching
\end{itemize}

\subsection{Scheduling Algorithms}
This section covers:
\begin{itemize}
    \item First-Come, First-Served (FCFS)
    \item Shortest Job Next (SJN)
    \item Round Robin (RR)
\end{itemize}
It explains how these algorithms are used to allocate CPU time to processes.

\subsection{Process Creation and Termination}
Details how processes are created and terminated by the operating system, including:
\begin{itemize}
    \item Process spawning
    \item Process termination conditions
\end{itemize}

\subsection{Introduction to Threads}
This section introduces the concept of threads and their relation to processes, covering:
\begin{itemize}
    \item Single-threaded vs. multi-threaded processes
    \item Benefits of multithreading
\end{itemize}

\begin{figure}[h]
    \centering
    \includegraphics[width=0.5\textwidth]{/Users/khawaritzmi/Unhas/os_report_mid2024/a_class/asset/example.png}  % Sesuaikan nama file dan ukurannya
    \caption{Ini adalah gambar contoh dari multithreading.}
    \label{fig:contoh_gambar}
\end{figure}

Seperti yang terlihat pada Gambar \ref{fig:contoh_gambar}, inilah cara menambahkan gambar dengan keterangan.

\subsection{File Systems}
File systems provide a way for the operating system to store, retrieve, and manage data. This section explains:
\begin{itemize}
    \item File system structure
    \item File access methods
    \item Directory management
    \begin{enumerate}
        \item File System Mounting 
        \item \textbf{File Sharing} adalah proses yang memungkinkan individu atau entitas untuk berbagi file digital dengan orang lain melalui jaringan komputer atau internet. Ini memungkinkan pengguna untuk mentransfer file, seperti dokumen, gambar, video, musik, dan lainnya, dari satu perangkat ke perangkat lainnya. File sharing dapat dilakukan secara lokal, melalui jaringan lokal seperti LAN (Local Area Network), atau secara online melalui layanan cloud atau platform file sharing.
        \begin{itemize}
            \item {Manfaat File Sharing} 
            
            Dalam era digital saat ini, di mana kolaborasi dan pertukaran informasi menjadi kunci kesuksesan, praktik file sharing telah menjadi salah satu aspek terpenting dalam berbagai lingkungan kerja. Baik itu dalam tim kecil, perusahaan besar, atau bahkan dalam lingkup pendidikan, kemampuan untuk berbagi file dengan cepat dan efisien telah menjadi landasan dari produktivitas dan inovasi.
            Berikut manfaat penting file sharing dalam meningkatkan efisiensi dan kolaborasi di lingkungan kerja modern:
            \begin{itemize}
               \item Akses Mudah dan Cepat 
               
                Salah satu keuntungan utama dari file sharing adalah akses mudah dan cepat terhadap informasi. Dengan menggunakan platform file sharing yang tepat, seperti Google Drive, Dropbox, atau OneDrive, individu atau tim dapat dengan mudah mengunggah dan mengunduh file dari mana saja, kapan saja, asalkan terhubung ke internet. Ini memungkinkan anggota tim untuk mengakses dokumen penting bahkan saat mereka tidak berada di kantor atau di lokasi fisik lainnya.
                \item Kolaborasi Tim yang Efisien

                File sharing juga memainkan peran kunci dalam memfasilitasi kolaborasi tim yang efisien. Dengan berbagi file secara online, anggota tim dapat secara bersama-sama mengedit dokumen, menyampaikan umpan balik, dan melacak perubahan dalam waktu nyata. Ini menghilangkan kebutuhan untuk mengirim file melalui email atau menyimpan versi terbaru secara terpisah, yang sering kali dapat mengakibatkan kekacauan dan kebingungan.
               \item Peningkatan Produktivitas 

               Dengan akses mudah dan kolaborasi yang ditingkatkan, file sharing berkontribusi secara langsung pada peningkatan produktivitas di tempat kerja. Tim dapat bekerja lebih efisien, menghemat waktu yang sebelumnya dihabiskan untuk mencari file yang diperlukan atau menunggu tanggapan dari rekan kerja. Selain itu, dengan kemampuan untuk mengakses file dari berbagai perangkat, seperti laptop, tablet, atau ponsel pintar, individu dapat tetap produktif bahkan saat mereka berada dalam perjalanan.
               \item Backup dan Penyimpanan
               
               Meskipun kemudahan akses dan kolaborasi yang ditawarkan oleh file sharing sangatlah penting, keamanan data juga merupakan faktor yang tidak boleh diabaikan. Platform file sharing modern sering dilengkapi dengan fitur keamanan canggih, seperti enkripsi end-to-end, kontrol akses, dan otorisasi dua faktor, yang membantu melindungi informasi sensitif dari akses yang tidak sah atau kebocoran data. Ini memberikan ketenangan pikiran kepada organisasi bahwa data mereka aman, bahkan saat berada dalam perjalanan di internet.
               \vspace{2pt}
            \end{itemize}
            \item Metode File Sharing

            File sharing adalah praktik yang memungkinkan individu atau organisasi untuk berbagi data dengan mudah dan efisien. Dalam dunia digital saat ini, terdapat berbagai metode yang dapat digunakan untuk melakukan file sharing, masing-masing dengan kelebihan dan kekurangan. Berikut adalah beberapa metode file sharing yang umum digunakan:

            \begin{itemize}
                \item Layanan Cloud Storage

                Email masih menjadi salah satu cara yang paling umum digunakan untuk berbagi file. Pengguna dapat melampirkan file ke pesan email dan mengirimkannya kepada penerima yang diinginkan. Sebagian besar penyedia layanan email memiliki batasan ukuran file yang dapat dilampirkan, jadi pastikan ukuran file tidak melebihi batas yang ditetapkan.
                \item Aplikasi P2P (Peer-to-Peer)

                Aplikasi P2P memungkinkan pengguna untuk berbagi file langsung antara satu sama lain, tanpa perantara. Pengguna menggunakan perangkat lunak khusus untuk terhubung ke jaringan P2P dan dapat mengunduh atau mengunggah file sesuai kebutuhan.
                
                \item Tautan Berbagi

                Banyak platform file sharing, termasuk layanan penyimpanan awan dan platform kolaborasi seperti Google Drive dan SharePoint, memungkinkan pengguna untuk membuat tautan berbagi yang dapat dibagikan kepada orang lain. Dengan mengirimkan tautan ini, penerima dapat mengakses file tanpa perlu login atau memiliki akun di platform tersebut.
                \item Jaringan Lokal (LAN)

                File sharing menggunakan jaringan lokal (LAN) memungkinkan berbagi file antar perangkat yang terhubung dalam jaringan internal tanpa memerlukan internet. Ini bisa dilakukan dengan berbagai metode, seperti berbagi folder di Windows atau Mac, menggunakan perangkat Network Attached Storage (NAS), atau melalui protokol SMB dan FTP. Metode ini memberikan kecepatan transfer yang tinggi, mudah diakses, dan cocok untuk kolaborasi dalam organisasi kecil seperti kantor atau sekolah. Namun, penting untuk memastikan keamanan dengan mengatur izin akses dan menjaga jaringan dari ancaman eksternal.
            \end{itemize}
            \item Keamanan dalam File Sharing

            Keamanan dalam file sharing sangat penting untuk menjaga data dari akses tidak sah, pencurian informasi, atau penyebaran malware. File sharing yang tidak dilindungi dengan baik dapat menjadi target serangan siber, yang berisiko menyebabkan kerugian besar, terutama bagi organisasi yang menangani informasi sensitif. Untuk itu, beberapa langkah dapat diambil guna memastikan bahwa file sharing dilakukan dengan aman.
            \begin{itemize}
                \item Enkripsi File

                Enkripsi file adalah proses mengamankan data sehingga hanya pengguna yang memiliki kunci enkripsi yang dapat membukanya. Enkripsi melindungi file selama penyimpanan (at rest) dan saat dikirim (in transit). Ketika file dibagikan melalui internet atau jaringan, enkripsi memastikan bahwa meskipun file tersebut dicegat oleh pihak yang tidak sah, isinya tidak dapat dibaca tanpa kunci yang sesuai.
                \item Izin Akses

                Izin akses bertujuan untuk mengatur siapa saja yang dapat melihat, mengedit, atau menghapus file. Ini penting untuk membatasi akses hanya kepada pengguna yang berwenang. Dengan membatasi hak akses, organisasi dapat menghindari potensi penyalahgunaan file oleh pengguna yang tidak berhak.

                \item Autentikasi dan Kata Sandi

                Autentikasi yang kuat dan penggunaan kata sandi yang kompleks sangat penting dalam menjaga keamanan file sharing. Autentikasi dapat dilakukan dengan memverifikasi identitas pengguna melalui metode seperti autentikasi dua faktor (2FA). Kata sandi yang kompleks dan unik mengurangi risiko peretasan yang sering terjadi akibat kata sandi yang lemah.

                \item Firewall dan Antivirus

                Firewall berfungsi sebagai pelindung terhadap akses tidak sah ke dalam jaringan, sementara antivirus melindungi perangkat dari serangan malware. Mengaktifkan firewall dan memastikan antivirus selalu diperbarui adalah cara untuk mencegah serangan siber yang dapat mencuri atau merusak file selama proses sharing.

                \item VPN (Virtual Private Network)

                VPN digunakan untuk mengenkripsi koneksi jaringan saat berbagi file jarak jauh melalui internet. VPN melindungi data dari penyadapan dengan mengenkripsi seluruh lalu lintas jaringan yang melewati server. Hal ini memastikan bahwa file yang dikirim tidak dapat dilihat oleh pihak ketiga selama proses transfer.
            \end{itemize}
        \end{itemize}

        \item Proteksi
    \end{enumerate}  
\end{itemize}
\end{itemize}

\subsection{Input and Output Management}
Input and output management is key for handling the interaction between the system and external devices. This section includes:
\begin{itemize}
    \item Device drivers
    \item I/O scheduling
\end{itemize}

\subsection{Deadlock Introduction and Prevention}
Explores the concept of deadlocks and methods for preventing them:
\begin{itemize}
    \item Deadlock conditions
    \item Deadlock prevention techniques
\end{itemize}

\subsection{User Interface Management}
This section discusses the role of the operating system in managing the user interface. Topics covered include:
\begin{itemize}
    \item Graphical User Interface (GUI)
    \item Command-Line Interface (CLI)
    \item Interaction between the user and the operating system
\end{itemize}

\subsection{Virtualization in Operating Systems}
Virtualization allows multiple operating systems to run concurrently on a single physical machine. This section explores:
\begin{itemize}
    \item Concept of virtualization
    \item Hypervisors and their types
    \item Benefits of virtualization in modern computing
\end{itemize}

\section{Assignments and Practical Work}
\subsection{Assignment 1: Process Scheduling}
Students were tasked with implementing various process scheduling algorithms (e.g., FCFS, SJN, and RR) and comparing their performance under different conditions.
\subsubsection{Group 1}
\begin{python}
    class Process:
    def __init__(self, pid, arrival_time, burst_time):
        self.pid = pid
        self.arrival_time = arrival_time
        self.burst_time = burst_time
        self.completion_time = 0
        self.turnaround_time = 0
        self.waiting_time = 0
\end{python}

\begin{table}[htbp] % Optional: For floating position
    \centering
    \begin{tabular}{|c|c|c|} % Defines number of columns and alignment (c = center, l = left, r = right). '|' creates vertical lines.
    \hline
    Header 1 & Header 2 & Header 3 \\ % Column headers
    \hline
    Row 1, Column 1 & Row 1, Column 2 & Row 1, Column 3 \\ % First row of data
    \hline
    Row 2, Column 1 & Row 2, Column 2 & Row 2, Column 3 \\ % Second row of data
    \hline
    \end{tabular}
    \caption{Your table caption} % Optional: For adding a caption
    \label{tab:your_label} % Optional: For cross-referencing the table
\end{table}
\subsection{Assignment 2: Deadlock Handling}
In this assignment, students were asked to simulate different deadlock scenarios and explore various prevention methods.

\subsection{Assignment 3: Multithreading and Amdahl's Law}
This assignment involved designing a multithreading scenario to solve a computationally intensive problem. Students then applied **Amdahl's Law** to calculate the theoretical speedup of the program as the number of threads increased.

\subsection{Assignment 4: Simple Command-Line Interface (CLI) for User Interface Management}
Students were tasked with creating a simple **CLI** for user interface management. The CLI should support basic commands such as file manipulation (creating, listing, and deleting files), process management, and system status reporting.

\subsection{Assignment 5: File System Access}
In this assignment, students implemented file system access routines, including:
\begin{itemize}
    \item File creation and deletion
    \item Reading from and writing to files
    \item Navigating directories and managing file permissions
\end{itemize}

\section{Conclusion}
The first half of the course introduced core operating system concepts, including process management, scheduling, multithreading, and file system access. These topics provided a foundation for more advanced topics to be covered in the second half of the course.

\end{document}