\documentclass[12pt]{article}
\usepackage{amsmath}
\usepackage{graphicx}
\usepackage{hyperref}
\usepackage{listings}
\usepackage{color}
\usepackage{pythonhighlight}

\title{Operating System Course Report - First Half of the Semester}
\author{B class}
\date{\today}

\begin{document}

\maketitle
\newpage

\tableofcontents
\newpage

\section{Introduction}
This report summarizes the topics covered during the first half of the Operating System course. It includes theoretical concepts, practical implementations, and assignments. The course focuses on the fundamentals of operating systems, including system architecture, process management, CPU scheduling, and deadlock handling.

\section{Course Overview}
\subsection{Objectives}
The main objectives of this course are:
\begin{itemize}
    \item To understand the basic components and architecture of a computer system.
    \item To learn process management, scheduling, and inter-process communication.
    \item To explore file systems, input/output management, and virtualization.
    \item To study the prevention and handling of deadlocks in operating systems.
\end{itemize}

\subsection{Course Structure}
The course is divided into two halves. This report focuses on the first half, which covers:
\begin{itemize}
    \item Basic Concepts and Components of Computer Systems
    \item System Performance and Metrics
    \item System Architecture of Computer Systems
    \item Process Description and Control
    \item Scheduling Algorithms
    \item Process Creation and Termination
    \item Introduction to Threads
    \item File Systems
    \item Input and Output Management
    \item Deadlock Introduction and Prevention
    \item User Interface Management
    \item Virtualization in Operating Systems
\end{itemize}

\section{Topics Covered}

\subsection{Basic Concepts and Components of Computer Systems}
This section explains the fundamental components that make up a computer system, including the CPU, memory, storage, and input/output devices.

\subsection{System Performance and Metrics}
This section introduces various system performance metrics used to measure the efficiency of a computer system, including throughput, response time, and utilization.

\subsection{System Architecture of Computer Systems}
Describes the architecture of modern computer systems, focusing on the interaction between hardware and the operating system.

\subsection{Process Description and Control}
Processes are a central concept in operating systems. This section covers:
\begin{itemize}
    \item Process states and state transitions
    \item Process control block (PCB)
    \item Context switching
\end{itemize}

\subsection{Scheduling Algorithms}
This section covers:
\begin{itemize}
    \item First-Come, First-Served (FCFS)
    \item Shortest Job Next (SJN)
    \item Round Robin (RR)
\end{itemize}
It explains how these algorithms are used to allocate CPU time to processes.

\subsection{Process Creation and Termination}
Details how processes are created and terminated by the operating system, including:
\begin{itemize}
    \item Process spawning
    \item Process termination conditions
\end{itemize}

\subsection{Introduction to Threads}
This section introduces the concept of threads and their relation to processes, covering:
\begin{itemize}
    \item Single-threaded vs. multi-threaded processes
    \item Benefits of multithreading
\end{itemize}

\begin{figure}[h]
    \centering
    \includegraphics[width=0.5\textwidth]{/Users/khawaritzmi/Unhas/os_report_mid2024/b_class/asset/example.png}  % Sesuaikan nama file dan ukurannya
    \caption{Ini adalah gambar contoh dari multithreading.}
    \label{fig:contoh_gambar}
\end{figure}

Seperti yang terlihat pada Gambar \ref{fig:contoh_gambar}, inilah cara menambahkan gambar dengan keterangan.

\subsection{File Systems}
File systems provide a way for the operating system to store, retrieve, and manage data. This section explains:
\begin{itemize}
    \item File system structure
    \item File access methods
    \item Directory management
\end{itemize}

\subsection{Input and Output Management}
Input and output management is key for handling the interaction between the system and external devices. This section includes:
\begin{itemize}
    \item Device drivers
    \item I/O scheduling
\end{itemize}

\subsection{Deadlock Introduction and Prevention}
Deadlock adalah kondisi dalam sistem komputasi di mana dua atau lebih proses tidak dapat melanjutkan eksekusi karena saling menunggu sumber daya yang sedang digunakan oleh proses lain. Deadlock terjadi ketika setiap proses dalam himpunan tersebut menunggu sumber daya yang dipegang oleh proses lain, sehingga menyebabkan lingkaran penantian yang tak berujung. Akibatnya, tidak ada proses yang dapat melanjutkan, dan sistem mengalami kebuntuan.

Model deadlock dapat dijelaskan menggunakan diagram graf alokasi sumber daya (\textit{Resource Allocation Graph} - RAG). Pada RAG, terdapat dua jenis simpul:

\begin{enumerate}
    \item \textbf{Proses}: Digambarkan sebagai lingkaran.
    \item \textbf{Sumber daya}: Digambarkan sebagai kotak atau persegi panjang.
\end{enumerate}

Setiap sumber daya mungkin memiliki beberapa unit yang dapat dialokasikan ke proses. Terdapat dua jenis busur (panah) dalam graf ini:

\begin{itemize}
    \item \textbf{Busur permintaan (request edge)}: Dari proses ke sumber daya, menunjukkan bahwa proses meminta sumber daya.
    \item \textbf{Busur alokasi (assignment edge)}: Dari sumber daya ke proses, menunjukkan bahwa sumber daya telah dialokasikan ke proses tersebut.
\end{itemize}

Contoh sederhana deadlock: Misalkan ada dua proses, P1 dan P2, serta dua sumber daya, R1 dan R2. Proses P1 memegang sumber daya R1 dan meminta R2, sementara P2 memegang R2 dan meminta R1. Maka terbentuk lingkaran penantian sebagai berikut:

\begin{enumerate}
    \item P1 $\rightarrow$ (meminta) R2
    \item P2 $\rightarrow$ (meminta) R1
    \item R1 $\rightarrow$ (dialokasikan) P1
    \item R2 $\rightarrow$ (dialokasikan) P2
\end{enumerate}

Karena kedua proses saling menunggu, sistem mengalami deadlock.

\begin{figure}[htbp]
    \centering
    \includegraphics[width=0.5\linewidth]
    {asset/deadlockRAG.drawio.png}
    \label{fig:deadlock-RAG}
\end{figure}

Dalam diagram ini, P1 menunggu R2 dan P2 menunggu R1, sehingga keduanya tidak dapat melanjutkan eksekusi.

\subsubsection{Deadlock: Konsep Dasar, Contoh, dan Deteksi}

Deadlock adalah situasi di mana dua atau lebih proses mengalami kebuntuan karena saling menunggu sumber daya yang dipegang oleh proses lain. Deadlock sering terjadi pada sistem multiproses dan multithreading, di mana proses-proses berbagi sumber daya seperti memori, berkas, atau perangkat keras.

Sumber daya yang terlibat dalam deadlock dapat berupa:

\begin{itemize}
    \item \textbf{Sumber daya fisik}: seperti pencetak (printer), port jaringan, CPU, atau memori.
    \item \textbf{Sumber daya logis}: seperti berkas, mutex, semafor, atau basis data.
\end{itemize}

Sistem operasi modern menghadapi risiko deadlock terutama pada sistem yang mendukung multitasking.

\subsubsection{Contoh Kasus Deadlock dalam Kehidupan Nyata}

Beberapa analogi yang menjelaskan deadlock secara visual:

\begin{itemize}
    \item \textbf{Contoh Jalan Raya}: Dua mobil berhadapan di jalan sempit, masing-masing menunggu yang lain mundur. Keduanya saling menunggu dan tidak ada yang dapat bergerak. Ini adalah situasi deadlock.
    \item \textbf{Contoh Restoran}: Seorang pelayan A memiliki kopi tetapi membutuhkan sendok dari pelayan B. Sementara itu, pelayan B memiliki sendok tetapi membutuhkan cangkir dari pelayan A. Keduanya tidak dapat menyelesaikan tugas karena saling menunggu.
\end{itemize}

\subsubsection{Deteksi Deadlock}

Pendekatan utama dalam menangani deadlock adalah deteksi deadlock, di mana sistem secara aktif memeriksa apakah telah terjadi deadlock dan mengambil langkah-langkah untuk menyelesaikannya.

Berikut beberapa metode untuk mendeteksi deadlock:

\paragraph{a. Penggunaan Resource Allocation Graph (RAG)}

RAG digunakan untuk memodelkan alokasi sumber daya ke proses. Jika ada siklus dalam graf ini, maka deadlock mungkin terjadi. Dalam RAG:

\begin{itemize}
    \item \textbf{Proses}: Digambarkan sebagai lingkaran.
    \item \textbf{Sumber daya}: Digambarkan sebagai kotak atau persegi panjang.
\end{itemize}

Jika ada siklus dalam graf ini, deadlock telah terjadi.

\paragraph{b. Algoritma Deteksi Deadlock pada Sumber Daya Tunggal}

Pada sistem di mana setiap sumber daya hanya memiliki satu instansi, deteksi deadlock dilakukan dengan memeriksa apakah ada siklus dalam graf alokasi sumber daya. Jika siklus terdeteksi, maka deadlock terjadi.

\paragraph{c. Algoritma Deteksi Deadlock pada Sumber Daya Multipel}

Jika sebuah sistem memiliki beberapa instansi dari jenis sumber daya yang sama, maka algoritma seperti \textit{Banker's Algorithm} dapat digunakan. Algoritma ini menilai apakah sistem dapat memenuhi semua permintaan sumber daya tanpa memasuki kondisi deadlock.

Dengan mengidentifikasi cara aman untuk mengalokasikan sumber daya, algoritma ini membantu mencegah terjadinya deadlock.

\paragraph{Referensi}
Ali A. \textit{151444239-Sistem-Operasi-Buku.pdf}. C.V.ANDI; 2005. p.212.
Mengeksplorasi konsep deadlock dan metode untuk mencegahnya:
\begin{itemize}
    \item Deadlock conditions
    \item Deadlock prevention techniques
\end{itemize}

\subsection{User Interface Management}
This section discusses the role of the operating system in managing the user interface. Topics covered include:
\begin{itemize}
    \item Graphical User Interface (GUI)
    \item Command-Line Interface (CLI)
    \item Interaction between the user and the operating system
\end{itemize}

\subsection{Virtualization in Operating Systems}
Virtualization allows multiple operating systems to run concurrently on a single physical machine. This section explores:
\begin{itemize}
    \item Concept of virtualization
    \item Hypervisors and their types
    \item Benefits of virtualization in modern computing
\end{itemize}

\section{Assignments and Practical Work}
\subsection{Assignment 1: Process Scheduling}
Students were tasked with implementing various process scheduling algorithms (e.g., FCFS, SJN, and RR) and comparing their performance under different conditions.
\subsubsection{Group 10}

\begin{enumerate}
    \item No 1

Sebuah sistem memiliki tiga proses dengan waktu kedatangan (arrival time) dan waktu eksekusi (burst time) sebagai berikut:  
\[P1: arrival\_time = 0, burst\_time = 8\]  
\[P2: arrival\_time = 1, burst\_time = 4\]  
\[P3: arrival\_time = 2, burst\_time = 9\]  

Implementasikan algoritma penjadwalan First-Come, First-Served (FCFS) dan hitung waktu penyelesaian (completion time), waktu turnaround (turnaround time), dan waktu tunggu (waiting time) untuk setiap proses.

\textit{JAWABAN : }
\begin{python}
class Process:
    def __init__(self, pid, arrival_time, burst_time):
        self.pid = pid
        self.arrival_time = arrival_time
        self.burst_time = burst_time
        self.completion_time = 0
        self.turnaround_time = 0
        self.waiting_time = 0

def fcfs_scheduling(processes):
    current_time = 0
    
    # Sorting processes based on arrival time
    processes.sort(key=lambda x: x.arrival_time)
    
    for process in processes:
        if current_time < process.arrival_time:
            current_time = process.arrival_time  # Adjust the current time if no process has arrived yet
        process.completion_time = current_time + process.burst_time
        process.turnaround_time = process.completion_time - process.arrival_time
        process.waiting_time = process.turnaround_time - process.burst_time
        current_time = process.completion_time  # Update current time

    return processes

# Problem data
processes = [
    Process("P1", 0, 8),
    Process("P2", 1, 4),
    Process("P3", 2, 9)
]

# FCFS scheduling
scheduled_processes = fcfs_scheduling(processes)

print("PID\tArrival\tBurst\tCompletion\tTurnaround\tWaiting")
for process in scheduled_processes:
    print(f"{process.pid}\t{process.arrival_time}\t{process.burst_time}\t{process.completion_time}\t{process.turnaround_time}\t{process.waiting_time}")
\end{python}

\textit{PENJELASAN : }
\begin{enumerate}
    \item Membuat kelas \textit{Process} yang menyimpan informasi mengenai \textit{pid} (ID proses), \textit{arrival\_time} (waktu kedatangan), \textit{burst\_time} (waktu eksekusi), \textit{completion\_time} (waktu penyelesaian), \textit{turnaround\_time} (waktu turnaround), dan \textit{waiting\_time} (waktu tunggu).
    \item Fungsi \textit{fcfs\_scheduling} melakukan penjadwalan berdasarkan algoritma FCFS. Algoritma ini berjalan sesuai urutan kedatangan proses, dengan langkah-langkah:
        \begin{enumerate}
            \item Mengurutkan proses berdasarkan waktu kedatangan (\textit{arrival\_time}).
            \item Memeriksa apakah \textit{current\_time} kurang dari waktu kedatangan proses, jika ya, memperbarui \textit{current\_time}.
            \item Menghitung \textit{completion\_time}, \textit{turnaround\_time}, dan \textit{waiting\_time}.
        \end{enumerate}
    \item Memasukkan daftar proses dengan waktu kedatangan dan waktu eksekusi sesuai soal, dan menjalankan algoritma FCFS.
    \item Menampilkan hasil dalam format tabel dengan informasi PID, waktu kedatangan, waktu eksekusi, waktu penyelesaian, waktu turnaround, dan waktu tunggu.
\end{enumerate}
\end{enumerate}


\subsection{Assignment 2: Deadlock Handling}
In this assignment, students were asked to simulate different deadlock scenarios and explore various prevention methods.

\subsubsection{Group 10}
\begin{enumerate}



    \item No 1


    
    Sebuah sistem memiliki 5 proses (P1, P2, P3, P4, P5) dan 3 jenis sumber daya (R1, R2, R3). Jumlah setiap sumber daya adalah:  
\[R1: 10 unit, R2: 5 unit, R3: 7 unit\]

Setiap proses membutuhkan berbagai unit dari setiap sumber daya untuk menjalankan tugasnya. Tugas Anda adalah mensimulasikan skenario deadlock dan mengimplementasikan Banker's Algorithm untuk mencegah deadlock dalam sistem ini.

\textit{JAWABAN : }
\begin{python}
import numpy as np

# Function to check if the system is in a safe state
def is_safe(available, allocation, max_need):
    n_processes = len(allocation)
    n_resources = len(available)
    
    work = available.copy()
    finish = [False] * n_processes
    safe_sequence = []
    
    while len(safe_sequence) < n_processes:
        found_process = False
        for i in range(n_processes):
            if not finish[i]:
                # Check if process can proceed
                if all(max_need[i][j] - allocation[i][j] <= work[j] for j in range(n_resources)):
                    work = [work[j] + allocation[i][j] for j in range(n_resources)]
                    finish[i] = True
                    safe_sequence.append(f"P{i+1}")
                    found_process = True
                    break
        if not found_process:
            return False, []
    return True, safe_sequence

# Problem data: allocation and maximum needs for each process
allocation = np.array([[0, 1, 0], [2, 0, 0], [3, 0, 2], [2, 1, 1], [0, 0, 2]])
max_need = np.array([[7, 5, 3], [3, 2, 2], [9, 0, 2], [2, 2, 2], [4, 3, 3]])
available = np.array([3, 3, 2])  # Available resources (R1, R2, R3)

# Check if the system is in a safe state
safe, sequence = is_safe(available, allocation, max_need)

if safe:
    print("Sistem dalam keadaan aman.")
    print("Urutan aman: ", " -> ".join(sequence))
else:
    print("Sistem dalam keadaan deadlock.")
\end{python}

\textit{PENJELASAN : }
\begin{enumerate}
    \item Fungsi \textit{is\_safe} digunakan untuk memeriksa apakah sistem dalam keadaan aman atau tidak. Ini adalah implementasi dari Algoritma Bankir untuk mencegah deadlock.
        \begin{enumerate}
            \item \textit{work} menyimpan jumlah sumber daya yang tersedia pada awalnya.
            \item \textit{finish} adalah array boolean yang menunjukkan apakah suatu proses sudah selesai atau belum.
            \item Algoritma memeriksa apakah setiap proses dapat dilanjutkan berdasarkan alokasi sumber daya yang ada dan kebutuhan maksimum. Jika proses dapat dilanjutkan, alokasinya ditambahkan kembali ke \textit{work}.
            \item Jika tidak ada proses yang dapat dilanjutkan dan urutan aman tidak ditemukan, sistem dianggap berada dalam keadaan deadlock.
        \end{enumerate}
    \item Data masalah terdiri dari:
        \begin{enumerate}
            \item \textit{allocation}: Matriks alokasi saat ini untuk setiap proses.
            \item \textit{max\_need}: Matriks kebutuhan maksimum setiap proses.
            \item \textit{available}: Sumber daya yang tersedia untuk digunakan.
        \end{enumerate}
    \item Algoritma memeriksa apakah keadaan sistem aman, dan jika aman, menampilkan urutan proses yang bisa dijalankan tanpa deadlock.
\end{enumerate}
\end{enumerate}


\subsection{Assignment 3: Multithreading and Amdahl's Law}
This assignment involved designing a multithreading scenario to solve a computationally intensive problem. Students then applied **Amdahl's Law** to calculate the theoretical speedup of the program as the number of threads increased.

\subsubsection{Group 10}
\begin{enumerate}
    \item No 1

    Sebuah program memiliki 60\% bagian yang dapat diparalelkan. Jika program ini dijalankan pada sistem dengan 8 core prosesor, berapakah speedup maksimum yang dapat dicapai menurut Hukum Amdahl?

\textit{JAWABAN : }
\begin{python}
import math

def calculate_speedup(parallel_fraction, num_processors):
    serial_fraction = 1 - parallel_fraction
    return 1 / (serial_fraction + (parallel_fraction / num_processors))

# Problem parameters
parallel_percentage = 60  # 60% can be parallelized
num_cores = 8

# Convert percentage to fraction
parallel_fraction = parallel_percentage / 100

# Calculate speedup
speedup = calculate_speedup(parallel_fraction, num_cores)

print(f"Untuk program dengan {parallel_percentage}% kode yang dapat diparalelkan berjalan pada {num_cores} cores:")
print(f"Kecepatan maksimum menurut Amdahl's Law adalah: {speedup:.2f}x")

# Calculate theoretical maximum speedup (with infinite cores)
max_theoretical_speedup = calculate_speedup(parallel_fraction, math.inf)
print(f"Kecepatan maksimum teoritis (dengan cores tak terbatas) adalah: {max_theoretical_speedup:.2f}x")
\end{python}

\textit{PENJELASAN : }
\begin{enumerate}
    \item Kita mengimpor modul \textit{math} untuk menggunakan konstanta inf (tak terhingga) nanti.
    \item Membuat fungsi \textit{calculate\_speedup} yang menerima dua parameter:
        \begin{enumerate}
            \item \textit{parallel\_fraction}: Bagian program yang dapat diparalelkan (0 sampai 1)
            \item \textit{num\_processors}: Jumlah prosesor atau \textit{thread} yang digunakan
        \end{enumerate}
    \item Implementasi Hukum Amdahl:
        \begin{enumerate}
            \item Ini adalah implementasi langsung dari rumus Hukum Amdahl: \textit{Speedup} = 1 / ((1 - P) + (P / N))
            \item \textit{serial\_fraction} adalah bagian yang tidak dapat diparalelkan (1 - P)
            \item Rumus menghitung \textit{speedup} berdasarkan fraksi paralel dan jumlah prosesor
        \end{enumerate}
    \item Pengaturan parameter masalah :
        \begin{enumerate}
            \item Sesuai soal, 60\% program dapat diparalelkan dan menggunakan 8 \textit{core}
        \end{enumerate}
    \item Konversi persentase ke fraksi :
        \begin{enumerate}
            \item Mengubah 60\% menjadi 0.60 untuk perhitungan
        \end{enumerate}
    \item Perhitungan \textit{speedup} :
        \begin{enumerate}
            \item Memanggil fungsi dengan parameter yang telah ditentukan
        \end{enumerate}
    \item  Mencetak hasil :
        \begin{enumerate}
            \item Menampilkan hasil perhitungan dengan format yang mudah dibaca
            \item .2f memformat angka dengan 2 desimal
        \end{enumerate}
    \item  Perhitungan \textit{speedup} teoritis maksimum :
        \begin{enumerate}
            \item Menghitung \textit{speedup} dengan jumlah prosesor tak terbatas (\textit{math.inf})
            \item Ini menunjukkan batas atas teoritis dari \textit{speedup} yang mungkin dicapai
        \end{enumerate}
    \item Mencetak \textit{speedup} teoritis maksimum :
        \begin{enumerate}
            \item Menampilkan batas atas teoritis \textit{speedup}
        \end{enumerate}


\end{enumerate}

\end{enumerate}

\subsection{Assignment 4: Simple Command-Line Interface (CLI) for User Interface Management}
Students were tasked with creating a simple **CLI** for user interface management. The CLI should support basic commands such as file manipulation (creating, listing, and deleting files), process management, and system status reporting.

\subsubsection{Group 10}


\begin{enumerate}


    \item No 1

    
    Buatlah sebuah **Command-Line Interface (CLI)** sederhana untuk manajemen antarmuka pengguna yang mendukung perintah dasar seperti manipulasi file (membuat, menampilkan daftar, dan menghapus file), manajemen proses, dan pelaporan status sistem.

Berikut adalah contoh implementasi CLI sederhana menggunakan Python.

\textit{JAWABAN : }
\begin{python}
import os
import sys
import psutil

def create_file(filename):
    with open(filename, 'w') as file:
        file.write("")  # Create an empty file
    print(f"File '{filename}' created successfully.")

def list_files():
    files = os.listdir()
    if files:
        print("Files in the current directory:")
        for file in files:
            print(file)
    else:
        print("No files found in the current directory.")

def delete_file(filename):
    if os.path.exists(filename):
        os.remove(filename)
        print(f"File '{filename}' deleted successfully.")
    else:
        print(f"File '{filename}' not found.")

def list_processes():
    processes = psutil.pids()
    print(f"Listing {len(processes)} processes running on the system:")
    for pid in processes[:10]:  # List only the first 10 processes for brevity
        process = psutil.Process(pid)
        print(f"PID: {pid}, Name: {process.name()}")

def system_status():
    print("System Status Report:")
    print(f"CPU Usage: {psutil.cpu_percent()}%")
    print(f"Memory Usage: {psutil.virtual_memory().percent}%")
    print(f"Disk Usage: {psutil.disk_usage('/').percent}%")

def cli_interface():
    while True:
        print("\n--- CLI Menu ---")
        print("1. Create File")
        print("2. List Files")
        print("3. Delete File")
        print("4. List Processes")
        print("5. System Status")
        print("6. Exit")

        choice = input("Enter your choice: ")

        if choice == "1":
            filename = input("Enter the name of the file to create: ")
            create_file(filename)
        elif choice == "2":
            list_files()
        elif choice == "3":
            filename = input("Enter the name of the file to delete: ")
            delete_file(filename)
        elif choice == "4":
            list_processes()
        elif choice == "5":
            system_status()
        elif choice == "6":
            print("Exiting CLI.")
            sys.exit()
        else:
            print("Invalid choice. Please try again.")

if __name__ == "__main__":
    cli_interface()
\end{python}

\textit{PENJELASAN : }
\begin{enumerate}
    \item Modul yang digunakan:
        \begin{enumerate}
            \item \textit{os}: Untuk operasi file dan direktori.
            \item \textit{sys}: Untuk mengontrol eksekusi program (seperti keluar dari program).
            \item \textit{psutil}: Untuk manajemen proses dan pelaporan status sistem.
        \end{enumerate}
    \item Fungsi-fungsi CLI:
        \begin{enumerate}
            \item \textbf{create\_file}: Membuat file baru dengan nama yang ditentukan.
            \item \textbf{list\_files}: Menampilkan daftar semua file di direktori saat ini.
            \item \textbf{delete\_file}: Menghapus file yang ditentukan.
            \item \textbf{list\_processes}: Menampilkan daftar proses yang berjalan di sistem, menggunakan modul \textit{psutil}.
            \item \textbf{system\_status}: Menampilkan laporan status sistem, termasuk penggunaan CPU, memori, dan disk.
        \end{enumerate}
    \item Fungsi utama, \textbf{cli\_interface}, menampilkan menu dan memungkinkan pengguna untuk memilih opsi yang diinginkan. Pengguna dapat:
        \begin{enumerate}
            \item Membuat file baru.
            \item Menampilkan daftar file yang ada di direktori.
            \item Menghapus file tertentu.
            \item Menampilkan daftar proses sistem.
            \item Melihat status sistem (CPU, memori, disk).
            \item Keluar dari program.
        \end{enumerate}
    \item Pengguna akan diberikan prompt untuk memasukkan perintah, dan CLI akan menangani perintah sesuai pilihan.
\end{enumerate}
\end{enumerate}





\subsection{Assignment 5: File System Access}
In this assignment, students implemented file system access routines, including:
\begin{itemize}
    \item File creation and deletion
    \item Reading from and writing to files
    \item Navigating directories and managing file permissions
\end{itemize}
\subsubsection{Group 10}



\textbf{No 1}


    Gunakan Python dan modul bawaan seperti `os` dan `shutil` untuk mengimplementasikan operasi ini. Tulis kode yang akan melakukan hal berikut:
\begin{enumerate}
    \item Membuat sebuah direktori baru bernama \texttt{test\_dir}.
    \item Membuat file baru di dalam direktori tersebut bernama \texttt{example.txt} dan menulis teks ke file itu.
    \item Membaca konten dari file \texttt{example.txt}.
    \item Menghapus file \texttt{example.txt}.
    \item Mengatur izin file dalam direktori \texttt{test\_dir} untuk membuat file hanya dapat dibaca.
\end{enumerate}
\textit{JAWABAN :}
\begin{python}
import os
import shutil

def create_directory(directory_name):
    if not os.path.exists(directory_name):
        os.makedirs(directory_name)
        print(f"Directory '{directory_name}' created.")
    else:
        print(f"Directory '{directory_name}' already exists.")

def create_file(file_path, content):
    with open(file_path, 'w') as file:
        file.write(content)
        print(f"File '{file_path}' created and written.")

def read_file(file_path):
    if os.path.exists(file_path):
        with open(file_path, 'r') as file:
            content = file.read()
            print(f"Reading from '{file_path}':\n{content}")
    else:
        print(f"File '{file_path}' does not exist.")

def delete_file(file_path):
    if os.path.exists(file_path):
        os.remove(file_path)
        print(f"File '{file_path}' deleted.")
    else:
        print(f"File '{file_path}' does not exist.")

def change_permissions(directory_path):
    if os.path.exists(directory_path):
        # Set read-only permission to the directory and its contents
        os.chmod(directory_path, 0o444)
        print(f"Permissions for directory '{directory_path}' set to read-only.")
    else:
        print(f"Directory '{directory_path}' does not exist.")

# Assignment steps implementation
dir_name = 'test_dir'
file_name = 'example.txt'
file_path = os.path.join(dir_name, file_name)

# Step 1: Create a directory
create_directory(dir_name)

# Step 2: Create a file and write to it
create_file(file_path, "This is a test file for file system access assignment.")

# Step 3: Read the file
read_file(file_path)

# Step 4: Delete the file
delete_file(file_path)

# Step 5: Change permissions of the directory to read-only
change_permissions(dir_name)
\end{python}

\textit{PENJELASAN :}
\begin{enumerate}
    \item Fungsi \textit{create\_directory} digunakan untuk membuat direktori baru dengan nama yang ditentukan jika direktori tersebut belum ada.
    \item Fungsi \textit{create\_file} membuka file dalam mode tulis dan menulis teks yang diberikan ke file. Jika file belum ada, maka akan dibuat.
    \item Fungsi \textit{read\_file} membaca dan menampilkan isi file jika file ada. Ini menggunakan metode pembacaan standar dalam Python.
    \item Fungsi \textit{delete\_file} digunakan untuk menghapus file jika file tersebut ada di sistem.
    \item Fungsi \textit{change\_permissions} mengatur izin file atau direktori menjadi hanya-baca (\textit{read-only}) menggunakan `os.chmod()`.

\end{enumerate}
    Program di atas menciptakan direktori baru bernama `test\_dir`, membuat file `example.txt` di dalamnya, menulis teks ke dalam file, membaca isi file, menghapus file, dan mengatur izin direktori agar menjadi hanya-baca.



\section{Conclusion}
The first half of the course introduced core operating system concepts, including process management, scheduling, multithreading, and file system access. These topics provided a foundation for more advanced topics to be covered in the second half of the course.

\end{document}