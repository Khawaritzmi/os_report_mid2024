\documentclass[12pt]{article}
\usepackage{amsmath}
\usepackage{graphicx}
\usepackage{hyperref}
\usepackage{listings}
\usepackage{color}
\usepackage{pythonhighlight}

\title{Operating System Course Report - First Half of the Semester}
\author{B class}
\date{\today}

\begin{document}

\maketitle
\newpage

\tableofcontents
\newpage

\section{Introduction}
This report summarizes the topics covered during the first half of the Operating System course. It includes theoretical concepts, practical implementations, and assignments. The course focuses on the fundamentals of operating systems, including system architecture, process management, CPU scheduling, and deadlock handling.

\section{Course Overview}
\subsection{Objectives}
The main objectives of this course are:
\begin{itemize}
    \item To understand the basic components and architecture of a computer system.
    \item To learn process management, scheduling, and inter-process communication.
    \item To explore file systems, input/output management, and virtualization.
    \item To study the prevention and handling of deadlocks in operating systems.
\end{itemize}

\subsection{Course Structure}
The course is divided into two halves. This report focuses on the first half, which covers:
\begin{itemize}
    \item Basic Concepts and Components of Computer Systems
    \item System Performance and Metrics
    \item System Architecture of Computer Systems
    \item Process Description and Control
    \item Scheduling Algorithms
    \item Process Creation and Termination
    \item Introduction to Threads
    \item File Systems
    \item Input and Output Management
    \item Deadlock Introduction and Prevention
    \item User Interface Management
    \item Virtualization in Operating Systems
\end{itemize}

\section{Topics Covered}

\subsection{Basic Concepts and Components of Computer Systems}
This section explains the fundamental components that make up a computer system, including the CPU, memory, storage, and input/output devices.

\subsection{System Performance and Metrics}
This section introduces various system performance metrics used to measure the efficiency of a computer system, including throughput, response time, and utilization.

\subsection{System Architecture of Computer Systems}
Describes the architecture of modern computer systems, focusing on the interaction between hardware and the operating system.

\subsection{Process Description and Control}
Processes are a central concept in operating systems. This section covers:
\begin{itemize}
    \item Process states and state transitions
    \item Process control block (PCB)
    \item Context switching
\end{itemize}

\subsection{Scheduling Algorithms}
This section covers:
\begin{itemize}
    \item First-Come, First-Served (FCFS)
    \item Shortest Job Next (SJN)
    \item Round Robin (RR)
\end{itemize}
It explains how these algorithms are used to allocate CPU time to processes.

\subsection{Process Creation and Termination}
Details how processes are created and terminated by the operating system, including:
\begin{itemize}
    \item Process spawning
    \item Process termination conditions
    
    \subsubsection{Terminasi Normal Proses}
    
    \begin{itemize}
        Proses terminasi normal adalah kondisi di mana suatu proses menyelesaikan fungsinya dengan baik tanpa mengalami kesalahan atau pengecualian. Ini adalah akhir eksekusi proses yang terkontrol dan disengaja. Dalam kode program, biasanya terdapat instruksi atau pernyataan yang menunjukkan akhir dari proses, seperti \textit{`return`} di fungsi utama dalam bahasa C atau Java. Proses ini merupakan bagian dari siklus hidup normal sebuah proses.
    \end{itemize}

    \begin{figure}
        \centering
        \includegraphics[width=0.2\linewidth]{asset/Normal.png}
        \caption{Terminasi Normal}
        \label{fig:enter-label}
    \end{figure}
    
    Skenario umum untuk terminasi normal:
    \begin{itemize} 
        \item Penyelesaian proses: Proses telah menyelesaikan semua tugasnya dan mencapai titik akhir secara alami.
        \item Terminasi yang diinisiasi oleh pengguna: Pengguna secara eksplisit meminta proses untuk berhenti, biasanya melalui perintah atau antarmuka.
        \item Pengembalian dari fungsi: Sebuah proses dapat berhenti ketika sebuah fungsi yang dijalankan mengembalikan nilai yang menandakan penyelesaian.
    \end{itemize}
    
    Karakteristik terminasi normal:
    \begin{itemize} 
        \item Proses berakhir secara teratur dan dapat diprediksi.
        \item Tidak ada kesalahan atau pengecualian yang tidak terduga yang menyebabkan proses berhenti sebelum waktunya.
        \item Proses melepaskan semua sumber daya yang digunakannya (misalnya, memori, file, koneksi jaringan) sebelum terminasi.
    \end{itemize}
    
    Contoh terminasi normal:
    \begin{itemize}
        \item Sebuah proses peramban web menyelesaikan pemuatan halaman web.
        \item Sebuah proses editor teks menyimpan dokumen dan kemudian keluar.
        \item Sebuah tugas latar belakang menyelesaikan pekerjaan yang dijadwalkan dan berakhir.
    \end{itemize}
    
    \subsubsection{Terminasi Tidak Normal Proses}
    
    Terminasi tidak normal terjadi ketika sebuah proses berhenti secara tidak teratur akibat adanya kesalahan yang tidak dapat ditangani dengan baik. Hal ini sering kali disebabkan oleh kondisi tak terduga yang membuat proses tidak dapat melanjutkan eksekusi.
    
        \begin{figure}
            \centering
            \includegraphics[width=0.5\linewidth]{asset/Abnormal.png}
            \caption{Terminasi Tidak Normal}
            \label{Abnormal}
        \end{figure}
    
    Penyebab terminasi tidak normal:
    \begin{itemize}
        \item Kesalahan logika: Kesalahan dalam kode program, seperti pembagian dengan nol, akses array di luar batas, atau penggunaan pointer yang tidak valid.
        \item Kehabisan sumber daya: Proses kehabisan sumber daya yang dibutuhkan untuk beroperasi, seperti memori, deskriptor file, atau waktu CPU.
        \item Sinyal: Proses menerima sinyal yang memaksanya untuk berhenti, seperti sinyal\textit{ SIGKILL} (berhenti segera) atau \textit{SIGSEGV} (kesalahan segmentasi).
        \item Kesalahan perangkat keras: Kerusakan perangkat keras dapat menyebabkan proses berhenti secara tiba-tiba.
        \item \textit{Deadlock}: Dua atau lebih proses saling menunggu satu sama lain, sehingga tidak ada yang dapat melanjutkan eksekusi.
        \item Interupsi: Terjadinya interupsi yang tidak dapat ditangani oleh sistem operasi.
    \end{itemize}
    
    Kontras dengan terminasi tidak normal:
    \begin{itemize}
        \item Terminasi tidak normal terjadi ketika sebuah proses berhenti secara tak terduga karena kesalahan, pengecualian, atau keadaan tak terduga lainnya. Hal ini dapat menyebabkan perilaku tidak terkendali, kebocoran sumber daya, atau ketidakstabilan sistem.
    \end{itemize}
    
    Dampak terminasi tidak normal:
    \begin{itemize}
        \item Kehilangan data: Data yang sedang diproses oleh proses yang berhenti secara tidak normal mungkin hilang atau rusak.
        \item Ketidakstabilan sistem: Terminasi tidak normal dapat menyebabkan sistem menjadi tidak stabil atau bahkan \textit{ crash.}
        \item Kerusakan data: Data yang disimpan di disk atau memori dapat rusak jika proses tidak sempat menyimpan data dengan benar sebelum berhenti.
        \item Gangguan proses lain: Terminasi tidak normal suatu proses dapat mempengaruhi proses lain yang berinteraksi dengannya.
    \end{itemize}
    
    Mekanisme penanganan terminasi tidak normal:
    \begin{itemize}
        Sistem operasi: Sistem operasi berusaha mendeteksi dan menangani terminasi tidak normal. Ini melibatkan mekanisme seperti:
    \end{itemize}
    
    \begin{itemize}
        \item Penanganan pengecualian: Sistem operasi menangkap pengecualian yang terjadi selama eksekusi proses dan mengambil tindakan yang sesuai, seperti menghentikan proses atau menampilkan pesan kesalahan.
        \item Pengawasan memori: Sistem operasi memantau penggunaan memori oleh proses untuk mencegah akses memori yang tidak sah.
        \item Pengawasan waktu: Sistem operasi membatasi waktu eksekusi suatu proses untuk mencegah proses berjalan terlalu lama dan menghabiskan sumber daya sistem.
        \item \textit{Debugging}: Pengembang dapat menggunakan debugger untuk melacak kesalahan yang menyebabkan terminasi tidak normal dan memperbaiki kode program.
    \end{itemize}
    
    Contoh terminasi tidak normal:
    \begin{itemize}
        \item Program \textit{crash}: Sebuah program tiba-tiba berhenti bekerja dan menampilkan pesan kesalahan.
        \item \textit{Blue Screen of Death (BSOD)}: Sistem operasi Windows mengalami kegagalan fatal dan menampilkan layar biru.
        \item \textit{Kernel panic}: Sistem operasi Linux mengalami kegagalan yang sangat serius dan tidak dapat melanjutkan operasi.
    \end{itemize}
    
    Pencegahan terminasi tidak normal:
    \begin{itemize}
        \item Pengujian yang memadai: Melakukan pengujian menyeluruh pada perangkat lunak untuk menemukan dan memperbaiki \textit{bug} sebelum di-deploy.
        \item Penanganan pengecualian: Menulis kode yang dapat menangani berbagai jenis pengecualian yang mungkin terjadi.
        \item Manajemen memori yang baik: Memastikan alokasi dan dealokasi memori dilakukan dengan benar untuk mencegah kebocoran memori.
        \item Validasi input: Memeriksa semua input pengguna untuk memastikan bahwa input tersebut valid dan tidak menyebabkan kesalahan.
    \end{itemize}
    
\end{itemize}
    
    \begin{itemize}
    \item {Wahab, A., & Mubarok, R. (2019). Pengenalan Sistem Operasi (Edisi Revisi). Penerbit Andi.}
    \item {Hidayat, M., & Santoso, A. (2018). "Analisis Penanganan Error dan Exception Handling Pada Bahasa Pemrograman Java." Jurnal Ilmu Komputer dan Teknologi Informasi, 5(1), 55-62.}
    \end{itemize}


\subsection{Introduction to Threads}
This section introduces the concept of threads and their relation to processes, covering:
\begin{itemize}
    \item Single-threaded vs. multi-threaded processes
    \item Benefits of multithreading
\end{itemize}

\begin{figure}[h]
    \centering
    \includegraphics[width=0.5\textwidth]{/Users/khawaritzmi/Unhas/os_report_mid2024/b_class/asset/example.png}  % Sesuaikan nama file dan ukurannya
    \caption{Ini adalah gambar contoh dari multithreading.}
    \label{fig:contoh_gambar}
\end{figure}

Seperti yang terlihat pada Gambar \ref{fig:contoh_gambar}, inilah cara menambahkan gambar dengan keterangan.

\subsection{File Systems}
File systems provide a way for the operating system to store, retrieve, and manage data. This section explains:
\begin{itemize}
    \item File system structure
    \item File access methods
    \item Directory management
\end{itemize}

\subsection{Input and Output Management}
Input and output management is key for handling the interaction between the system and external devices. This section includes:
\begin{itemize}
    \item Device drivers
    \item I/O scheduling
\end{itemize}

\subsection{Deadlock Introduction and Prevention}
Explores the concept of deadlocks and methods for preventing them:
\begin{itemize}
    \item Deadlock conditions
    \item Deadlock prevention techniques
\end{itemize}

\subsection{User Interface Management}
This section discusses the role of the operating system in managing the user interface. Topics covered include:
\begin{itemize}
    \item Graphical User Interface (GUI)
    \item Command-Line Interface (CLI)
    \item Interaction between the user and the operating system
\end{itemize}

\subsection{Virtualization in Operating Systems}
Virtualization allows multiple operating systems to run concurrently on a single physical machine. This section explores:
\begin{itemize}
    \item Concept of virtualization
    \item Hypervisors and their types
    \item Benefits of virtualization in modern computing
\end{itemize}

\section{Assignments and Practical Work}
\subsection{Assignment 1: Process Scheduling}
Students were tasked with implementing various process scheduling algorithms (e.g., FCFS, SJN, and RR) and comparing their performance under different conditions.
\subsubsection{Group 1}
\begin{python}
    class Process:
    def __init__(self, pid, arrival_time, burst_time):
        self.pid = pid
        self.arrival_time = arrival_time
        self.burst_time = burst_time
        self.completion_time = 0
        self.turnaround_time = 0
        self.waiting_time = 0
\end{python}

\begin{table}[htbp] % Optional: For floating position
    \centering
    \begin{tabular}{|c|c|c|} % Defines number of columns and alignment (c = center, l = left, r = right). '|' creates vertical lines.
    \hline
    Header 1 & Header 2 & Header 3 \\ % Column headers
    \hline
    Row 1, Column 1 & Row 1, Column 2 & Row 1, Column 3 \\ % First row of data
    \hline
    Row 2, Column 1 & Row 2, Column 2 & Row 2, Column 3 \\ % Second row of data
    \hline
    \end{tabular}
    \caption{Your table caption} % Optional: For adding a caption
    \label{tab:your_label} % Optional: For cross-referencing the table
\end{table}

\subsection{Assignment 2: Deadlock Handling}
In this assignment, students were asked to simulate different deadlock scenarios and explore various prevention methods.

\subsection{Assignment 3: Multithreading and Amdahl's Law}
This assignment involved designing a multithreading scenario to solve a computationally intensive problem. Students then applied **Amdahl's Law** to calculate the theoretical speedup of the program as the number of threads increased.

\subsection{Assignment 4: Simple Command-Line Interface (CLI) for User Interface Management}
Students were tasked with creating a simple **CLI** for user interface management. The CLI should support basic commands such as file manipulation (creating, listing, and deleting files), process management, and system status reporting.

\subsection{Assignment 5: File System Access}
In this assignment, students implemented file system access routines, including:
\begin{itemize}
    \item File creation and deletion
    \item Reading from and writing to files
    \item Navigating directories and managing file permissions
\end{itemize}

\subsubsection{Group 6}

\begin{itemize}
    \item File creation and deletion
\end{itemize}


Tulis sebuah program Python yang membuat file bernama \textit{example.txt}, menuliskan string \textit{"Hello, World!"} ke dalam file tersebut, kemudian membaca isinya dan menghapus file tersebut.

\begin{python}
    import os

    # Membuat file dan menulis ke dalamnya
    with open("example.txt", "w") as file:
        file.write("Hello, World!")

    # Membaca isi file
    with open("example.txt", "r") as file:
        content = file.read()

        print("Isi file:", content)

    # Menghapus file
    if os.path.exists("example.txt"):
        os.remove("example.txt")
        print("File berhasil dihapus.")
    else:
        print("File tidak ditemukan.")
\end{python}


\begin{itemize}
    \item Reading from and writing to files
\end{itemize}
Tulis sebuah program Python yang membaca data dari file input.txt dan menuliskan data tersebut ke file baru bernama output.txt.

\begin{python}
    # Membuat file input.txt dan menulis data ke dalamnya
    with open("input.txt", "w") as infile:
        infile.write("Ini adalah data yang akan disalin ke output.txt.")

    # Membaca isi file input.txt
    with open("input.txt", "r") as infile:
        data = infile.read()

    # Menulis data ke file output.txt
    with open("output.txt", "w") as outfile:
        outfile.write(data)

    print("Data berhasil dipindahkan dari 'input.txt' ke 'output.txt'.")
\end{python}

\begin{itemize}
    \item Navigating directories and managing file permissions
\end{itemize}

Tulis sebuah program Python yang membuat sebuah direktori bernama \textit{new directory}, kemudian mengatur izin file dari example.txt menjadi hanya dapat dibaca \textit{(read-only)}.

\begin{python}
    import os

    # Membuat direktori baru
    os.makedirs("new_directory", exist_ok=True)
        print("Direktori 'new_directory' berhasil dibuat.")

    # Membuat file example.txt di dalam direktori baru
    file_path = os.path.join("new_directory", "example.txt")
    with open(file_path, "w") as file:
        file.write("This is a test file.")

    # Mengatur izin file menjadi read-only
    os.chmod(file_path, 0o444)  # 0o444 adalah izin untuk read-only
        print(f"Izin file '{file_path}' diatur menjadi read-only.")
\end{python}


\section{Conclusion}
The first half of the course introduced core operating system concepts, including process management, scheduling, multithreading, and file system access. These topics provided a foundation for more advanced topics to be covered in the second half of the course.

\end{document}