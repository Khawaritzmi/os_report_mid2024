\documentclass[12pt]{article}
\usepackage{amsmath}
\usepackage{graphicx}
\usepackage{hyperref}
\usepackage{listings}
\usepackage{color}
\usepackage{pythonhighlight}

\title{Operating System Course Report - First Half of the Semester}
\author{B class}
\date{\today}

\begin{document}

\maketitle
\newpage

\tableofcontents
\newpage

\section{Introduction}
This report summarizes the topics covered during the first half of the Operating System course. It includes theoretical concepts, practical implementations, and assignments. The course focuses on the fundamentals of operating systems, including system architecture, process management, CPU scheduling, and deadlock handling.

\section{Course Overview}
\subsection{Objectives}
The main objectives of this course are:
\begin{itemize}
    \item To understand the basic components and architecture of a computer system.
    \item To learn process management, scheduling, and inter-process communication.
    \item To explore file systems, input/output management, and virtualization.
    \item To study the prevention and handling of deadlocks in operating systems.
\end{itemize}

\subsection{Course Structure}
The course is divided into two halves. This report focuses on the first half, which covers:
\begin{itemize}
    \item Basic Concepts and Components of Computer Systems
    \item System Performance and Metrics
    \item System Architecture of Computer Systems
    \item Process Description and Control
    \item Scheduling Algorithms
    \item Process Creation and Termination
    \item Introduction to Threads
    \item File Systems
    \item Input and Output Management
    \item Deadlock Introduction and Prevention
    \item User Interface Management
    \item Virtualization in Operating Systems
\end{itemize}

\section{Topics Covered}

\subsection{Basic Concepts and Components of Computer Systems}
This section explains the fundamental components that make up a computer system, including the CPU, memory, storage, and input/output devices.

\subsection{System Performance and Metrics}
This section introduces various system performance metrics used to measure the efficiency of a computer system, including throughput, response time, and utilization.

\subsection{System Architecture of Computer Systems}
Describes the architecture of modern computer systems, focusing on the interaction between hardware and the operating system.

\subsection{Process Description and Control}
Processes are a central concept in operating systems. This section covers:
\begin{itemize}
    \item Process states and state transitions
    \item Process control block (PCB)
    \item Context switching
\end{itemize}

\subsection{Scheduling Algorithms}
This section covers:
\begin{itemize}
    \item First-Come, First-Served (FCFS)
    \item Shortest Job Next (SJN)
    \item Round Robin (RR)
\end{itemize}
It explains how these algorithms are used to allocate CPU time to processes.

\subsection{Process Creation and Termination}
Details how processes are created and terminated by the operating system, including:
\begin{itemize}
    \item Process spawning
    \item Process termination conditions
\end{itemize}

\subsection{Introduction to Threads}
    \subsubsection{Definisi Process and Thread}
    \begin{itemize}
            \item\textbf{Proses} 
            \par \hspace{2em} Process adalah unit eksekusi independen dalam sistem operasi yang mencakup ruang memori, sumber daya, dan status eksekusinya sendiri. Setiap proses dihasilkan dari eksekusi suatu program dan berjalan secara terpisah dengan proses lainnya. Karena memiliki ruang memori dan sumber daya sendiri, proses tidak saling berbagi data secara langsung kecuali menggunakan mekanisme komunikasi antar-proses (IPC). Proses bertanggung jawab untuk menjalankan instruksi program, dan sistem operasi mengelola eksekusi, alokasi sumber daya, dan status dari setiap proses. 
            
            \item\textbf{Thread}
            \par \hspace{2em} Thread adalah unit eksekusi yang lebih ringan dibandingkan proses. Thread merupakan bagian dari suatu proses yang lebih besar, dan beberapa thread dapat berjalan secara bersamaan dalam satu proses. Tidak seperti proses, thread berbagi memori dan sumber daya dengan thread lain di dalam proses yang sama, sehingga lebih efisien dalam penggunaan sumber daya. Karena berbagi ruang memori, thread dapat berkomunikasi lebih cepat dengan thread lain dalam satu proses, namun juga meningkatkan risiko konflik data. Thread sering digunakan untuk tugas-tugas yang membutuhkan pemrosesan paralel dalam aplikasi.
            \end{itemize}
            
    \subsubsection{Process vs Threads}
    \begin{enumerate}
        \item\textbf{Memori}
        \begin{itemize}
            \item\textbf{Proses} 
            \par \hspace{2em} Setiap proses memiliki ruang alamat memori sendiri yang terpisah (dikenal sebagai isolasi memori). Proses tidak dapat langsung mengakses memori dari proses lain kecuali dengan mekanisme komunikasi antar proses (IPC). Karena setiap proses memiliki ruang memori sendiri, switching antar proses membutuhkan lebih banyak memori karena data seperti state, kode, dan data proses disimpan secara terpisah untuk setiap proses.
            \item\textbf{Thread}
            \par \hspace{2em} Thread berbagi ruang alamat memori dengan proses induknya. Semua thread dalam satu proses berbagi kode, data, dan file descriptors yang sama. Karena thread berbagi memori, mereka dapat dengan mudah berkomunikasi satu sama lain tanpa perlu IPC yang rumit, tetapi ini juga menyebabkan risiko race condition atau deadlock jika tidak dikelola dengan baik.
        \end{itemize}
        \item\textbf{Overhead} 
        \begin{itemize}
            \item\textbf{Proses} 
            \par \hspace{2em} Proses memiliki overhead yang lebih besar dibandingkan thread. Proses mencakup informasi lengkap seperti ruang alamat terpisah, tabel proses, dan lain-lain. Context switching antara proses membutuhkan lebih banyak sumber daya dan waktu karena melibatkan penyimpanan dan pemulihan state dari proses yang berbeda, yang mencakup informasi memori, register, dan lain-lain.
            \item\textbf{Thread}
            \par \hspace{2em} Thread memiliki overhead yang lebih rendah dibandingkan proses. Switching antar thread dalam satu proses lebih cepat karena thread berbagi sebagian besar sumber daya (seperti memori).Thread hanya perlu menyimpan dan memulihkan informasi minimal (misalnya, register CPU dan pointer stack), sehingga context switching antar thread jauh lebih ringan dibandingkan dengan antar proses.
        \end{itemize}
        \item\textbf{Eksekusi}
        \begin{itemize}
            \item\textbf{Proses} 
            \par \hspace{2em} Setiap proses adalah entitas independen yang dijalankan secara terpisah di CPU. Sebuah proses dapat terdiri dari satu atau lebih thread. Komunikasi antar proses lebih rumit karena proses memiliki ruang memori terpisah, sehingga diperlukan mekanisme Inter-Process Communication (IPC) seperti pipes, message queues, atau shared memory.
            \item\textbf{Thread}
            \par \hspace{2em} Thread merupakan unit eksekusi yang lebih kecil dalam sebuah proses. Semua thread dalam satu proses dijalankan secara bersamaan dan berbagi sumber daya yang sama. Komunikasi antar thread lebih efisien karena mereka dapat mengakses data bersama secara langsung dalam ruang memori yang sama, tetapi harus berhati-hati dengan sinkronisasi untuk menghindari kesalahan (seperti race conditions).
        \end{itemize}
        \item\textbf{Manajemen Sumber Daya}
        \begin{itemize}
            \item\textbf{Proses} 
            \par \hspace{2em} Setiap proses memiliki sumber dayanya sendiri, termasuk memori, file handles, dan state CPU. Manajemen sumber daya lebih kompleks dan sering membutuhkan operasi yang lebih berat seperti pembagian sumber daya atau komunikasi antar proses. Karena proses memiliki lingkungan eksekusi yang terisolasi, kesalahan pada satu proses biasanya tidak memengaruhi proses lain.
            \item\textbf{Thread}
            \par \hspace{2em} Thread berbagi sumber daya dengan thread lain dalam proses yang sama. Mereka berbagi memori, file handles, dan lainnya. Ini membuat manajemen sumber daya antar thread lebih efisien, tetapi juga lebih rentan terhadap konflik jika tidak dikelola dengan baik (misalnya, melalui penggunaan mutexes, semaphores, atau mekanisme sinkronisasi lainnya). Kesalahan dalam satu thread dapat memengaruhi thread lain dalam proses yang sama karena mereka berbagi memori dan sumber daya lainnya.
        \end{itemize}
    \end{enumerate}

\subsection{File Systems}
File systems provide a way for the operating system to store, retrieve, and manage data. This section explains:
\begin{itemize}
    \item File system structure
    \item File access methods
    \item Directory management
\end{itemize}

\subsection{Input and Output Management}
Input and output management is key for handling the interaction between the system and external devices. This section includes:
\begin{itemize}
    \item Device drivers
    \item I/O scheduling
\end{itemize}

\subsection{Deadlock Introduction and Prevention}
Explores the concept of deadlocks and methods for preventing them:
\begin{itemize}
    \item Deadlock conditions
    \item Deadlock prevention techniques
\end{itemize}

\subsection{User Interface Management}
This section discusses the role of the operating system in managing the user interface. Topics covered include:
\begin{itemize}
    \item Graphical User Interface (GUI)
    \item Command-Line Interface (CLI)
    \item Interaction between the user and the operating system
\end{itemize}

\subsection{Virtualization in Operating Systems}
Virtualization allows multiple operating systems to run concurrently on a single physical machine. This section explores:
\begin{itemize}
    \item Concept of virtualization
    \item Hypervisors and their types
    \item Benefits of virtualization in modern computing
\end{itemize}

\section{Assignments and Practical Work}
\subsection{Assignment 1: Process Scheduling}
Students were tasked with implementing various process scheduling algorithms (e.g., FCFS, SJN, and RR) and comparing their performance under different conditions.
\subsubsection{Group 1}
\begin{python}
    class Process:
    def __init__(self, pid, arrival_time, burst_time):
        self.pid = pid
        self.arrival_time = arrival_time
        self.burst_time = burst_time
        self.completion_time = 0
        self.turnaround_time = 0
        self.waiting_time = 0
\end{python}

\begin{table}[htbp] % Optional: For floating position
    \centering
    \begin{tabular}{|c|c|c|} % Defines number of columns and alignment (c = center, l = left, r = right). '|' creates vertical lines.
    \hline
    Header 1 & Header 2 & Header 3 \\ % Column headers
    \hline
    Row 1, Column 1 & Row 1, Column 2 & Row 1, Column 3 \\ % First row of data
    \hline
    Row 2, Column 1 & Row 2, Column 2 & Row 2, Column 3 \\ % Second row of data
    \hline
    \end{tabular}
    \caption{Your table caption} % Optional: For adding a caption
    \label{tab:your_label} % Optional: For cross-referencing the table
\end{table}

\subsection{Assignment 2: Deadlock Handling}
In this assignment, students were asked to simulate different deadlock scenarios and explore various prevention methods.

\subsection{Assignment 3: Multithreading and Amdahl's Law}
This assignment involved designing a multithreading scenario to solve a computationally intensive problem. Students then applied **Amdahl's Law** to calculate the theoretical speedup of the program as the number of threads increased.

\subsection{Assignment 4: Simple Command-Line Interface (CLI) for User Interface Management}
Students were tasked with creating a simple **CLI** for user interface management. The CLI should support basic commands such as file manipulation (creating, listing, and deleting files), process management, and system status reporting.

\subsection{Assignment 5: File System Access}
In this assignment, students implemented file system access routines, including:
\begin{itemize}
    \item File creation and deletion
    \item Reading from and writing to files
    \item Navigating directories and managing file permissions
\end{itemize}

\section{Conclusion}
The first half of the course introduced core operating system concepts, including process management, scheduling, multithreading, and file system access. These topics provided a foundation for more advanced topics to be covered in the second half of the course.

\end{document}