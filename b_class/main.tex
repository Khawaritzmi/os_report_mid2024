\documentclass[12pt]{article}
\usepackage{amsmath}
\usepackage{graphicx}
\usepackage{hyperref}
\usepackage{listings}
\usepackage{color}
\usepackage{pythonhighlight}

\title{Operating System Course Report - First Half of the Semester}
\author{B class}
\date{\today}

\begin{document}

\maketitle
\newpage

\tableofcontents
\newpage

\section{Introduction}
This report summarizes the topics covered during the first half of the Operating System course. It includes theoretical concepts, practical implementations, and assignments. The course focuses on the fundamentals of operating systems, including system architecture, process management, CPU scheduling, and deadlock handling.

\section{Course Overview}
\subsection{Objectives}
The main objectives of this course are:
\begin{itemize}
    \item To understand the basic components and architecture of a computer system.
    \item To learn process management, scheduling, and inter-process communication.
    \item To explore file systems, input/output management, and virtualization.
    \item To study the prevention and handling of deadlocks in operating systems.
\end{itemize}

\subsection{Course Structure}
The course is divided into two halves. This report focuses on the first half, which covers:
\begin{itemize}
    \item Basic Concepts and Components of Computer Systems
    \item System Performance and Metrics
    \item System Architecture of Computer Systems
    \item Process Description and Control
    \item Scheduling Algorithms
    \item Process Creation and Termination
    \item Introduction to Threads
    \item File Systems
    \item Input and Output Management
    \item Deadlock Introduction and Prevention
    \item User Interface Management
    \item Virtualization in Operating Systems
\end{itemize}

\section{Topics Covered}

\subsection{Basic Concepts and Components of Computer Systems}
This section explains the fundamental components that make up a computer system, including the CPU, memory, storage, and input/output devices.

\subsection{System Performance and Metrics}
This section introduces various system performance metrics used to measure the efficiency of a computer system, including throughput, response time, and utilization.

\subsection{System Architecture of Computer Systems}
Describes the architecture of modern computer systems, focusing on the interaction between hardware and the operating system.

\subsection{Process Description and Control}
Processes are a central concept in operating systems. This section covers:
\begin{itemize}
    \item Process states and state transitions
    \item Process control block (PCB)
    \item Context switching
\end{itemize}

\subsection{Scheduling Algorithms}
This section covers:
\begin{itemize}
    \item First-Come, First-Served (FCFS)
    \item Shortest Job Next (SJN)
    \item Round Robin (RR)
\end{itemize}
It explains how these algorithms are used to allocate CPU time to processes.

\subsection{Process Creation and Termination}
Details how processes are created and terminated by the operating system, including:
\begin{itemize}
    \item Process spawning
    \item Process termination conditions
\end{itemize}

\subsection{Introduction to Threads}
This section introduces the concept of threads and their relation to processes, covering:
\subsubsection{Single-Threading}
\begin{itemize}
    \item\textbf{Definisi} 
    \par \hspace{2em} Single-Threading adalah model pemrograman di mana hanya satu thread atau alur eksekusi dijalankan dalam suatu program pada satu waktu. Dalam single-threading, satu instruksi dijalankan setelah yang lain secara berurutan. Tidak ada eksekusi paralel, sehingga semua tugas diselesaikan satu per satu, dan jika satu tugas memakan waktu lama, seluruh aplikasi akan menunggu hingga tugas tersebut selesai sebelum melanjutkan ke yang berikutnya.
    \item\textbf{Contoh}
    \par \hspace{2em} Sebuah aplikasi kalkulator sederhana: Setiap operasi (seperti penjumlahan atau pengurangan) 	dilakukan satu per satu, dan tidak ada operasi lain yang dapat berjalan sampai operasi sebelumnya selesai. Jika pengguna meminta hasil kalkulasi yang rumit, aplikasi akan menjadi tidak responsif hingga proses kalkulasi selesai.
\end{itemize}

\subsubsection{Multi-Threading}
\begin{itemize}
    \item\textbf{Definisi}
    \par \hspace{2em} Multi-Threading adalah model pemrograman di mana suatu proses dapat dibagi menjadi beberapa thread atau alur eksekusi yang berjalan secara bersamaan. Setiap thread dapat menjalankan tugas yang berbeda secara paralel dalam satu proses, sehingga program dapat menjalankan banyak operasi secara bersamaan, mempercepat kinerja dan responsivitas.
    \item\textbf{Contoh} 
    \par \hspace{2em} Browser Web: Saat membuka tab baru di browser, setiap tab mungkin berjalan dalam thread terpisah. Ini memungkinkan Anda untuk memuat halaman web di satu tab tanpa mengganggu tab lain. Jika satu halaman memerlukan waktu lama untuk memuat, tab lain tetap bisa digunakan tanpa masalah.
\end{itemize}

\subsection{File Systems}
File systems provide a way for the operating system to store, retrieve, and manage data. This section explains:
\begin{itemize}
    \item File system structure
    \item File access methods
    \item Directory management
\end{itemize}

\subsection{Input and Output Management}
Input and output management is key for handling the interaction between the system and external devices. This section includes:
\begin{itemize}
    \item Device drivers
    \item I/O scheduling
\end{itemize}

\subsection{Deadlock Introduction and Prevention}
Explores the concept of deadlocks and methods for preventing them:
\begin{itemize}
    \item Deadlock conditions
    \item Deadlock prevention techniques
\end{itemize}

\subsection{User Interface Management}
This section discusses the role of the operating system in managing the user interface. Topics covered include:
\begin{itemize}
    \item Graphical User Interface (GUI)
    \item Command-Line Interface (CLI)
    \item Interaction between the user and the operating system
\end{itemize}

\subsection{Virtualization in Operating Systems}
Virtualization allows multiple operating systems to run concurrently on a single physical machine. This section explores:
\begin{itemize}
    \item Concept of virtualization
    \item Hypervisors and their types
    \item Benefits of virtualization in modern computing
\end{itemize}

\section{Assignments and Practical Work}
\subsection{Assignment 1: Process Scheduling}
Students were tasked with implementing various process scheduling algorithms (e.g., FCFS, SJN, and RR) and comparing their performance under different conditions.
\subsubsection{Group 1}
\begin{python}
    class Process:
    def __init__(self, pid, arrival_time, burst_time):
        self.pid = pid
        self.arrival_time = arrival_time
        self.burst_time = burst_time
        self.completion_time = 0
        self.turnaround_time = 0
        self.waiting_time = 0
\end{python}

\begin{table}[htbp] % Optional: For floating position
    \centering
    \begin{tabular}{|c|c|c|} % Defines number of columns and alignment (c = center, l = left, r = right). '|' creates vertical lines.
    \hline
    Header 1 & Header 2 & Header 3 \\ % Column headers
    \hline
    Row 1, Column 1 & Row 1, Column 2 & Row 1, Column 3 \\ % First row of data
    \hline
    Row 2, Column 1 & Row 2, Column 2 & Row 2, Column 3 \\ % Second row of data
    \hline
    \end{tabular}
    \caption{Your table caption} % Optional: For adding a caption
    \label{tab:your_label} % Optional: For cross-referencing the table
\end{table}

\subsection{Assignment 2: Deadlock Handling}
In this assignment, students were asked to simulate different deadlock scenarios and explore various prevention methods.
\subsubsection{kelompok 7}
\begin{itemize}
    \item\textbf{soal}
    \par\hspace{2em}Sebuah sistem memiliki total 3 unit resource yang melayani dua proses. Masing-masing proses membutuhkan 2 unit resource dan telah dialokasikan 1 unit resource. Tentukan apakah sistem dalam keadaan aman atau berpotensi mengalami deadlock!
    
    \item\textbf{jawaban}
    \par\hspace{2em}Total resource yang tersedia adalah 3 unit, dan resource yang sudah dialokasikan untuk kedua proses adalah 2 unit. Ini berarti resource yang tersisa di sistem adalah 1 unit.
    \par\hspace{2em}Kedua proses masing-masing masih membutuhkan 1 unit resource lagi untuk menyelesaikan eksekusi. Karena resource yang tersedia cukup untuk memenuhi kebutuhan salah satu proses, maka proses tersebut dapat selesai. Setelah proses pertama selesai, resource yang dialokasikan dikembalikan, sehingga tersedia 2 unit resource. Dengan resource ini, proses kedua juga dapat diselesaikan.
    \par Sistem dalam keadaan aman.

     \item\textbf{Implementasi Kode Python}
    \lstdefinestyle{ystyle}{
        basicstyle=\ttfamily\footnotesize,
        breakatwhitespace=false,         
        breaklines=true,                 
        captionpos=b,                    
        keepspaces=true,                                    
        numbersep=5pt,                  
        showspaces=false,                
        showstringspaces=false,
        showtabs=false,                  
        tabsize=2,
        frame=single                   
    }
         \lstset{style=ystyle}
    \begin{lstlisting}def is_safe(total_resources, max_demand, allocated):
    available = total_resources - sum(allocated)
    need = [max_demand[i] - allocated[i] for i in range(len(max_demand))]

    for i in range(len(need)):
        if need[i] <= available:
            available += allocated[i]
        else:
            return False
    return True

    # Data
    total_resources = 3
    max_demand = [2, 2]  # Kedua proses membutuhkan 2 unit
    allocated = [1, 1]   # Kedua proses sudah dialokasikan 1 unit

    # Cek apakah aman
    if is_safe(total_resources, max_demand, allocated):
        print("Sistem dalam keadaan aman.")
    else:
        print("Sistem dalam keadaan deadlock.")
    \end{lstlisting}
    
\end{itemize}
\subsection{Assignment 3: Multithreading and Amdahl's Law}
This assignment involved designing a multithreading scenario to solve a computationally intensive problem. Students then applied **Amdahl's Law** to calculate the theoretical speedup of the program as the number of threads increased.

\subsection{Assignment 4: Simple Command-Line Interface (CLI) for User Interface Management}
Students were tasked with creating a simple **CLI** for user interface management. The CLI should support basic commands such as file manipulation (creating, listing, and deleting files), process management, and system status reporting.
\subsubsection{kelompok 7}
\begin{itemize}
    \item\textbf{soal}
\par\hspace{2em}Buat sebuah program Command-Line Interface (CLI) sederhana untuk mengelola daftar pengguna. Program harus memiliki menu dengan opsi untuk menambahkan pengguna baru, menghapus pengguna, menampilkan semua pengguna, dan keluar dari program. Program akan terus berjalan hingga pengguna memilih untuk keluar.

    \item\textbf{Jawaban}
\par\hspace{2em}Program ini memungkinkan pengguna untuk mengelola daftar pengguna melalui antarmuka baris perintah. Pengguna dapat memilih untuk membuat pengguna baru, menghapus pengguna, menampilkan semua pengguna, atau keluar dari program. Program akan terus berjalan hingga pengguna memilih untuk keluar.

    \item\textbf{Implementasi Kode Python}
    \lstdefinestyle{mystyle}{
        basicstyle=\ttfamily\footnotesize,
        breakatwhitespace=false,         
        breaklines=true,                 
        captionpos=b,                    
        keepspaces=true,                                    
        numbersep=5pt,                  
        showspaces=false,                
        showstringspaces=false,
        showtabs=false,                  
        tabsize=2,
        frame=single                   
    }

        \lstset{style=mystyle}
    \begin{lstlisting}def display_menu():
    print("1. Create User\n2. Delete User\n3. List Users\n4. Exit")
    return input("Choose an option: ")

    users = []
    while True:
        choice = display_menu()
        if choice == '1':
            user = input("Enter new username: ")
            users.append(user) if user not in users else print("User already exists.")
        elif choice == '2':
            user = input("Enter username to delete: ")
            users.remove(user) if user in users else print("User not found.")
        elif choice == '3':
            print("\n".join(users) if users else "No users available.")
        elif choice == '4':
            break
        else:
            print("Invalid choice.")
         \end{lstlisting}
\end{itemize}

\subsection{Assignment 5: File System Access}
In this assignment, students implemented file system access routines, including:
\begin{itemize}
    \item File creation and deletion
    \item Reading from and writing to files
    \item Navigating directories and managing file permissions
\end{itemize}

\section{Conclusion}
The first half of the course introduced core operating system concepts, including process management, scheduling, multithreading, and file system access. These topics provided a foundation for more advanced topics to be covered in the second half of the course.

\end{document}