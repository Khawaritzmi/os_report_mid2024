\documentclass[12pt]{article}
\usepackage{amsmath}
\usepackage{graphicx}
\usepackage{hyperref}
\usepackage{listings}
\usepackage{color}
\usepackage{pythonhighlight}

\title{Operating System Course Report - First Half of the Semester}
\author{B class}
\date{\today}

\begin{document}

\maketitle
\newpage

\tableofcontents
\newpage

\section{Introduction}
This report summarizes the topics covered during the first half of the Operating System course. It includes theoretical concepts, practical implementations, and assignments. The course focuses on the fundamentals of operating systems, including system architecture, process management, CPU scheduling, and deadlock handling.

\section{Course Overview}
\subsection{Objectives}
The main objectives of this course are:
\begin{itemize}
    \item To understand the basic components and architecture of a computer system.
    \item To learn process management, scheduling, and inter-process communication.
    \item To explore file systems, input/output management, and virtualization.
    \item To study the prevention and handling of deadlocks in operating systems.
\end{itemize}

\subsection{Course Structure}
The course is divided into two halves. This report focuses on the first half, which covers:
\begin{itemize}
    \item Basic Concepts and Components of Computer Systems
    \item System Performance and Metrics
    \item System Architecture of Computer Systems
    \item Process Description and Control
    \item Scheduling Algorithms
    \item Process Creation and Termination
    \item Introduction to Threads
    \item File Systems
    \item Input and Output Management
    \item Deadlock Introduction and Prevention
    \item User Interface Management
    \item Virtualization in Operating Systems
\end{itemize}

\section{Topics Covered}

\subsection{Basic Concepts and Components of Computer Systems}
This section explains the fundamental components that make up a computer system, including the CPU, memory, storage, and input/output devices.

\subsection{System Performance and Metrics}
This section introduces various system performance metrics used to measure the efficiency of a computer system, including throughput, response time, and utilization.

\subsection{System Architecture of Computer Systems}
Describes the architecture of modern computer systems, focusing on the interaction between hardware and the operating system.

\subsection{Process Description and Control}
Processes are a central concept in operating systems. This section covers:
\begin{itemize}
    \item Process states and state transitions
    \item Process control block (PCB)
    \item Context switching
\end{itemize}

\subsection{Scheduling Algorithms}
This section covers:
\begin{itemize}
    \item First-Come, First-Served (FCFS)
    \item Shortest Job Next (SJN)
    \item Round Robin (RR)
\end{itemize}
It explains how these algorithms are used to allocate CPU time to processes.

\subsection{Process Creation and Termination}
Details how processes are created and terminated by the operating system, including:
\begin{itemize}
    \item Process spawning
    \item Process termination conditions
\end{itemize}

\subsection{Introduction to Threads}
This section introduces the concept of threads and their relation to processes, covering:
\begin{itemize}
    \item Single-threaded vs. multi-threaded processes
    \item Benefits of multithreading
\end{itemize}

\begin{figure}[h]
    \centering
    \includegraphics[width=0.5\textwidth]{b_class/asset/example.png}  % Sesuaikan nama file dan ukurannya
    \caption{Ini adalah gambar contoh dari multithreading.}
    \label{fig:contoh_gambar}
\end{figure}

Seperti yang terlihat pada Gambar \ref{fig:contoh_gambar}, inilah cara menambahkan gambar dengan keterangan.

\subsection{File Systems}
File systems provide a way for the operating system to store, retrieve, and manage data. This section explains:
\begin{itemize}
    \item File system structure
    \item File access methods
    \item Directory management
\end{itemize}

\subsection{Input dan Output Management} \subsubsection{Apa itu Manajemen I/O?} Manajemen I/O merupakan entitas yang bertanggung jawab mengontrol perangkat eksternal dan untuk pertukaran data antarperangkat tersebut dengan memori dan CPU. Perangkat eksternal itu seperti \textit{keyboard, mouse,} dan lain sebagainya.

\subsubsection{Fungsi Utama Modul I/O} 
Fungsi utama modul I/O dikategorikan menjadi lima, yaitu: 
\begin{itemize} 
    \item \textit{Control dan timing}: Modul I/O berfungsi untuk mengatur aliran data antara sistem inti komputer (seperti CPU dan memori) dan perangkat eksternal. Fungsi ini membutuhkan \textit{control dan timing} yang tepat karena ada perbedaan kecepatan dan mekanisme antara CPU dan perangkat eksternal yang biasanya lebih lambat. Modul I/O harus memastikan bahwa data dikirim dan diterima sesuai dengan waktu yang tepat, tanpa menyebabkan konflik atau kerusakan data. 
    \item Komunikasi CPU: Modul I/O harus mampu berkomunikasi baik dengan CPU maupun dengan perangkat eksternal. Komunikasi ini mencakup beberapa aspek: 
    \begin{itemize} 
        \item \textit{Command Decoding} (Dekode Perintah): Modul I/O menerima perintah dari CPU, yang biasanya dikirim melalui sinyal pada bus control. Contoh perintah untuk modul I/O pada disk adalah seperti \textit{READ SECTOR, WRITE SECTOR, SEEK} nomor \textit{track, dan SCAN record ID.} Beberapa perintah ini mungkin menyertakan parameter yang dikirim melalui bus data. 
        \item Data: Data ditransfer antara CPU dan modul I/O melalui bus data. Ini adalah proses pertukaran data yang terjadi setelah perintah diterima dan diproses oleh modul I/O. 
        \item \textit{Status Reporting} (Laporan Status): Karena perangkat eksternal sering kali lebih lambat daripada CPU, modul I/O harus melaporkan statusnya kepada CPU, misalnya dengan sinyal status seperti \textit{BUSY} (sibuk) atau \textit{READY} (siap). Ini penting agar CPU tahu apakah modul I/O siap untuk melakukan operasi atau sedang sibuk dengan perintah lain. 
        \item \textit{Address Recognition} (Pengenalan Alamat): Sama seperti memori memiliki alamat unik, perangkat I/O juga memiliki alamat yang unik. Modul I/O harus mampu mengenali alamat unik dari perangkat-perangkat yang dikendalikannya untuk mengarahkan data ke perangkat yang benar. \end{itemize} 
    \item Komunikasi Perangkat: Modul I/O juga harus bisa berkomunikasi langsung dengan perangkat eksternal. Komunikasi ini meliputi: 
    \begin{itemize} 
        \item Perintah (instruksi yang dikirim ke perangkat). \item Informasi Status (kondisi perangkat, seperti siap atau sibuk). 
        \item Data (informasi yang ditransfer antara perangkat dan CPU). 
    \end{itemize} 
    \item \textit{Data Buffering}: Modul I/O bertugas untuk menyiapkan data dari atau untuk perangkat eksternal melalui \textit{buffering }(penyimpanan sementara data) sehingga data dapat diproses dengan lebih efisien, mengatasi perbedaan kecepatan antara CPU dan perangkat eksternal. \item Deteksi Kesalahan: Modul I/O juga bertanggung jawab untuk mendeteksi dan melaporkan kesalahan yang terjadi saat komunikasi dengan perangkat eksternal. Kesalahan ini bisa berupa masalah mekanis (seperti kertas yang menggulung pada printer) atau kesalahan elektris (seperti kesalahan bit saat mentransfer data). Modul I/O kemudian akan melaporkan kesalahan ini ke CPU agar dapat diatasi. 
\end{itemize}

\subsubsection{Struktur Modul I/O} 
\begin{figure}[h] 
    \centering 
    \includegraphics[width=0.80\linewidth]{b_class/asset/sisop.drawio.png} 
    \caption{Struktur Modul I/O} \label{fig} 
\end{figure} 
Penjelasan: \begin{itemize} \item Data, alamat, dan kontrol dikirim dari sistem komputer (\textit{bus system}) ke modul I/O. \item \textit{I/O Logic} mengelola aliran data, kontrol, dan alamat yang masuk. \item \textit{Data Register} menyimpan data sementara, dan \textit{Status/Control Register} mengatur status dan kontrol perangkat. \item \textit{External Device Interface Logic} menjembatani data, status, dan kontrol ke perangkat eksternal. \item Data, status, dan kontrol diolah dan dikirim ke perangkat eksternal, dan informasi status kembali ke sistem komputer. \end{itemize}

\subsubsection{Control dan Timing} 
\textit{Control dan Timing} berfungsi untuk mengatur agar kecepatan transfer, data yang berbeda-beda antar periferal dapat tersinkronisasi. Misalnya, kontrol pemindahan data dari sebuah perangkat eksternal ke CPU. Contoh kontrol pemindahan data dari sebuah perangkat eksternal ke CPU meliputi langkah-langkah berikut: 
\begin{enumerate} 
    \item CPU meminta modul I/O untuk memeriksa status perangkat yang terhubung. 
    \item Modul I/O memberikan jawabannya tentang status perangkat. 
    \item Bila perangkat sedang beroperasi dan berada dalam keadaan siap untuk mengirimkan, maka CPU meminta pemindahan data, dengan menggunakan perintah tertentu ke modul I/O. \item Modul I/O akan memperoleh unit data (misalnya, 8 atau 16 bit) dari perangkat eksternal. 
    \item Data akan dipindahkan dari modul I/O ke CPU. 
\end{enumerate}

\subsection{Deadlock Introduction and Prevention}
Explores the concept of deadlocks and methods for preventing them:
\begin{itemize}
    \item Deadlock conditions
    \item Deadlock prevention techniques
\end{itemize}

\subsection{User Interface Management}
This section discusses the role of the operating system in managing the user interface. Topics covered include:
\begin{itemize}
    \item Graphical User Interface (GUI)
    \item Command-Line Interface (CLI)
    \item Interaction between the user and the operating system
\end{itemize}

\subsection{Virtualization in Operating Systems}
Virtualization allows multiple operating systems to run concurrently on a single physical machine. This section explores:
\begin{itemize}
    \item Concept of virtualization
    \item Hypervisors and their types
    \item Benefits of virtualization in modern computing
\end{itemize}

\section{Assignments and Practical Work}
\subsection{Assignment 1: Process Scheduling}
Students were tasked with implementing various process scheduling algorithms (e.g., FCFS, SJN, and RR) and comparing their performance under different conditions.
\subsubsection{Group 1}
\begin{python}
    class Process:
    def __init__(self, pid, arrival_time, burst_time):
        self.pid = pid
        self.arrival_time = arrival_time
        self.burst_time = burst_time
        self.completion_time = 0
        self.turnaround_time = 0
        self.waiting_time = 0
\end{python}
\begin{table}[htbp] % Optional: For floating position
    \centering
    \begin{tabular}{|c|c|c|} % Defines number of columns and alignment (c = center, l = left, r = right). '|' creates vertical lines.
    \hline
    Header 1 & Header 2 & Header 3 \\ % Column headers
    \hline
    Row 1, Column 1 & Row 1, Column 2 & Row 1, Column 3 \\ % First row of data
    \hline
    Row 2, Column 1 & Row 2, Column 2 & Row 2, Column 3 \\ % Second row of data
    \hline
    \end{tabular}
    \caption{Your table caption} % Optional: For adding a caption
    \label{tab:your_label} % Optional: For cross-referencing the table
\end{table}
\subsubsection{}
\subsubsection{}
\subsubsection{}
\subsubsection{}
\subsubsection{}
\subsubsection{}
\subsubsection{}
\subsubsection{Group 9}
\begin{itemize}
    \item Soal : Diberikan sebuah skenario di mana terdapat beberapa proses yang ingin dijalankan oleh CPU. Setiap proses memiliki waktu kedatangan (arrival time) dan waktu burst (burst time). Tugas Anda adalah menghitung rata-rata waktu tunggu (average waiting time) dari proses-proses tersebut menggunakan tiga algoritma penjadwalan CPU, yaitu:
    \begin{itemize}
        \item First Come First Serve (FCFS)
        \item Shortest Job Next (SJN)
        \item Round Robin (RR) dengan quantum sebesar 4
    \end{itemize}
    
    \begin{table}[h!]
    \centering
    \begin{tabular}{|c|c|c|}
        \hline
        \textbf{Proses} & \textbf{burst\_times} & \textbf{waiting\_times} \\ 
        \hline
        1              & 10                   & 0                         \\ 
        \hline
        2              & 5                    & 1                         \\ 
        \hline
        3              & 8                    & 2                         \\ 
        \hline
        4              & 6                    & 3                         \\ 
        \hline
        5              & 2                    & 4                         \\ 
        \hline
        \end{tabular}
        \caption{Tabel Data Proses}
        \label{tabel:proses}
    \end{table}
    \item Penyelesaian Kode Python :
\begin{python}
# Fungsi untuk menghitung waktu tunggu rata-rata
def hitung_rata2_waktu(n, burst_times, waiting_times):
    total_wt = sum(waiting_times)
    rata2_wt = total_wt / n
    return rata2_wt

# Algoritma FCFS
def fcfs(proses, n, burst_times, waktu_kedatangan):
    waiting_times = [0] * n

    # Menghitung waktu tunggu
    for i in range(1, n):
        waiting_times[i] = waiting_times[i - 1] + burst_times[i - 1]

    rata2_wt = hitung_rata2_waktu(n, burst_times, waiting_times)
    return rata2_wt

# Algoritma SJN
def sjn(proses, n, burst_times, waktu_kedatangan):
    # Urutkan proses berdasarkan waktu burst
    proses_urut = sorted(zip(proses, burst_times, waktu_kedatangan), key=lambda x: (x[1], x[2]))
    
    waiting_times = [0] * n
    waktu_sekarang = 0
    
    for i in range(n):
        id_proses, burst, kedatangan = proses_urut[i]
        
        if waktu_sekarang < kedatangan:
            waktu_sekarang = kedatangan
        
        waiting_times[id_proses - 1] = waktu_sekarang - kedatangan
        waktu_sekarang += burst

    rata2_wt = hitung_rata2_waktu(n, burst_times, waiting_times)
    return rata2_wt

# Algoritma Round Robin
def round_robin(proses, n, burst_times, waktu_kedatangan, kuantum):
    sisa_burst = burst_times[:]  
    waiting_times = [0] * n
    waktu_sekarang = 0
    selesai = False
    
    while not selesai:
        selesai = True
        for i in range(n):
            if sisa_burst[i] > 0:
                selesai = False
                if sisa_burst[i] > kuantum:
                    waktu_sekarang += kuantum
                    sisa_burst[i] -= kuantum
                else:
                    waktu_sekarang += sisa_burst[i]
                    waiting_times[i] = waktu_sekarang -
                    burst_times[i]
                    sisa_burst[i] = 0

    rata2_wt = hitung_rata2_waktu(n, burst_times, waiting_times)
    return rata2_wt

# Kode utama
def main():
    # Data proses yang diberikan
    proses = [1, 2, 3, 4, 5]
    burst_times = [10, 5, 8, 6, 2]
    waktu_kedatangan = [0, 1, 2, 3, 4]
    kuantum = 4

    n = len(proses)

    # Menjalankan setiap algoritma
    rata2_wt_fcfs = fcfs(proses, n, burst_times, waktu_kedatangan)
    rata2_wt_sjn = sjn(proses, n, burst_times, waktu_kedatangan)
    rata2_wt_rr = round_robin(proses, n, burst_times, waktu_kedatangan, kuantum)

    # Menampilkan hasil
    print("Algoritma FCFS")
    print(f"Rata-rata Waktu Tunggu: {rata2_wt_fcfs:.2f}\n")

    print("Algoritma SJN")
    print(f"Rata-rata Waktu Tunggu: {rata2_wt_sjn:.2f}\n")

    print("Algoritma Round Robin")
    print(f"Rata-rata Waktu Tunggu: {rata2_wt_rr:.2f}\n")

if __name__ == "__main__":
    main()
\end{python}
    \item Output : 
    \begin{figure}[h]
        \centering
        \includegraphics[width=0.80\linewidth]{b_class/asset/output1.png}
        \caption{Output Soal 1}
        \label{fig:enter-label}
    \end{figure}
\end{itemize}

\subsection{Assignment 2: Deadlock Handling}
In this assignment, students were asked to simulate different deadlock scenarios and explore various prevention methods.
\subsubsection{}
\subsubsection{}
\subsubsection{}
\subsubsection{}
\subsubsection{}
\subsubsection{}
\subsubsection{}
\subsubsection{}
\subsubsection{Group 9}
\begin{itemize}
    \item Soal : 
    Kondisi awal:

    \begin{itemize}
        \item Proses A membutuhkan R1 untuk mulai berjalan dan menunggu R2 untuk menyelesaikan.
        \item Proses B membutuhkan R2 untuk mulai berjalan dan menunggu R1 untuk menyelesaikan.
    \end{itemize}
    Gunakan Algoritma Banker's untuk mengecek apakah sistem dalam keadaan aman atau terjadi deadlock, dengan data berikut:
    \begin{enumerate}
        \item Proses A dan Proses B membutuhkan masing-masing maksimal 3 unit dari R1 dan R2.
        \item Saat ini, Proses A telah dialokasikan 1 unit dari R1 dan 2 unit dari R2.
        \item Proses B telah dialokasikan 2 unit dari R1 dan 1 unit dari R2.
        \item Tersisa 2 unit R1 dan 1 unit R2 di sistem.
    \end{enumerate}

    \item Penyelesaian Kode Python:
    \begin{python}
# Fungsi untuk mensimulasikan skenario deadlock
def simulasi_deadlock():
    print("Mensimulasikan Skenario Deadlock...")
    # Dua sumber daya dan dua proses
    sumber_daya = [1, 1]  # Sumber daya yang tersedia (R1 dan R2)
    proses_A = [1, 0]  # Proses A awalnya membutuhkan sumber daya R1
    proses_B = [0, 1]  # Proses B awalnya membutuhkan sumber daya R2
    
    # Kedua proses mencoba untuk mendapatkan sumber daya
    if sumber_daya[0] >= proses_A[0]:
        sumber_daya[0] -= proses_A[0]
        print("Proses A mendapatkan Sumber Daya 1")

    if sumber_daya[1] >= proses_B[1]:
        sumber_daya[1] -= proses_B[1]
        print("Proses B mendapatkan Sumber Daya 2")
    
    # Sekarang kedua proses mencoba untuk mendapatkan sumber daya satu sama lain
    if sumber_daya[1] >= 1 and sumber_daya[0] < 1:
        print("Proses A menunggu Sumber Daya 2, tetapi dipegang oleh Proses B")

    if sumber_daya[0] >= 1 and sumber_daya[1] < 1:
        print("Proses B menunggu Sumber Daya 1, tetapi dipegang oleh Proses A")
        
    print("Deadlock telah terjadi!\n")

# Algoritma Banker's untuk pencegahan deadlock
def algoritma_bankir(proses, sumber_daya_maks, sumber_daya_alokasi, sumber_daya_tersedia):
    print("Menerapkan Algoritma Banker's...")
    n = len(proses)  # Jumlah proses
    m = len(sumber_daya_maks[0])  # Jumlah jenis sumber daya

    # Menghitung matriks kebutuhan
    kebutuhan = [[0] * m for _ in range(n)]
    for i in range(n):
        for j in range(m):
            kebutuhan[i][j] = sumber_daya_maks[i][j] - sumber_daya_alokasi[i][j]

    selesai = [False] * n  # Melacak apakah proses telah selesai
    urutan_aman = []    # Daftar untuk menyimpan urutan aman jika ditemukan
    
    while len(urutan_aman) < n:
        dialokasikan = False
        for i in range(n):
            # Memeriksa apakah proses dapat dieksekusi dengan aman
            if not selesai[i] and all(kebutuhan[i][j] <= sumber_daya_tersedia[j] for j in range(m)):
                # Jika bisa, tambahkan sumber daya yang dialokasikan kembali ke sumber daya tersedia
                for k in range(m):
                    sumber_daya_tersedia[k] += sumber_daya_alokasi[i][k]
                
                # Tandai proses sebagai selesai dan tambahkan ke urutan aman
                urutan_aman.append(proses[i])
                selesai[i] = True
                dialokasikan = True
                print(f"Proses {proses[i]} dieksekusi dengan aman")
        
        # Jika tidak ada proses yang bisa dieksekusi di iterasi ini, ada deadlock
        if not dialokasikan:
            print("Deadlock terdeteksi oleh Algoritma Banker's - Sistem tidak dalam keadaan aman\n")
            return False

    print("Sistem dalam keadaan aman. Urutan aman:", urutan_aman, "\n")
    return True

# Kode Utama
def main():
    print("Memulai Simulasi Deadlock dan Pencegahan Menggunakan Algoritma Banker's...\n")
    
    # Mensimulasikan skenario deadlock
    simulasi_deadlock()
    
    # Menerapkan Algoritma Banker's untuk mencegah deadlock
    proses = ["P1", "P2"]
    sumber_daya_maks = [[3, 3], [4, 2]]  # Sumber daya maksimum yang dibutuhkan oleh setiap proses
    sumber_daya_alokasi = [[1, 2], [2, 1]]  # Sumber daya yang saat ini dialokasikan ke setiap proses
    sumber_daya_tersedia = [2, 1]  # Total sumber daya yang tersedia

    algoritma_bankir(proses, sumber_daya_maks, sumber_daya_alokasi, sumber_daya_tersedia)

if __name__ == "__main__":
    main()
    \end{python}
    
    \item Output : 
    \begin{figure}[h]
        \centering
        \includegraphics[width=0.70\linewidth]{b_class/asset/output2.png}
        \caption{Output Soal 2}
        \label{fig:enter-label}
    \end{figure}

\newline Berikut adalah output dari kode tersebut:
\end{itemize}
\subsection{Assignment 3: Multithreading and Amdahl's Law}
This assignment involved designing a multithreading scenario to solve a computationally intensive problem. Students then applied **Amdahl's Law** to calculate the theoretical speedup of the program as the number of threads increased.

\subsection{Assignment 4: Simple Command-Line Interface (CLI) for User Interface Management}
Students were tasked with creating a simple **CLI** for user interface management. The CLI should support basic commands such as file manipulation (creating, listing, and deleting files), process management, and system status reporting.

\subsection{Assignment 5: File System Access}
In this assignment, students implemented file system access routines, including:
\begin{itemize}
    \item File creation and deletion
    \item Reading from and writing to files
    \item Navigating directories and managing file permissions
\end{itemize}
\section{Conclusion}
The first half of the course introduced core operating system concepts, including process management, scheduling, multithreading, and file system access. These topics provided a foundation for more advanced topics to be covered in the second half of the course.

\end{document}