\documentclass[12pt]{article}
\usepackage{amsmath}
\usepackage{graphicx}
\usepackage{hyperref}
\usepackage{listings}
\usepackage{color}
\usepackage{pythonhighlight}

\title{Operating System Course Report - First Half of the Semester}
\author{B class}
\date{\today}

\begin{document}

\maketitle
\newpage

\tableofcontents
\newpage

\section{Introduction}
This report summarizes the topics covered during the first half of the Operating System course. It includes theoretical concepts, practical implementations, and assignments. The course focuses on the fundamentals of operating systems, including system architecture, process management, CPU scheduling, and deadlock handling.

\section{Course Overview}
\subsection{Objectives}
The main objectives of this course are:
\begin{itemize}
    \item To understand the basic components and architecture of a computer system.
    \item To learn process management, scheduling, and inter-process communication.
    \item To explore file systems, input/output management, and virtualization.
    \item To study the prevention and handling of deadlocks in operating systems.
\end{itemize}

\subsection{Course Structure}
The course is divided into two halves. This report focuses on the first half, which covers:
\begin{itemize}
    \item Basic Concepts and Components of Computer Systems
    \item System Performance and Metrics
    \item System Architecture of Computer Systems
    \item Process Description and Control
    \item Scheduling Algorithms
    \item Process Creation and Termination
    \item Introduction to Threads
    \item File Systems
    \item Input and Output Management
    \item Deadlock Introduction and Prevention
    \item User Interface Management
    \item Virtualization in Operating Systems
\end{itemize}

\section{Topics Covered}

\subsection{Basic Concepts and Components of Computer Systems}
This section explains the fundamental components that make up a computer system, including the CPU, memory, storage, and input/output devices.

\subsection{System Performance and Metrics}
This section introduces various system performance metrics used to measure the efficiency of a computer system, including throughput, response time, and utilization.

\subsection{System Architecture of Computer Systems}
Describes the architecture of modern computer systems, focusing on the interaction between hardware and the operating system.

\subsection{Process Description and Control}
Processes are a central concept in operating systems. This section covers:
\begin{itemize}
    \item Process states and state transitions
    \item Process control block (PCB)
    \item Context switching
\end{itemize}

\subsection{Scheduling Algorithms}
This section covers:
\begin{itemize}
    \item First-Come, First-Served (FCFS)
    \item Shortest Job Next (SJN)
    \item Round Robin (RR)
\end{itemize}
It explains how these algorithms are used to allocate CPU time to processes.

\subsection{Process Creation and Termination}
Details how processes are created and terminated by the operating system, including:
\begin{itemize}
    \item Process spawning
    \item Process termination conditions
\end{itemize}

\subsection{Introduction to Threads}
This section introduces the concept of threads and their relation to processes, covering:
\begin{itemize}
    \item Single-threaded vs. multi-threaded processes
    \item Benefits of multithreading
\end{itemize}

\begin{figure}[h]
    \centering
    \includegraphics[width=0.5\textwidth]{/Users/khawaritzmi/Unhas/os_report_mid2024/b_class/asset/example.png}  % Sesuaikan nama file dan ukurannya
    \caption{Ini adalah gambar contoh dari multithreading.}
    \label{fig:contoh_gambar}
\end{figure}

Seperti yang terlihat pada Gambar \ref{fig:contoh_gambar}, inilah cara menambahkan gambar dengan keterangan.

\subsection{File Systems}
File systems provide a way for the operating system to store, retrieve, and manage data. This section explains:
\begin{itemize}
    \item File system structure
    \item File access methods
    \item Directory management
\end{itemize}

\subsection{Input and Output Management}
Input and output management is key for handling the interaction between the system and external devices. This section includes:
\begin{itemize}
    \item Device drivers
    \item I/O scheduling
\end{itemize}

\subsection{Deadlock Introduction and Prevention}
Explores the concept of deadlocks and methods for preventing them:
\begin{itemize}
    \item Deadlock conditions
    \item Deadlock prevention techniques
\end{itemize}

\subsection{User Interface Management}
\subsubsection{Interaksi antara Pengguna dan Sistem Operasi}
		
		Interaksi antara pengguna dan sistem operasi adalah proses dinamis yang melibatkan berbagai komponen sistem. Sistem operasi modern berupaya untuk menyediakan pengalaman yang mulus dan intuitif, sering kali menggabungkan elemen GUI dan CLI untuk memenuhi kebutuhan berbagai jenis pengguna \cite{Tanenbaum2015}.
		
		Aspek-aspek kunci dari interaksi ini meliputi:
		\begin{itemize}
			\item Input/Output Management: Sistem operasi mengelola berbagai perangkat input (keyboard, mouse, touchscreen) dan output (display, audio).
			\item Process Management: Mengatur eksekusi dan prioritas aplikasi dan layanan yang berjalan.
			\item File System Interaction: Menyediakan antarmuka untuk mengakses, memodifikasi, dan mengelola file dan direktori.
			\item Network Communication: Mengelola koneksi jaringan dan menyediakan antarmuka untuk konfigurasi dan monitoring.
			\item Security and Authentication: Mengimplementasikan mekanisme untuk mengamankan sistem dan memverifikasi identitas pengguna.
			\item System Configuration: Menyediakan alat untuk menyesuaikan perilaku sistem sesuai preferensi pengguna.
		\end{itemize}
		
		Tren terbaru dalam interaksi pengguna-sistem operasi meliputi:
		\begin{itemize}
			\item Antarmuka Suara dan AI: Integrasi asisten virtual dan kontrol suara untuk interaksi hands-free \cite{Murad2021}.
			\item Personalisasi Berbasis AI: Sistem yang mempelajari kebiasaan pengguna dan menyesuaikan antarmuka secara otomatis \cite{Yang2019}.
			\item Antarmuka Gesture: Peningkatan dukungan untuk kontrol berbasis gerakan, terutama pada perangkat mobile dan AR/VR.
			\item Aksesibilitas Lanjutan: Fitur yang lebih canggih untuk mendukung pengguna dengan berbagai kemampuan dan preferensi.
			\item Integrasi Cross-Platform: Peningkatan sinkronisasi dan konsistensi pengalaman pengguna di berbagai perangkat dan platform.
		\end{itemize}
		\begin{thebibliography}{9}
			
			\bibitem{Lazar2017} 
			Lazar, J., Feng, J. H., \& Hochheiser, H. (2017). \textit{Research methods in human-computer interaction} (2nd ed.). Morgan Kaufmann.
			
			\bibitem{Murad2021} 
			Murad, C., Munteanu, C., Clark, L., \& Cowan, B. R. (2021). Meta-analysis of voice interaction user studies: A 20-year retrospective. \textit{ACM Transactions on Computer-Human Interaction}, 28(5), 1-48. https://doi.org/10.1145/3446393
			
			\bibitem{Shneiderman2016} 
			Shneiderman, B., Plaisant, C., Cohen, M., Jacobs, S., Elmqvist, N., \& Diakopoulos, N. (2016). \textit{Designing the user interface: Strategies for effective human-computer interaction} (6th ed.). Pearson.
			
			\bibitem{Shotts2019} 
			Shotts, W. (2019). \textit{The Linux command line: A complete introduction} (2nd ed.). No Starch Press.
			
			\bibitem{Tanenbaum2015} 
			Tanenbaum, A. S., \& Bos, H. (2015). \textit{Modern operating systems} (4th ed.). Pearson.
			
			\bibitem{Yang2019} 
			Yang, Q., Steinfeld, A., \& Zimmerman, J. (2019). Unremarkable AI: Fitting intelligent decision support into critical, clinical decision-making processes. In \textit{Proceedings of the 2019 CHI Conference on Human Factors in Computing Systems} (pp. 1-11). https://doi.org/10.1145/3290605.3300641
			
			\bibitem{Greenberg2017} 
			Greenberg, J., Mates, P., \& Nam, Y. J. (2017). Robust command line interfaces: Enhancing usability while maintaining power. \textit{Journal of Systems and Software}, 133, 176-189. https://doi.org/10.1016/j.jss.2017.07.011
			
		\end{thebibliography}
	

\subsection{Virtualization in Operating Systems}
Virtualization allows multiple operating systems to run concurrently on a single physical machine. This section explores:
\begin{itemize}
    \item Concept of virtualization
    \item Hypervisors and their types
    \item Benefits of virtualization in modern computing
\end{itemize}

\section{Assignments and Practical Work}
\subsection{Assignment 1: Process Scheduling}
Students were tasked with implementing various process scheduling algorithms (e.g., FCFS, SJN, and RR) and comparing their performance under different conditions.
\subsubsection{Group 1}
\begin{python}
    class Process:
    def __init__(self, pid, arrival_time, burst_time):
        self.pid = pid
        self.arrival_time = arrival_time
        self.burst_time = burst_time
        self.completion_time = 0
        self.turnaround_time = 0
        self.waiting_time = 0
\end{python}

\begin{table}[htbp] % Optional: For floating position
    \centering
    \begin{tabular}{|c|c|c|} % Defines number of columns and alignment (c = center, l = left, r = right). '|' creates vertical lines.
    \hline
    Header 1 & Header 2 & Header 3 \\ % Column headers
    \hline
    Row 1, Column 1 & Row 1, Column 2 & Row 1, Column 3 \\ % First row of data
    \hline
    Row 2, Column 1 & Row 2, Column 2 & Row 2, Column 3 \\ % Second row of data
    \hline
    \end{tabular}
    \caption{Your table caption} % Optional: For adding a caption
    \label{tab:your_label} % Optional: For cross-referencing the table
\end{table}

\subsection{Assignment 2: Deadlock Handling}
In this assignment, students were asked to simulate different deadlock scenarios and explore various prevention methods.

\subsection{Assignment 3: Multithreading and Amdahl's Law}
This assignment involved designing a multithreading scenario to solve a computationally intensive problem. Students then applied **Amdahl's Law** to calculate the theoretical speedup of the program as the number of threads increased.

\subsection{Assignment 4: Simple Command-Line Interface (CLI) for User Interface Management}
Students were tasked with creating a simple **CLI** for user interface management. The CLI should support basic commands such as file manipulation (creating, listing, and deleting files), process management, and system status reporting.

\subsection{Assignment 5: File System Access}
In this assignment, students implemented file system access routines, including:
\begin{itemize}
    \item File creation and deletion
    \item Reading from and writing to files
    \item Navigating directories and managing file permissions
\end{itemize}

\section{Conclusion}
The first half of the course introduced core operating system concepts, including process management, scheduling, multithreading, and file system access. These topics provided a foundation for more advanced topics to be covered in the second half of the course.

\end{document}