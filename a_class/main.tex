\documentclass[12pt]{article}
\usepackage{amsmath}
\usepackage{graphicx}
\usepackage{hyperref}
\usepackage{listings}
\usepackage{color}

\title{Operating System Course Report - First Half of the Semester}
\author{A class}
\date{\today}

\begin{document}

\maketitle
\newpage

\tableofcontents
\newpage

\section{Introduction}
This report summarizes the topics covered during the first half of the Operating System course. It includes theoretical concepts, practical implementations, and assignments. The course focuses on the fundamentals of operating systems, including system architecture, process management, CPU scheduling, and deadlock handling.

\section{Course Overview}
\subsection{Objectives}
The main objectives of this course are:
\begin{itemize}
    \item To understand the basic components and architecture of a computer system.
    \item To learn process management, scheduling, and inter-process communication.
    \item To explore file systems, input/output management, and virtualization.
    \item To study the prevention and handling of deadlocks in operating systems.
\end{itemize}

\subsection{Course Structure}
The course is divided into two halves. This report focuses on the first half, which covers:
\begin{itemize}
    \item Basic Concepts and Components of Computer Systems
    \item System Performance and Metrics
    \item System Architecture of Computer Systems
    \item Process Description and Control
    \item Scheduling Algorithms
    \item Process Creation and Termination
    \item Introduction to Threads
    \item File Systems
    \item Input and Output Management
    \item Deadlock Introduction and Prevention
    \item User Interface Management
    \item Virtualization in Operating Systems
\end{itemize}

\section{Topics Covered}

\subsection{Basic Concepts and Components of Computer Systems}
This section explains the fundamental components that make up a computer system, including the CPU, memory, storage, and input/output devices.

\subsection{System Performance and Metrics}
This section introduces various system performance metrics used to measure the efficiency of a computer system, including throughput, response time, and utilization.

\subsection{Sistem arsitektur dari sistem komputer}
Describes the architecture of modern computer systems, focusing on the interaction between hardware and the operating system.
\subsubsection{Pengertian Sistem Arsitektur Komputer}
\subsubsection{Fungsi Arsitektur Sistem Komputer}
\begin{enumerate}
    \item Perancangan awal komputer
    \par Perancangan awal ini artinya komputer disusun dan dirancang sebaik mungkin agar kinerjanya maksimal. Perancangan ini juga bertujuan untuk mengetahui hal apa yang kurang atau perlu diperbaiki.
    \item Kontrol Komponen dalam Meningkatkan Efisiensi Komputer
    \par Mengontrol komponen ini bertujuan agar kinerja komputer bisa berjalan dengan baik atau maksimal. Fungsi kontrol ini juga bisa membantu pengguna untuk bisa menjalankan banyak pekerjaan atau aplikasi dalam satu komputer.
    \item Peran Arsitektur Komputer dalam Pemilihan Aplikasi yang Tepat
    \par Arsitektur komputer dapat membantu programmer dalam menentukan aplikasi atau program apa yang cocok dengan komputer tersebut. Sehingga bisa disesuaikan dengan kebutuhan pengguna dan berjalan dengan maksimal.
\end{enumerate}

\subsection{Assignments}
\begin{table}[h]
    \centering
    \begin{tabular}{|c|c|c|} 
    \hline
    Header 1 & Header 2 & Header 3 \\ % Column headers
    \hline
    Row 1, Column 1 & Row 1, Column 2 & Row 1, Column 3 \\ % First row of data
    \hline
    Row 2, Column 1 & Row 2, Column 2 & Row 2, Column 3 \\ % Second row of data
    \hline
    \end{tabular}
    \caption{Your table caption} % Optional: For adding a caption
    \label{tab:your_label} % Optional: For cross-referencing the table
\end{table}

\subsection{Assignment 2: Deadlock Handling}
In this assignment, students were asked to simulate different deadlock scenarios and explore various prevention methods.

\subsection{Assignment 3: Multithreading and Amdahl's Law}
This assignment involved designing a multithreading scenario to solve a computationally intensive problem. Students then applied **Amdahl's Law** to calculate the theoretical speedup of the program as the number of threads increased.

\subsection{Assignment 4: Simple Command-Line Interface (CLI) for User Interface Management}
Students were tasked with creating a simple **CLI** for user interface management. The CLI should support basic commands such as file manipulation (creating, listing, and deleting files), process management, and system status reporting.

\subsection{Assignment 5: File System Access}
In this assignment, students implemented file system access routines, including:
\begin{itemize}
    \item File creation and deletion
    \item Reading from and writing to files
    \item Navigating directories and managing file permissions
\end{itemize}

\section{Conclusion}
The first half of the course introduced core operating system concepts, including process management, scheduling, multithreading, and file system access. These topics provided a foundation for more advanced topics to be covered in the second half of the course.

\begin{thebibliography}{}
\bibitem{}
Arsitektur dan organisasi komputer oleh Bambang Hariyanto (2015), Penerbit Informatika Bandung.

\bibitem{}
"Organisasi dan Arsitektur Komputer" oleh William Stallings, diterjemahkan oleh Thamir Abdul Hafedh Al-Hamdany (2010), Penerbit PT Indeks.

\bibitem{}
Pengantar Teknologi Informasi: Teknik Komputer dan Telekomunikasi oleh Abdul Kadir dan Terra Ch. Triwahyuni (2013), Penerbit Andi Publisher.

\bibitem{}
Dasar-dasar Komputer oleh Hapnes Toba (2009), Penerbit Informatika Bandung.

\bibitem{}
Konsep Dasar Sistem Komputer oleh Yani Maulita, Heru Budianto, dan Onno W. Purbo (2005), Penerbit Elex Media Komputindo.
\end{thebibliography}

\end{document}