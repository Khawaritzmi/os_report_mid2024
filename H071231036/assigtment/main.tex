\documentclass{article}
\usepackage{graphicx} % Required for inserting images
\documentclass[12pt]{article}
\usepackage{amsmath}
\usepackage{graphicx}
\usepackage{hyperref}
\usepackage{listings}
\usepackage{color}

\title{Operating System Course Report - First Half of the Semester}
\author{A class}
\date{\today}

\begin{document}

\maketitle


\section{Introduction}
This report summarizes the topics covered during the first half of the Operating System course. It includes theoretical concepts, practical implementations, and assignments. The course focuses on the fundamentals of operating systems, including system architecture, process management, CPU scheduling, and deadlock handling.

\section{Course Overview}
\subsection{Objectives}
The main objectives of this course are:
\begin{itemize}
    \item To understand the basic components and architecture of a computer system.
    \item To learn process management, scheduling, and inter-process communication.
    \item To explore file systems, input/output management, and virtualization.
    \item To study the prevention and handling of deadlocks in operating systems.
\end{itemize}

\subsection{Course Structure}
The course is divided into two halves. This report focuses on the first half, which covers:
\begin{itemize}
    \item Basic Concepts and Components of Computer Systems
    \item System Performance and Metrics
    \item System Architecture of Computer Systems
    \item Process Description and Control
    \item Scheduling Algorithms
    \item Process Creation and Termination
    \item Introduction to Threads
    \item File Systems
    \item Input and Output Management
    \item Deadlock Introduction and Prevention
    \item User Interface Management
    \item Virtualization in Operating Systems
\end{itemize}

\section{Topics Covered}

\subsection{Basic Concepts and Components of Computer Systems}
This section explains the fundamental components that make up a computer system, including the CPU, memory, storage, and input/output devices.

\subsection{System Performance and Metrics}
This section introduces various system performance metrics used to measure the efficiency of a computer system, including throughput, response time, and utilization.

\subsection{System Architecture of Computer Systems}
Describes the architecture of modern computer systems, focusing on the interaction between hardware and the operating system.

\subsection{Process Description and Control}
Processes are a central concept in operating systems. This section covers:
\begin{itemize}
    \item Process states and state transitions
    \item Process control block (PCB)
    \item Context switching
\end{itemize}

\subsection{Scheduling Algorithms}
This section covers:
\begin{itemize}
    \item First-Come, First-Served (FCFS)
    \item Shortest Job Next (SJN)
    \item Round Robin (RR)
\end{itemize}
It explains how these algorithms are used to allocate CPU time to processes.

\subsection{Process Creation and Termination}
Details how processes are created and terminated by the operating system, including:
\begin{itemize}
    \item Process spawning
    \item Process termination conditions
\end{itemize}

\subsection{Introduction to Threads}
This section introduces the concept of threads and their relation to processes, covering:
\begin{itemize}
    \item Single-threaded vs. multi-threaded processes
    \item Benefits of multithreading
\end{itemize}

\subsection{File Systems}
File systems provide a way for the operating system to store, retrieve, and manage data. This section explains:
\begin{itemize}
    \item File system structure
    \item File access methods
    \item Directory management
\end{itemize}

\subsection{Input and Output Management}
Input and output management is key for handling the interaction between the system and external devices. This section includes:
\begin{itemize}
    \item Device drivers
    \item I/O scheduling
\end{itemize}

\subsection{Deadlock Introduction and Prevention}
Explores the concept of deadlocks and methods for preventing them:
\begin{itemize}
    \item Deadlock introduction
    \item Conditions for deadlock
    \item Deadlock detection and recovery
    \item Deadlock prevention techniques
\end{itemize}

\subsubsection{Deadlock in Everyday Life}
Deadlock is not just a concept in computer systems; it can also occur in daily situations where multiple parties are waiting for each other to release resources, leading to a standstill. Here are a few examples of deadlock in real life:

\begin{itemize}
    \item \textbf{Traffic Jam at an Intersection:} Imagine a four-way intersection with cars coming from all directions. If each car enters the intersection and blocks the others, no one can move forward. This is a classic example of deadlock, where each car is waiting for the others to move.
    
    \item \textbf{Diners Waiting for Cutlery:} In a restaurant, if several diners are sharing a set of utensils, deadlock can occur if each diner holds onto one item (e.g., one diner holds a knife and waits for a fork, while the other holds a fork and waits for the knife). Neither can proceed without the other.
    
    \item \textbf{Locked Doors and Keys:} Two people are trying to get into separate rooms, but each room’s key is in the other room. If person A is waiting for person B to open their room to get the key, and person B is waiting for person A to do the same, they will both be stuck indefinitely.
    
    \item \textbf{Shared Resources at the Workplace:} In a collaborative work environment, if two teams are waiting for shared resources like a printer or project approval from another team, neither can proceed until the other has finished, leading to a deadlock in the workflow.
\end{itemize}

\subsection{User Interface Management}
This section discusses the role of the operating system in managing the user interface. Topics covered include:
\begin{itemize}
    \item Graphical User Interface (GUI)
    \item Command-Line Interface (CLI)
    \item Interaction between the user and the operating system
\end{itemize}

\subsection{Virtualization in Operating Systems}
Virtualization allows multiple operating systems to run concurrently on a single physical machine. This section explores:
\begin{itemize}
    \item Concept of virtualization
    \item Hypervisors and their types
    \item Benefits of virtualization in modern computing
\end{itemize}

\section{Assignments and Practical Work}
\subsection{Assignment 1: Process Scheduling}
Students were tasked with implementing various process scheduling algorithms (e.g., FCFS, SJN, and RR) and comparing their performance under different conditions.

\subsection{Assignment 2: Deadlock Handling}
In this assignment, students were asked to simulate different deadlock scenarios and explore various prevention methods.

\subsection{Assignment 3: Multithreading and Amdahl's Law}
This assignment involved designing a multithreading scenario to solve a computationally intensive problem. Students then applied *Amdahl's Law* to calculate the theoretical speedup of the program as the number of threads increased.

\subsection{Assignment 4: Simple Command-Line Interface (CLI) for User Interface Management}
Students were tasked with creating a simple *CLI* for user interface management. The CLI should support basic commands such as file manipulation (creating, listing, and deleting files), process management, and system status reporting.

\subsection{Assignment 5: Akses File Sistem}
Dalam assignment ini, siswa diminta untuk melakukan akses file sistem menggunakan Python. Soal ini mencakup pembuatan file, membaca isi file, dan menampilkan daftar file dalam direktori. Berikut adalah soal yang harus dikerjakan:

\textbf{Soal:} Buatlah sebuah program Python yang melakukan hal berikut:
1. Membuat sebuah file baru dengan nama `contoh.txt` dan menuliskan beberapa kalimat ke dalamnya.
2. Membaca dan menampilkan isi dari file `contoh.txt`.
3. Menampilkan semua file yang ada di direktori saat ini.

\textbf{Kode Python:}
\begin{lstlisting}[language=Python]
import os

# 1. Membuat file dan menuliskan isi
with open('contoh.txt', 'w') as file:
    file.write('Ini adalah contoh file.\n')
    file.write('File ini digunakan untuk demonstrasi akses file sistem.\n')

# 2. Membaca dan menampilkan isi file
with open('contoh.txt', 'r') as file:
    isi_file = file.read()
    print('Isi dari contoh.txt:')
    print(isi_file)

# 3. Menampilkan semua file di direktori saat ini
print('\nDaftar file di direktori saat ini:')
for nama_file in os.listdir('.'):
    print(nama_file)
\end{lstlisting}

\textbf{Jawaban:} Ketika kode di atas dijalankan, output yang diharapkan adalah sebagai berikut:
\begin{verbatim}
Isi dari contoh.txt:
Ini adalah contoh file.
File ini digunakan untuk demonstrasi akses file sistem.

Daftar file di direktori saat ini:
contoh.txt
\end{verbatim}


\section{Conclusion}
The first half of the course introduced core operating system concepts, including process management, scheduling, multithreading, and file system access. These topics provided a foundation for more advanced topics to be covered in the second half of the course.

\end{document}




