\documentclass[12pt]{article}
\usepackage{amsmath}
\usepackage{graphicx}
\usepackage{hyperref}
\usepackage{listings}
\usepackage{color}
\usepackage{pythonhighlight}

\title{Operating System Course Report - First Half of the Semester}
\author{A class}
\date{\today}

\begin{document}

\maketitle
\newpage

\tableofcontents
\newpage

\section{Introduction}
This report summarizes the topics covered during the first half of the Operating System course. It includes theoretical concepts, practical implementations, and assignments. The course focuses on the fundamentals of operating systems, including system architecture, process management, CPU scheduling, and deadlock handling.

\section{Course Overview}
\subsection{Objectives}
The main objectives of this course are:
\begin{itemize}
    \item To understand the basic components and architecture of a computer system.
    \item To learn process management, scheduling, and inter-process communication.
    \item To explore file systems, input/output management, and virtualization.
    \item To study the prevention and handling of deadlocks in operating systems.
\end{itemize}

\subsection{Course Structure}
The course is divided into two halves. This report focuses on the first half, which covers:
\begin{itemize}
    \item Basic Concepts and Components of Computer Systems
    \item System Performance and Metrics
    \item System Architecture of Computer Systems
    \item Process Description and Control
    \item Scheduling Algorithms
    \item Process Creation and Termination
    \item Introduction to Threads
    \item File Systems
    \item Input and Output Management
    \item Deadlock Introduction and Prevention
    \item User Interface Management
    \item Virtualization in Operating Systems
\end{itemize}

\section{Topics Covered}

\subsection{Basic Concepts and Components of Computer Systems}
This section explains the fundamental components that make up a computer system, including the CPU, memory, storage, and input/output devices.

\subsection{System Performance and Metrics}
This section introduces various system performance metrics used to measure the efficiency of a computer system, including throughput, response time, and utilization.

\subsection{System Architecture of Computer Systems}
Describes the architecture of modern computer systems, focusing on the interaction between hardware and the operating system.

\subsection{Process Description and Control}
Processes are a central concept in operating systems. This section covers:
\begin{itemize}
    \item Process states and state transitions
    \item Process control block (PCB)
    \item Context switching
\end{itemize}

\subsection{Scheduling Algorithms}
This section covers:
\begin{itemize}
    \item First-Come, First-Served (FCFS)
    \item Shortest Job Next (SJN)
    \item Round Robin (RR)
\end{itemize}
It explains how these algorithms are used to allocate CPU time to processes.

\subsection{Process Creation and Termination}
Details how processes are created and terminated by the operating system, including:
\begin{itemize}
    \item Process spawning
    \item Process termination conditions
\end{itemize}

\subsection{Introduction to Threads}
This section introduces the concept of threads and their relation to processes, covering:
\begin{itemize}
    \item Single-threaded vs. multi-threaded processes
    \item Benefits of multithreading
\end{itemize}

\begin{figure}[h]
    \centering
    \includegraphics[width=0.5\textwidth]{/Users/khawaritzmi/Unhas/os_report_mid2024/a_class/asset/example.png}  % Sesuaikan nama file dan ukurannya
    \caption{Ini adalah gambar contoh dari multithreading.}
    \label{fig:contoh_gambar}
\end{figure}

Seperti yang terlihat pada Gambar \ref{fig:contoh_gambar}, inilah cara menambahkan gambar dengan keterangan.

\subsection{File Systems}
File systems provide a way for the operating system to store, retrieve, and manage data. This section explains:
\begin{itemize}
    \item File system structure
    \item File access methods
    \item Directory management
\end{itemize}

\subsection{Input and Output Management}
Input and output management is key for handling the interaction between the system and external devices. This section includes:
\begin{itemize}
    \item Device drivers
    \item I/O scheduling
\end{itemize}

\subsection{Deadlock Introduction and Prevention}
Explores the concept of deadlocks and methods for preventing them:
\begin{itemize}
    \item Deadlock conditions
    \end{itemize}
    \subsection{Deadlock Prevention} 
\hspace{1cm} \textit{Deadlock Prevention} adalah teknik yang digunakan untuk mencegah sistem memasuki keadaan buntu (\textit{deadlock}). Dengan berusaha memastikan setidaknya  satu dari keempat kondisi \textit{deadlock} tidak terpenuhi. 4 kondisi yang menyebabkan terjadinya \textit{deadlock}, yaitu:
\begin{itemize}
    \item Mutual Exclusion : Kondisi ini terjadi ketika sumber daya hanya dapat digunakan oleh satu proses pada satu waktu. Jadi, ketika satu proses sedang menggunakan sumber daya, sumber daya tersebut tidak bisa digunakan oleh proses lain. Contoh sederhana proses terjadinya \textit{mutual exclusion} adalah pada \textit{printer}, dimana \textit{printer} hanya dapat digunakan oleh satu proses pada satu waktu. Jika proses A sedang mencetak, proses B harus menunggu sampai proses A selesai melakukan tugasnya.
    
    \item Hold and Wait: Ketika sebuah proses sudah memegang satu atau lebih sumber daya dan menunggu untuk mendapatkan sumber daya tambahan yang sedang dipegang oleh proses lain. Proses tersebut tidak melepaskan sumber daya yang sudah dipegangnya saat menunggu sumber daya tambahan. Ssederhananya, kondisi \textit{Hold-and-Wait} ini adalah proses menahan sumber daya sambil menuggu sumber daya lain.  Kondisi ini dapat menyebabkan situasi di mana proses lain juga menahan sumber daya yang dibutuhkan sehingga menciptakan siklus saling menunggu.
    
    \item No Preemption:  Sumber daya yang sudah dialokasikan untuk sebuah proses tidak bisa diputuskan atau diambil secara paksa. Sumber daya hanya dapat dilepaskan secara sukarela oleh proses yang memegangnya setelah selesai menggunakannya. Jika suatu proses menunggu sumber daya yang dipegang proses lain, dan sumber daya tersebut tidak bisa diambil paksa, maka proses pertama akan terjebak menunggu selamanya. Kondisi seperti inilah yang bisa menyebabkan \textit{deadlock} itu terjadi.
    
    \item Circular Wait:  Kondisi di mana serangkaian proses membentuk sebuah lingkaran, di mana setiap proses dalam lingkaran menunggu sumber daya yang sedang dipegang oleh proses lain. Siklus ini mengakibatkan tidak ada proses yang bisa melanjutkan karena semuanya saling menunggu sumber daya yang tidak akan pernah dilepaskan.
    Dalam \textit{circular wait}, proses saling menunggu satu sama lain, membentuk siklus yang mencegah kemajuan setiap proses. Misalkan, ada empat proses (P1, P2, P3, P4) dan empat sumber daya atau \textit{resource} (R1, R2, R3, R4). \textit{Circular wait} terjadi jika:

    - P1 memegang R1 dan menunggu R2 (yang dipegang oleh P2),
    
    - P2 memegang R2 dan menunggu R3 (yang dipegang oleh P3),
    
    - P3 memegang R3 dan menunggu R4 (yang dipegang oleh P4),
    
    - P4 memegang R4 dan menunggu R1 (yang dipegang oleh P1).
\end{itemize}
\begin{enumerate}

\end{enumerate}

\begin{thebibliography}{5}
    \bibitem  {Dhiraj2012} Dhiraj, G., & V. K., G. (2012). Approaches for Deadlock Detection and Deadlock Prevention Distributed systems. \textit{Resource Journal of Recent Sciences}, 1(ISC-2011), 422-425.
    \bibitem {Habermann1969} Habermann, A. N. (1969). Prevention of system deadlocks. \textit{Communications of the ACM}, 12(7), 373-ff.

    \end{thebibliography}


\subsection{User Interface Management}
This section discusses the role of the operating system in managing the user interface. Topics covered include:
\begin{itemize}
    \item Graphical User Interface (GUI)
    \item Command-Line Interface (CLI)
    \item Interaction between the user and the operating system
\end{itemize}

\subsection{Virtualization in Operating Systems}
Virtualization allows multiple operating systems to run concurrently on a single physical machine. This section explores:
\begin{itemize}
    \item Concept of virtualization
    \item Hypervisors and their types
    \item Benefits of virtualization in modern computing
\end{itemize}

\section{Assignments and Practical Work}
\subsection{Assignment 1: Process Scheduling}
Students were tasked with implementing various process scheduling algorithms (e.g., FCFS, SJN, and RR) and comparing their performance under different conditions.
\subsubsection{Group 1}
\begin{python}
    class Process:
    def __init__(self, pid, arrival_time, burst_time):
        self.pid = pid
        self.arrival_time = arrival_time
        self.burst_time = burst_time
        self.completion_time = 0
        self.turnaround_time = 0
        self.waiting_time = 0
\end{python}

\begin{table}[htbp] % Optional: For floating position
    \centering
    \begin{tabular}{|c|c|c|} % Defines number of columns and alignment (c = center, l = left, r = right). '|' creates vertical lines.
    \hline
    Header 1 & Header 2 & Header 3 \\ % Column headers
    \hline
    Row 1, Column 1 & Row 1, Column 2 & Row 1, Column 3 \\ % First row of data
    \hline
    Row 2, Column 1 & Row 2, Column 2 & Row 2, Column 3 \\ % Second row of data
    \hline
    \end{tabular}
    \caption{Your table caption} % Optional: For adding a caption
    \label{tab:your_label} % Optional: For cross-referencing the table
\end{table}
\subsection{Assignment 2: Deadlock Prevention}
    \subsubsection{Deadlock handling prevention} Sebuah sistem operasi memiliki tiga jenis sumber daya: R1, R2, dan R3. Ada empat proses yang membutuhkan sumber daya tersebut, yaitu P1, P2, P3, dan P4. Setiap proses hanya dapat meminta satu sumber daya dalam satu waktu. Identifikasi jenis penyebab \textit{deadlock} dan jelaskan penyebab terjadinya \textit{deadlock} sertakan dalam bentuk kode \textit{python} dari kondisi \textit{deadlock} tersebut!

    \paragraph{Jawaban:} 
    \begin{itemize}
    \item Kondisi ini termasuk kedalam  \textit{Hold-and-Wait conditions}
    \item Karena siklus dari ketiga proses saling terjebak menuggu satu sama lain, sehingga tidak ada proses yang dapat melanjutkan eksekusinya.
    \item Kode \textit{python}:
\begin{python}
    import time
import threading

class AlokasiSumberDaya:
    def __init__(self):
        #Status sumber daya: False berarti tersedia, dan True berarti sedang digunakan
        self.sumber_daya = {'R1': False, 'R2': False, 'R3': False}
        #Lock untuk menghindari race condition
        self.lock = threading.Lock()  

    def minta_sumber_daya(self, proses, sumber_daya):
        with self.lock:
            if not self.sumber_daya[sumber_daya]:
    # Jika sumber daya tersedia, berikan kepada proses
                print(f"{proses} mendapatkan {sumber_daya}")
                self.sumber_daya[sumber_daya] = True
                return True
            else:
    # Jika sumber daya tidak tersedia, proses menunggu
                print(f"{proses} menunggu {sumber_daya}")
                return False

    def lepas_sumber_daya(self, proses, sumber_daya):
        with self.lock:
    #Proses melepaskan sumber daya
        print(f"{proses} melepaskan {sumber_daya}")
            self.sumber_daya[sumber_daya] = False

# Fungsi untuk setiap proses
def proses(proses_name, alokasi, permintaan):
    for sumber_daya in permintaan:
        while not alokasi.minta_sumber_daya(proses_name, sumber_daya):
            time.sleep(0.1)  # Tunggu sebentar sebelum mencoba lagi
    time.sleep(1)  # Simulasikan penggunaan sumber daya
    for sumber_daya in permintaan:
        alokasi.lepas_sumber_daya(proses_name, sumber_daya)

# Simulasi alokasi sumber daya
alokasi = AlokasiSumberDaya()

# Daftar permintaan sumber daya untuk setiap proses, disusun untuk menciptakan deadlock
permintaan_proses = {
    'P1': ['R1'],  # P1 meminta R1
    'P2': ['R2'],  # P2 meminta R2
    'P3': ['R3'],  # P3 meminta R3
    'P4': ['R1']   # P4 juga meminta R1 (yang sudah dimiliki oleh P1)
}

# Membuat thread untuk setiap proses
threads = []
for proses_name, permintaan in permintaan_proses.items():
    t = threading.Thread(target=proses, args=(proses_name, alokasi, permintaan))
    threads.append(t)
    t.start()

# Menunggu semua thread selesai
for t in threads:
    t.join()
\end{python}

      
    \end{itemize}
    
\subsection{Assignment 3: Multithreading and Amdahl's Law}
This assignment involved designing a multithreading scenario to solve a computationally intensive problem. Students then applied **Amdahl's Law** to calculate the theoretical speedup of the program as the number of threads increased.

\subsection{Assignment 4: Simple Command-Line Interface (CLI) for User Interface Management}
Students were tasked with creating a simple **CLI** for user interface management. The CLI should support basic commands such as file manipulation (creating, listing, and deleting files), process management, and system status reporting.

\subsection{Assignment 5: File System Access}
In this assignment, students implemented file system access routines, including:
\begin{itemize}
    \item File creation and deletion
    \item Reading from and writing to files
    \item Navigating directories and managing file permissions
\end{itemize}

\section{Conclusion}
The first half of the course introduced core operating system concepts, including process management, scheduling, multithreading, and file system access. These topics provided a foundation for more advanced topics to be covered in the second half of the course.

\end{document}