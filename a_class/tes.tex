\subsubsection{\textit{Interfacing Input/Output}}
\textit{Interfacing I/O} adalah peralatan yang digunakan untuk menghubungkan suatu piranti dengan CPU melalui \textit{bus}. Tujuannya adalah mengontrol dan mentransfer data antara perangkat I/O dan CPU/memori melalui \textit{bus} dan \textit{controller}.
Komponen \textit{Interfacing I/O}:
\begin{itemize}
    \item Perangkat I/O: \textit{keyboard, mouse, monitor, printer,} dll.
    \item I/O \textit{Controller}: Perangkat keras yang mengontrol dan mengelola komunikasi antara 
    \item \textit{Bus}: Jalur komunikasi data antara perangkat I/O dan sistem komputer.
    \item \textit{Driver Software}: Perangkat lunak yang memungkinkan sistem operasi mengenali dan mengontrol perangkat I/O.
\end{itemize}

\subsubsection{Sistem Prosesor}
\begin{enumerate}
    \item {Saluran I/O}
Merupakan sebuah prosesor khusus dengan kemampuan terbatas yang disusun untuk interface beberapa piranti I/O ke memori. Saluran I/O dapat melakukan pendeteksian dan pembetulan kesaIahan dan beroperasi dalam basis \textit{cycle stealing}. Saluran I/O berkomunikasi dengan CPU sebagai suatu fasiIitas DMA dan berkomunikasi dengan piranti I/O seolah¬-olah sebuah CPU. Karena piranti I/O mempunyai kecepatan transfer yang berbeda-beda, maka saluran dibagi menjadi 3 pelayanan, yaitu:
\begin{enumerate}
    \item Saluran \textit{Multiplexer} \\
    Digunakan untuk menghubungkan piranti yang berkecepatan rendah dan sedang serta mengoperasikannya secara bersamaan dengan \textit{multiplexing}.
    \item  Saluran \textit{Selector} \\
    Digunakan untuk menghubungkan piranti I/O yang berkecepatan tinggi tanpa \textit{multiplexing}. Contoh: pita magnetis, \textit{disk}.
    \item Saluran \textit{Multiplexer} Blok \\
    Merupakan kombinasi dari dua pelayanan diatas.
\end{enumerate}
\end{enumerate}


\subsubsection{Prosesor I/O}
Merupakan komputer umum yang berkomunikasi dengan memori utama melalui fasilitas DMA system bus dan dengan piranti I/O atas satu atau lebih bus I/O.
Ada 2 mode yaitu:
\begin{enumerate}
    \item \textit{Single Shared bus} \\
    Setiap IOP mengendalikan sejumlah piranti I/O tertentu yang tetap.
\begin{figure}[h]
    \centering
    \includegraphics[width=0.5\textwidth]{/asset/Single.png}  % Sesuaikan nama file dan ukurannya
    \caption{Single Matrix bus}
    \label{fig:contoh_gambar}
\end{figure}
    \item \textit{Switching Matriks bus} \\
    Setiap IOP mengendalikan satu piranti I/O
    \begin{figure}[h]
    \centering
    \includegraphics[width=0.5\textwidth]{/asset/Switching.png}  % Sesuaikan nama file dan ukurannya
    \caption{Switching Matriks bus}
    \label{fig:contoh_gambar}
\end{figure}
    \item Konfigurasi \textit{Multiprosesor} \\
    Di dalam satu komputer seakan-akan terdapat beberapa \textit{mikroprosesor}, meskipun sebenarnya \textit{mikroprosesor} utamanya hanya satu, sedangkan yang Iainnya berupa prosesor I/O (lOP). Hubungan yang paling sederhana menggunakan \textit{common bus}.
    \begin{figure}[h]
    \centering
    \includegraphics[width=0.5\textwidth]{/asset/Konfigurasi.png}  % Sesuaikan nama file dan ukurannya
    \caption{Ini adalah gambar contoh dari multithreading.}
    \label{fig:contoh_gambar}
\end{figure}
    \begin{itemize}
    \item \textit{Bus} umum bersifat membagi waktu (\textit{time shared}) oleh semua prosesor dan hanya satu prosesor yang dapat mengakses memori pada waktu tertentu.Tetapi dapat juga menggunakan \textit{bus} umum ke dalam organisasi \textit{multiprosesor dual bus}.
    \item Setiap komputer dihubungkan suatu pengendali sistem ke \textit{bus} umum.
    \item Komunikasi interkomputer ini dilakukan pada sistem \textit{bus} melalui memori umum.
\end{itemize}
\begin{figure}[h]
    \centering
    \includegraphics[width=0.5\textwidth]{/asset/systemBus.png}  % Sesuaikan nama file dan ukurannya
    \caption{System bus}
    \label{fig:contoh_gambar}
\end{figure}
\end{enumerate}

\subsection{Assignment 3: Multithreading and Amdahl's Law}
This assignment involved designing a multithreading scenario to solve a computationally intensive problem. Students then applied **Amdahl's Law** to calculate the theoretical speedup of the program as the number of threads increased.\\
\\
Pertanyaan:
Misalkan sebuah program terdiri dari 80\% bagian yang dapat diparalelisasi, sedangkan 20\% tidak dapat diparalelisasi. Berdasarkan Hukum Amdahl, hitung percepatan teoritis yang dapat dicapai jika program tersebut dijalankan dengan 2, 4, 8, dan 16 threads. Tentukan juga batas atas percepatan teoritis saat jumlah threads mendekati tak terbatas.

Gunakan rumus berikut untuk menghitung percepatan teoritis:

\[
S(N) = \frac{1}{(1 - P) + \frac{P}{N}}
\]

Dimana:
\begin{itemize}
    \item \( S(N) \) adalah percepatan teoritis dengan \( N \) threads,
    \item \( P \) adalah bagian program yang dapat diparalelisasi,
    \item \( N \) adalah jumlah threads.
\end{itemize}

\begin{python}
    # Menghitung percepatan teoritis menggunakan Hukum Amdahl
def amdahl_law(P, N):
    return 1 / ((1 - P) + (P / N))

# Persentase bagian program yang dapat diparalelisasi
P = 0.80  # 80%

# Daftar jumlah thread yang digunakan
threads = [2, 4, 8, 16, float('inf')]

# Menghitung dan menampilkan percepatan untuk setiap jumlah thread
for N in threads:
    speedup = amdahl_law(P, N)
    print(f"Percepatan teoritis dengan {int(N) if N != float('inf') else 'tak terbatas'} threads: {speedup:.2f}")
\end{python}

\textbf{Penjelasan Output:}

Setelah menjalankan kode di atas, berikut adalah hasil percepatan teoritis (\(S(N)\)) untuk berbagai jumlah threads:

\begin{itemize}
    \item Percepatan dengan 2 threads: 1.33
    \item Percepatan dengan 4 threads: 1.60
    \item Percepatan dengan 8 threads: 1.78
    \item Percepatan dengan 16 threads: 1.88
    \item Percepatan dengan jumlah threads yang tak terbatas: 2.00
\end{itemize}

Hasil ini menunjukkan bahwa seiring bertambahnya jumlah threads, percepatan teoritis meningkat, tetapi tidak secara linear. Sesuai dengan Hukum Amdahl, percepatan maksimal dibatasi oleh bagian dari program yang tidak dapat diparalelisasi. Pada contoh ini, batas atas percepatan adalah 2.0, yang tercapai saat jumlah threads mendekati tak terbatas. Hal ini disebabkan oleh 20\% dari program yang tidak bisa diparalelisasi.