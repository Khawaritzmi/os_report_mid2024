\documentclass[12pt]{article}
\usepackage{amsmath}
\usepackage{graphicx}
\usepackage{hyperref}
\usepackage{listings}
\usepackage{color}
\usepackage{float}

\lstset{
    language=Python,                      % Set bahasa ke Python
    basicstyle=\ttfamily\footnotesize,    % Ukuran dan font kode
    keywordstyle=\color{blue},            % Warna keyword
    stringstyle=\color{red},              % Warna string
    commentstyle=\color{green},           % Warna komentar
    numbers=left,                         % Menampilkan nomor baris
    numberstyle=\tiny,                    % Ukuran nomor baris
    stepnumber=1,                         % Setiap baris diberi nomor
    breaklines=true,                      % Pemenggalan baris otomatis
    frame=single,                         % Bingkai di sekitar kode
    tabsize=4                             % Ukuran tab
}

\title{Operating System Course Report - First Half of the Semester}
\author{B class}
\date{\today}

\begin{document}

\maketitle
\newpage

\tableofcontents
\newpage

\section{Introduction}
This report summarizes the topics covered during the first half of the Operating System course. It includes theoretical concepts, practical implementations, and assignments. The course focuses on the fundamentals of operating systems, including system architecture, process management, CPU scheduling, and deadlock handling.

\section{Course Overview}
\subsection{Objectives}
The main objectives of this course are:
\begin{itemize}
    \item To understand the basic components and architecture of a computer system.
    \item To learn process management, scheduling, and inter-process communication.
    \item To explore file systems, input/output management, and virtualization.
    \item To study the prevention and handling of deadlocks in operating systems.
\end{itemize}

\subsection{Course Structure}
The course is divided into two halves. This report focuses on the first half, which covers:
\begin{itemize}
    \item Basic Concepts and Components of Computer Systems
    \item System Performance and Metrics
    \item System Architecture of Computer Systems
    \item Process Description and Control
    \item Scheduling Algorithms
    \item Process Creation and Termination
    \item Introduction to Threads
    \item File Systems
    \item Input and Output Management
    \item Deadlock Introduction and Prevention
    \item User Interface Management
    \item Virtualization in Operating Systems
\end{itemize}

\section{Topics Covered}

\subsection{Basic Concepts and Components of Computer Systems}
This section explains the fundamental components that make up a computer system, including the CPU, memory, storage, and input/output devices.

\subsection{System Performance and Metrics}
\subsubsection{\textit{Performance Metrics} Pada \textit{CPU}}
\hspace*{1cm} Jika kita membahas kinerja matriks pada komputer, maka kita tidak bisa lepas dari CPU. CPU atau \textit{Central Processing Unit} adalah komponen utama dari komputer yang bertanggung jawab mengeksekusi instruksi-instruksi yang diberikan kepada komputer. Kinerja suatu sistem sangat dipengaruhi oleh cara CPU mengeksekusi instruksi. Untuk memaksimalkan kinerja suatu sistem, kita tentu perlu meminimalkan waktu eksekusi, karena kinerja berbanding terbalik dengan waktu eksekusi.
\newline
\newline
\hspace*{1cm} Untuk menentukan waktu eksekusi CPU untuk suatu program, Anda dapat mencari tahu jumlah total \textit{\textit{clock} cycle} yang dibutuhkan program dan mengalikannya dengan waktu \textit{\textit{clock} cycle}. Setiap program terdiri dari sejumlah instruksi, dan setiap instruksi membutuhkan sejumlah \textit{\textit{clock} cycle} untuk dieksekusi. Jika Anda mengetahui jumlah total \textit{\textit{clock} cycle} per program serta mengetahui waktu \textit{\textit{clock} cycle} untuk setiap \textit{\textit{clock} cycle}, maka waktu eksekusi CPU dapat dihitung sebagai hasil perkalian jumlah total \textit{\textit{clock} cycle} CPU per program dengan waktu \textit{\textit{clock} cycle}. Karena waktu \textit{\textit{clock} cycle} dan \textit{\textit{clock} rate} saling terkait, hal ini juga dapat ditulis sebagai jumlah \textit{\textit{clock} cycle} CPU untuk suatu program dibagi dengan \textit{\textit{clock} rate}.

\begin{figure}
    \centering
    \includegraphics[width=\linewidth]{asset/image1.png}
    \caption{Rumus waktu CPU}
\end{figure}


\hspace*{1cm} Karena waktu eksekusi CPU merupakan hasil dari kedua faktor ini, kita dapat meningkatkan kinerja dengan mengurangi durasi waktu \textit{clock cycle} atau jumlah \textit{clock cycle} yang diperlukan untuk suatu program. Kecepatan \textit{clock} pada dasarnya bergantung pada organisasi CPU tertentu.

\subsubsection{Mengapa \textit{Performance Metrics} Penting}
\hspace*{1cm} Mengukur kinerja menggunakan data yang akurat sangat penting dalam sebuah sistem karena memberikan gambaran nyata tentang performa yang sedang berlangsung. Dengan data yang tepat, manajemen dapat membuat keputusan yang lebih terinformasi, sehingga memperkecil risiko kesalahan dalam strategi. Selain itu, pengukuran kinerja juga membantu mengidentifikasi area yang memerlukan perbaikan, sehingga organisasi dapat berfokus pada peningkatan yang signifikan. Memantau perkembangan dan kemajuan kinerja secara berkala meningkatkan akuntabilitas, baik di tingkat individu maupun tim, sehingga setiap anggota lebih bertanggung jawab atas hasil kerjanya. Pada akhirnya, penggunaan \textit{performance metrics} yang tepat dapat mendorong peningkatan berkelanjutan dengan memberikan dasar yang kuat untuk evaluasi dan inovasi secara sistematis dalam mencapai tujuan jangka panjang.

\subsubsection{Fungsi \textit{Performance Metrics} pada Sistem}
\hspace*{1cm} \textit{Performance metrics} berperan penting dalam berbagai aspek sistem operasi. Berikut adalah beberapa area di mana \textit{performance metrics} digunakan:
\begin{itemize}
    \item \textbf{\textit{Benchmark}} \newline
    \hspace*{1cm} \textit{Benchmark} biasanya menggunakan berbagai \textit{performance metrics} untuk mengukur dan melaporkan hasil kinerjanya. Misalnya, sebuah \textit{benchmark} untuk prosesor mungkin mengukur metrik seperti kecepatan \textit{clock}, jumlah instruksi per detik (\textit{IPS}), atau waktu komputasi untuk tugas tertentu.
    \item \textbf{\textit{Bandwidth}} \newline  
    \hspace*{1cm} Mengukur jumlah data yang dapat ditransmisikan melalui jaringan dalam jangka waktu tertentu. \textit{Bandwidth} yang terbatas membatasi kemampuan sistem untuk menangani banyak permintaan atau mentransfer data dalam jumlah besar, yang pada akhirnya memperlambat waktu muat atau respons. Pada sistem yang sangat bergantung pada transfer data, seperti aplikasi berbasis web atau layanan \textit{cloud}, \textit{bandwidth} adalah faktor penting yang memengaruhi performa keseluruhan.
    \item \textbf{\textit{Error Rate}} \newline  
    \hspace*{1cm} Menghitung persentase permintaan yang gagal diproses atau menghasilkan kesalahan. Metrik ini penting untuk menilai stabilitas dan keandalan sistem. Tingkat kesalahan yang tinggi menunjukkan masalah pada sistem, seperti bug dalam perangkat lunak, kesalahan konfigurasi, atau keterbatasan sumber daya. Tingkat kesalahan yang tinggi dapat mengganggu pengguna dan menurunkan kepercayaan mereka terhadap sistem.
    \item \textbf{Pengaturan Kualitas Layanan (\textit{QoS})} \newline  
    \hspace*{1cm} Dalam lingkungan yang membutuhkan kualitas layanan tertentu, \textit{performance metrics} digunakan untuk memastikan bahwa aplikasi dan layanan memenuhi \textit{SLA} (\textit{Service Level Agreement}). Metrik seperti latensi, \textit{throughput}, dan waktu respons digunakan untuk mengatur dan memantau \textit{QoS}.
    \item \textbf{Optimasi dan Tuning} \newline  
    \hspace*{1cm} \textit{Performance metrics} digunakan untuk mengidentifikasi peluang optimasi dan tuning. Misalnya, dengan
\end{itemize}

\hspace*{1cm} \textit{Performance metrics} berperan penting dalam berbagai aspek pengelolaan dan operasional sistem operasi. Dari monitoring dan manajemen sumber daya hingga optimasi, diagnosis, dan keamanan, metrik ini memberikan wawasan yang diperlukan untuk menjaga sistem tetap berjalan dengan efisien, aman, dan sesuai dengan kebutuhan kinerja yang diharapkan.

\subsection{System Architecture of Computer Systems}
Describes the architecture of modern computer systems, focusing on the interaction between hardware and the operating system.

\subsection{Process Description and Control}
Processes are a central concept in operating systems. This section covers:
\begin{itemize}
    \item Process states and state transitions
    \item Process control block (PCB)
    \item Context switching
\end{itemize}

\subsection{Scheduling Algorithms}
This section covers:
\begin{itemize}
    \item First-Come, First-Served (FCFS)
    \item Shortest Job Next (SJN)
    \item Round Robin (RR)
\end{itemize}
It explains how these algorithms are used to allocate CPU time to processes.

\subsection{Process Creation and Termination}
Details how processes are created and terminated by the operating system, including:
\begin{itemize}
    \item Process spawning
    \item Process termination conditions
\end{itemize}

\subsection{Introduction to Threads}
This section introduces the concept of threads and their relation to processes, covering:
\begin{itemize}
    \item Single-threaded vs. multi-threaded processes
    \item Benefits of multithreading
\end{itemize}

\subsection{File Systems}
File systems provide a way for the operating system to store, retrieve, and manage data. This section explains:
\begin{itemize}
    \item File system structure
    \item File access methods
    \item Directory management
\end{itemize}

\subsection{Input and Output Management}

\subsubsection{\textit{Interrupt Driven} I/O}
   \begin{itemize}
    \item Metode di mana perangkat I/O mengirimkan sinyal interupsi ke CPU
    saat siap melakukan transfer data. Berbeda dengan \textit{polled} I/O yang mengharuskan CPU secara terus-menerus memeriksa status perangkat, dalam \textit{interrupt-driven} I/O, CPU dapat fokus pada tugas lain hingga diinterupsi oleh perangkat I/O.
    \item Prinsip mekanisme transfer data pada \textit{interrupt-driven} I/O sama dengan \textit{programmed} I/O, yaitu CPU tetap mempertukarkan data dengan \textit{device} I/O melalui beberapa register CPU. Maka antarmuka I/O menyediakan \textit{control}, status, dan \textit{port} data. Perbedaan yang prinsip adalah siapa yang bertanggung jawab untuk berinisiatif melakukan transfer data. Pada\textit{ programmed }I/O tanggung jawabnya ada pada CPU. Dengan kata lain, program I/O harus sering melakukan pemeriksaan \textit{(scan)} status\textit{ device }I/O untuk menentukan apakah \textit{device} siap melakukan transfer data. Pada \textit{interrupt-driven} I/O, data transfer dimulai (atas inisiatif) \textit{device }I/O, yang menggunakan mekanisme interupsi untuk memberitahukan CPU tentang kesiapannya. Mekanisme ini menghilangkan beban \textit{scanning} status. \textit{Device} I/O juga dapat menggunakan mekanisme interupsi untuk keperluan yang lain. Sebagai contoh untuk menarik perhatian CPU pada saat terjadi kesalahan/ kerusakkan, atau untuk menunjukkan selesainya operasi lokal.\\
    Tanpa memperhatikan tipenya, apabila menerima interupsi, CPU akan:
\begin{enumerate}
    \item Menagguhkan eksekusi program yang sedang berlangsung.
    \item Menyimpan statusnya.
    \item Melompat ke \textit{interrupt service routine }(ISR).
    \item Kembali untuk melanjutkan eksekusi program yang diinterupsi.
\end{enumerate}
\end{itemize}
\subsubsection{Jenis Interupsi}
\begin{enumerate}
    \item \textit{Maskable interupts.}
    \item \textit{Non-Maskable interrupts.}
\end{enumerate}
\subsubsection{Struktur interupsi pada prosesor}
\begin{enumerate}
    \item Interupsi \textit{maskable} dan \textit{nonmaskable} dengan jalur terpisah.
\item Interupsi \textit{maskable} dan \textit{nonmaskable} dengan jalur bersama.\\
    Pola pertama berbasis pada dua jalur permintaan interupsi\textit{ (interrupt request)} yang terpisah, satu untuk interupsi\textit{ maskable} (INTREQ) dan satu lagi untuk \textit{nonmaskable} (NMINT). Jalur INTERQ biasanya berpasangan dengan jalur \textit{interrupt acknowledge }(INTACK). Melalui jalur ini CPU menjawab penerimaan permintaan interupsi dan menginstrusksikan \textit{device} I/O untuk meletakkan kode alamat ke bus data. Kode alamat menunjukkan CPU ke ISR yang mentransfer data atau melakukan pelayanan yang lainya atas nama \textit{device} I/O. INTERQ dan INTACK dapat dipandang sebagai pasangan jalur jabat tangan \textit{(handshaking)}. Beberapa CPU tidak menyediakan sinyal jawaban interupsi pada jalur yang terpisah, tetapi melakukan \textit{encode} siklus ini pada jalur status. Dalam kasus ini, sinyal INTACK diperoleh dengan cara mengkode jalur status CPU. Interupsi \textit{non maskable} tidak memerlukan \textit{acknowledge }karena interupsi ini selalu diterima dan karena tidak ada kode alamat yang diberikan secara eksternal. Dengan demikian tidak diperlukan jalur \textit{acknowledge} untuk interupsi ini. Contoh prosesor yang menggunakan pola ini adalah 8086 dan 8088.
\end{enumerate}
\subsubsection{Pemrosesan Interupsi}
\begin{enumerate}
    \item Menangguhkan eksekusi program.
    \item Menyimpan status program \textit{(context)}.
    \item Melompat ke \textit{Interrupt Service Routine} (ISR).
    \item Kembali ke program yang diinterupsi.
\end{enumerate}
\subsubsection{Klasifikasi Perangkat I/O}
\begin{itemize}
    \item Sifat aliran datanya
\begin{enumerate}
    \item Perangkat berorientasi blok \\
    Yaitu menyimpan, menerima, dan mengirim informasi sebagai blok-blok berukuran tetap yang berukuran 128 sampai 1024 byte dan memiliki alamat tersendiri, sehingga memungkinkan membaca atau menulis blok-blok secara \textit{independen} (mandiri), yaitu dapat membaca atau menulis sembarang blok tanpa harus melewati blok-blok lain. Contoh : disk,tape,CD ROM,optical disk.
    \item Perangkat berorientasi aliran karakter \\
    Perangkat yang menerima, dan mengirimkan aliran karakter tanpa membentuk suatu struktur blok. Contoh : terminal, line printer, pita kertas, kartu-kartu berlubang, interface jaringan, mouse.
\end{enumerate}
\end{itemize}
\begin{itemize}
    \item Sasaran komunikasi 
\begin{enumerate}
    \item Perangkat yang terbaca oleh manusia \\
    Perangkat yang digunakan untuk berkomunikasi dengan manusia. 
    Contoh : VDT \textit{(video display terminal)} : monitor, keyboard, mouse.
    \item Perangkat yang terbaca oleh mesin 
    Perangkat yang digunakan untuk berkomunikasi dengan perangkat elektronik. Contoh : Disk dan tape, sensor, controller.
    \item perangkat komunikasi\\
    Perangkat yang digunakan untuk komunikasi dengan perangkat jarak jauh. Contoh : Modem.
\end{enumerate}
\end{itemize}
\subsubsection{Prinsip Manajemen Perangkat I/O}
\begin{itemize}
    \item Terdapat dua sasaran perancangan I/O
\begin{enumerate}
    \item Efisiensi \\
    Aspek penting karena operasi I/O sering menimbulkan \textit{bottleneck}.
    \item Generalitas\textit{(device independence)} \\
    Manajemen perangkat I/O selain berkaitan dengan simplisitas dan bebas kesalahan, juga menangani perangkat secara seragam baik dari cara proses memandang maupun cara sistem operasi mengelola perangkat dan operasi I/O.
\end{enumerate}
\end{itemize}
\subsubsection{Masalah-Masalah Manajemen I/O}
\begin{itemize}
    \item Penamaan yang seragam \textit{(uniform naming)} \\
    Nama berkas atau perangkat adalah \textit{string} atau \textit{integer}, tidak bergantung pada perangkat sama sekali. 
    \item Penanganan kesalahan\textit{(errorhandling)}\\
    Umumnya penanganan kesalahan ditangani sedekat mungkin dengan perangkat keras.
    \item \textit{Transfer sinkron vs asinkron} \\
    Kebanyakan I/O adalah asinkron. Pemroses memulai transfer serta mengabaikan untuk melakukan kerja lain sampai interupsi tiba. Program pemakai sangat lebih mudah ditulis jika operasi I/O berorientasi blok. Setelah perintah \textit{read}, program kemudian ditunda secara otomatis sampai data tersedia di \textit{buffer}.
    \item \textit{Sharable vs dedicated} \\
    Beberapa perangkat dapat dipakai bersama seperti \textit{disk}, tapi ada juga perangkat yang hanya satu pemakai yang dibolehkan memakai pada satu saat. Contoh : printer.
\end{itemize}
\subsubsection{Direct Memory Access}
\textit{Dirrect Memory Acces} atau DMA merupakan suatu alat pengendali khusus yang disediakan untuk memungkinkan transfer blok data langsung antar perangkat eksternal dan memori utama, tanpa interversi terus menerus dari prosesor. \\
\begin{figure}[h]
    \centering
    \includegraphics[width=0.80\linewidth]{b_class/asset/I_O.jpg}
    \caption{Struktur Modul I/O}
    \label{fig:enter-label}
\end{figure}
Penjelasan :
\begin{itemize}
    \item CPU Mengatur Pekerjaan
    Saat CPU memulai pekerjaan, CPU ingin memindahkan data dari disk (penyimpanan) ke RAM (memori). Untuk melakukan ini, CPU akan memberi perintah ke DMA \textit{Controller}. DMA berfungsi membantu CPU agar tidak terlalu sibuk.
    \item DMA Controller Beraksi
    Setelah mendapat perintah, DMA \textit{controller} mengambil alih pekerjaan untuk sementara. Dia akan membuat jalur khusus agar data dari disk bisa langsung masuk ke memori tanpa harus melalui CPU. Hal ini dilakukan agar CPU bisa tetap fokus pada tugas lain dan tidak perlu bolak-balik mengurus transfer data.
    \item Transfer Data
    DMA controller menggunakan PCI bus untuk memindahkan data dari disk ke RAM dengan menggunakan bantuan jalur-jalur khusus yang sudah ada. Dalam perjalanan ini, pada transfer data terdapat \textit{buffer} yang berfungsi sebagai tempat persinggahan sementara. \textit{Buffer} menampung data sementara sebelum data benar-benar sampai ke RAM.
    \item Pemberitahuan ke CPU
    Setelah semua data berhasil dipindahkan ke RAM, DMA \textit{controller} akan memberi tahu CPU dengan "sinyal interupsi." \\
    Inti dari diagram ini adalah menunjukkan bagaimana sistem komputer bisa bekerja lebih efisien menggunakan DMA. Dengan menggunakan DMA \textit{controller}, CPU tidak perlu sibuk mengurus transfer data. CPU cukup memberi perintah sekali, lalu DMA yang akan menangani semua proses transfer. Hasilnya, komputer bekerja lebih cepat dan efisien karena CPU bisa mengerjakan tugas lain tanpa terhambat.
\end{itemize}

\subsubsection{Metode DMA Dalam Mentransfer Data}
\begin{itemize}
    \item Metode \textit{HALT} atau \textit{BurstMode} DMA
    \item Mengikut sertakan pengendali DMA \\
    Pada dasarnya DMA mempunyai dua metode yang berbeda dalam mentransfer data, metode yang pertama ialah metode yang sangat baku dan sederhana disebut\textit{ HALT}, atau \textit{BurstMode} DMA, karena pengendali DMA memegang \textit{control} dari \textit{system} bus dan mentransfer semua blok data atau dari memori pada \textit{single burst}. Selagi transfer masih dalam proses, system mikroprosessor di-\textit{setidle}, tidak melakukan intruksi operasi untuk menjaga internal \textit{register}. Tipe operasi DMA seperti ini ada pada kebanyakan komputer. Metode yang kedua adalah mengikut sertakan pengendali DMA untuk memegang \textit{control} dari \textit{system }bus untuk jangka waktu yang lebih pendek pada periode dimana mikroprosessor sibuk dengan operasi internal dan tidak membutuhkan akses ke\textit{ system }bus. Metode DMA ini disebut dengan \textit{cycle stealing mode}.\textit{ Cycle stealing mode} DMA lebih kompleks untuk diimplementasikan dibandingkan \textit{HALT} DMA, karena pengendali DMA harus mempunyai kepintaran untuk merasakan waktu pada saat sistem bus terbuka.
\end{itemize}

\subsubsection{Referensi}
\begin{itemize}
    \item Dancok, D. C. (2024, Mei 23). manajemen I-O. Retrieved from scribd.com: https://www.scribd.com/document/735310900/3-Manajemen-I-O 
\end{itemize}

\subsection{Deadlock Introduction and Prevention}
Explores the concept of deadlocks and methods for preventing them:
\begin{itemize}
    \item Deadlock conditions
    \item Deadlock prevention techniques
\end{itemize}

\subsection{User Interface Management}
This section discusses the role of the operating system in managing the user interface. Topics covered include:
\begin{itemize}
    \item Graphical User Interface (GUI)
    \item Command-Line Interface (CLI)
    \item Interaction between the user and the operating system
\end{itemize}

\subsection{Virtualization in Operating Systems}
Virtualization allows multiple operating systems to run concurrently on a single physical machine. This section explores:
\begin{itemize}
    \item Concept of virtualization
    \item Hypervisors and their types
    \item Benefits of virtualization in modern computing
\end{itemize}


\section{Assignments and Practical Work}
\subsection{Assignment 1: Process Scheduling}
\hspace*{1cm} Implementasikan algoritma penjadwalan \textit{Shortest Job First} (SJF) dan \textit{Round Robin} (RR), lalu bandingkan rata-rata waktu tunggu (\textit{average waiting time}) dan rata-rata waktu penyelesaian (\textit{average turnaround time}) dari kedua algoritma tersebut. Manakah yang lebih baik dalam menangani proses? 
\newline
\newline

\begin{center}
    \underline{JAWABAN}
\end{center}

\textit{Short Job First (SJF) algorithms}
\begin{lstlisting}[language=Python]
    def sjf_non_preemptive(Jumlah_proses):
        Jumlah_proses.sort(key=lambda x: x[1])
    
        waktu_tunggu = 0
        total_waktu_tunggu = 0
    
        for pid, burst_time in Jumlah_proses:
            total_waktu_tunggu += waktu_tunggu
            waktu_tunggu += burst_time
    
        n = len(Jumlah_proses)
        avg_waktu_tunggu = total_waktu_tunggu / n
    
        print(total_waktu_tunggu)
        print(f"\nAverage Waiting Time: {avg_waktu_tunggu}")    
    
\end{lstlisting}

\textit{Round Robin (RR) algorithms}
\begin{lstlisting}[language=Python]
    def round_robin(jumlah_proses, quantum):
        n = len(jumlah_proses)
        remaining_time = [bt for _, bt in jumlah_proses]
        waktu_tunggu = [0] * n
        time = 0
    
        while True:
            done = True
            for i in range(n):
                if remaining_time[i] > 0:
                    done = False
                    if remaining_time[i] > quantum:
                        time += quantum
                        remaining_time[i] -= quantum
                    else:
                        time += remaining_time[i]
                        waktu_tunggu[i] = time - jumlah_proses[i][1]
                        remaining_time[i] = 0
            if done:
                break
    
        avg_waktu_tunggu = sum(waktu_tunggu) / n
        print(f"\nAverage Waiting Time: {avg_waktu_tunggu:.2f}")
    
\end{lstlisting}

\begin{figure}[H]
    \centering
    \includegraphics[width=1\linewidth]{asset/41.png}
    \caption{Output}
\end{figure}

\begin{figure}[H]
    \centering
    \includegraphics[width=1\linewidth]{asset/411.png}
    \caption{Perbandingan SJF dan RR(quantum = 2)}
\end{figure}

\paragraph{
    \hspace*{1cm} Dari hasil perbandingan rata-rata waktu tunggu pada kedua algoritma, dapat dilihat bahwa rata-rata waktu tunggu SJF lebih kecil dibandingkan dengan \textit{Round Robin}. Salah satu alasannya adalah karena sifat RR yang memberikan setiap proses sejumlah waktu yang sama (\textit{quantum}) dan memaksa semua proses menunggu giliran mereka secara bergantian. 
}

\paragraph{
    \hspace*{1cm} Kita dapat mengurangi kelemahan \textit{Round Robin} dengan memilih \textit{quantum} yang lebih besar. Pada visualisasi di atas terlihat bahwa waktu tunggu RR dapat sama dengan SJF. Namun, tetap saja, SJF sering kali lebih baik dalam kasus dengan banyak proses yang memiliki \textit{burst time} yang bervariasi.
}

\begin{figure}[H]
    \centering
    \includegraphics[width=1\linewidth]{asset/412.png}
    \caption{Perbandingan SJF dan RR(quantum = 8)}
\end{figure}


\subsection{Assignment 2: Deadlock Handling}
\hspace*{1cm} Simulasikan bagaimana \textit{algoritma safety} bekerja untuk mendeteksi \textit{deadlock}.
\newline
\newline

\begin{center}
    \underline{JAWABAN}
\end{center}

\begin{lstlisting}[language=Python]

    import numpy as np
    
    def is_safe(Jumlah_proses, avail, max_need, allocation):
        P, R = len(Jumlah_proses), len(avail) 
        need = max_need - allocation 
        finish = [False] * P 
        safe_sequence = []  
        work = avail.copy() 
    
        for _ in range(P):
            for p in range(P):
                if not finish[p] and all(need[p] <= work): 
                    work += allocation[p]
                    finish[p] = True
                    safe_sequence.append(p)
                    break
            else:
                print("Sistem tidak dalam keadaan aman.")
                return False, []
    
        print("Sistem dalam keadaan aman.")
        return True, safe_sequence
    
    
    Jumlah_proses = [0, 1, 2, 3, 4]
    avail = np.array([3, 3, 2]) 
    max_need = np.array([[7, 5, 3],
                         [3, 2, 2],
                         [9, 0, 2],
                         [2, 2, 2],
                         [4, 3, 3]]) 
    allocation = np.array([[0, 1, 0],
                           [2, 0, 0],
                           [3, 0, 2],
                           [2, 1, 1],
                           [0, 0, 2]]) 
    
    is_safe_state, safe_sequence = is_safe(Jumlah_proses, avail, max_need, allocation)
    
    if is_safe_state:
        print("Sistem dalam urutan aman:", safe_sequence)
    else:
        print("Tidak ada urutan aman, sistem dalam keadaan deadlock.")
    
\end{lstlisting}

\begin{figure}[H]
    \centering
    \includegraphics[width=1\linewidth]{asset/421.png}
    \caption{Output dan Penjelasannya}
\end{figure}

\paragraph{
    \hspace*{1cm} Dengan menggunakan metode \textit{Banker's} untuk algoritma \textit{Safety}, kita bisa memastikan bahwa sistem aman jika menjalankan proses-proses tertentu. Di sini, kita memeriksa setiap proses dan memastikan bahwa alokasi sumber daya untuk setiap proses dapat dilakukan dengan aman sehingga tidak akan menyebabkan \textit{deadlock}. Pada output, sistem akan menampilkan status apakah sistem aman atau tidak, dan menampilkan urutan proses yang aman.
}


\subsection{Assignment 3: Multithreading and Amdahl's Law}
\hspace*{1cm} Simulasikan skenario \textit{multithreading} untuk menghitung jumlah kuadrat dari rentang angka yang diberikan dan menjumlahkannya, dan terapkan \textit{Amdahl law} untuk menghitung prediksi kecepatan seiring penambahan \textit{thread} untuk mengetahui penggunaan jumlah \textit{thread} yang optimal.
\newline
\newline

\begin{center}
    \underline{JAWABAN}
\end{center}

\begin{lstlisting}[language=Python]

    import threading
    import time
    
    def compute_squares(start, end, results, index, execution_times):
        start_time = time.time()
        print(f"Thread {index} mulai menghitung dari {start} hingga {end}...")
        result = 0
        for number in range(start, end):
            time.sleep(0.0001) 
            result += number * number
        results[index] = result
        end_time = time.time()
        execution_times[index] = end_time - start_time
        print(f"Thread {index} selesai menghitung dalam {execution_times[index]:.4f} detik.")
    
    total_range = 10000
    num_threads = 4  
    step = total_range // num_threads
    
    
    results = [0] * num_threads
    execution_times = [0] * num_threads
    
    threads = []
    
    start_time = time.time()
    for i in range(num_threads):
        start = i * step
        end = (i + 1) * step if i != num_threads - 1 else total_range
        thread = threading.Thread(target=compute_squares, args=(start, end, results, i, execution_times))
        threads.append(thread)
        thread.start()
    
    for thread in threads:
        thread.join()
    end_time = time.time()
    
    total_execution_time = end_time - start_time
    total_result = sum(results)
    
    def amdahl_law(P, N):
        return 1 / ((1 - P) + (P / N))
    
    
    sequential_fraction = 0.1  
    P = 1 - sequential_fraction
    
    print("Semua thread selesai.")
    print(f"Hasil perhitungan: {total_result}")
    print(f"Waktu eksekusi total: {total_execution_time:.4f} detik")
    
    print("Speedup teoretis menurut Hukum Amdahl:")
    for N in range(1, num_threads + 1):  
        speedup = amdahl_law(P, N)
        print(f"Speedup teoretis dengan {N} thread: {speedup:.2f}x")
    
\end{lstlisting}

\begin{figure}[H]
    \centering
    \includegraphics[width=1\linewidth]{asset/31.png}
    \caption{Eksekusi dengan 4 threads (kiri) dan eksekusi dengan 8 threads (kanan)}
\end{figure}

\paragraph{
    \hspace*{1cm} Pada algoritma di atas, \textit{multithreading} berfungsi untuk membagi proses berdasarkan jumlah \textit{thread} yang digunakan, sehingga kita bisa membuat waktu eksekusi menjadi lebih singkat. Dengan menggunakan algoritma \textit{Amdahl's Law}, kita bisa melihat rasio peningkatan kecepatan seiring meningkatnya jumlah \textit{thread} yang digunakan. Dapat dilihat pada perbedaan output di atas, bahwa dengan meningkatkan jumlah \textit{thread} dari 4 ke 8, waktu eksekusi juga meningkat.
}



\subsection{Assignment 4: Simple Command-Line Interface (CLI) for User Interface Management}
Students were tasked with creating a simple **CLI** for user interface management. The CLI should support basic commands such as file manipulation (creating, listing, and deleting files), process management, and system status reporting.

\subsection{Assignment 5: File System Access}
In this assignment, students implemented file system access routines, including:
\begin{itemize}
    \item File creation and deletion
    \item Reading from and writing to files
    \item Navigating directories and managing file permissions
\end{itemize}
\subsubsection{Group 9}
\subsubsection*{Soal 1: File creation and deletion}
Di sebuah perusahaan teknologi, Imam adalah seorang \textit{programmer} yang diminta untuk membuat sebuah program yang dapat \textbf{membuat dan menghapus file}. Manager meminta agar Imam membuat sebuah file bernama \texttt{report.txt} yang berisi teks ``Laporan Bulanan'' dan kemudian menghapus file tersebut setelah manager membaca laporannya.

\textbf{Tugas:} Buatlah sebuah program dalam Python yang dapat:
\begin{enumerate}
    \item Membuat file \texttt{report.txt} dan menuliskan teks ``Laporan Bulanan''.
    \item Menghapus file tersebut setelah selesai.
\end{enumerate}

\textbf{Jawaban:}
\begin{python}
import os

def create_file(filename, content):
    with open(filename, 'w') as file:
        file.write(content)
    print(f"File '{filename}' telah dibuat.")

def delete_file(filename):
    if os.path.exists(filename):
        os.remove(filename)
        print(f"File '{filename}' telah dihapus.")
    else:
        print(f"File '{filename}' tidak ditemukan.")

# Implementasi
create_file("report.txt", "Laporan Bulanan")
delete_file("report.txt")
\end{python}

\textbf{Output:}
\begin{python}
File 'report.txt' telah dibuat.
File 'report.txt' telah dihapus.
\end{python}

\subsubsection{Group 9}
\subsubsection*{Soal 2: Reading from and writing to files}
Joy adalah pengembang di departemen dokumentasi perusahaan. Direktur meminta Joy untuk menambahkan catatan ke dalam file \texttt{log.txt} tanpa menghapus isinya yang lama. Setiap hari, Joy harus menambah catatan baru di akhir file tersebut.

\textbf{Tugas:} Buatlah program Python yang dapat:
\begin{enumerate}
    \item Membaca isi dari file \texttt{log.txt}.
    \item Menambahkan catatan baru berisi teks ``Catatan baru ditambahkan'' tanpa menghapus isi sebelumnya.
\end{enumerate}

\textbf{Jawaban:}
\begin{python}
def read_file(filename):
    try:
        with open(filename, 'r') as file:
            content = file.read()
            print(f"Isi file '{filename}':\n{content}")
    except FileNotFoundError:
        print(f"File '{filename}' tidak ditemukan.")

def append_to_file(filename, new_content):
    with open(filename, 'a') as file:
        file.write(new_content + "\n")
    print(f"Teks '{new_content}' telah ditambahkan ke file '{filename}'.")

# Implementasi
append_to_file("log.txt", "Catatan baru ditambahkan")
read_file("log.txt")
\end{python}

\textbf{Output:}
\begin{python}
Teks 'Catatan baru ditambahkan' telah ditambahkan ke file 'log.txt'.
Isi file 'log.txt':
Catatan baru ditambahkan
\end{python}

\subsubsection{Group 9}
\subsubsection*{Soal 3: Mencari Folder yang Hilang}
Di sebuah perusahaan, ada banyak folder yang sering berpindah-pindah. Kepala Divisi Anda meminta Anda untuk menemukan sebuah folder bernama \texttt{important\_docs}. Jika folder tersebut tidak ada, Anda harus membuatnya. Setelah itu, Anda perlu masuk ke dalam folder tersebut dan memeriksa apa saja isi di dalamnya.

\textbf{Tugas:} Buatlah program Python yang dapat:
\begin{enumerate}
    \item Membuat folder \texttt{important\_docs} jika belum ada.
    \item Berpindah ke folder tersebut.
    \item Menampilkan daftar file atau folder yang ada di dalamnya.
\end{enumerate}

\textbf{Jawaban:}
\begin{python}
import os

def list_directory(path='.'):
    print(f"Isi direktori '{path}':")
    for item in os.listdir(path):
        print(item)

def create_and_change_directory(new_dir):
    if not os.path.exists(new_dir):
        os.makedirs(new_dir)
        print(f"Direktori '{new_dir}' telah dibuat.")
    os.chdir(new_dir)
    print(f"Berpindah ke direktori '{new_dir}'.")

# Implementasi
create_and_change_directory("important_docs")
list_directory()
\end{python}

\textbf{Output:}
\begin{python}
Direktori 'important_docs' telah dibuat.
Berpindah ke direktori 'important_docs'.
Isi direktori 'important_docs':
\end{python}

\subsubsection{Group 9}
\subsubsection*{Soal 4: Managing file permissions}
Di perusahaan Anda, ada file rahasia bernama \texttt{secret.txt} yang hanya boleh dibaca oleh anggota tim keamanan. Anda diminta untuk membuat program yang dapat mengubah hak akses file tersebut menjadi hanya bisa dibaca \textit{(read-only)}. Setelah beberapa waktu, file tersebut perlu dimodifikasi lagi, jadi hak aksesnya harus diubah kembali menjadi bisa ditulis. \textit{(writeable)}.

\textbf{Tugas:} Buatlah program Python yang dapat:
\begin{enumerate}
    \item Mengubah hak akses file \texttt{secret.txt} menjadi hanya bisa dibaca.
    \item Mengembalikan hak akses file \texttt{secret.txt} menjadi bisa ditulis.
\end{enumerate}

\textbf{Jawaban:}
\begin{python}
import os

def set_read_only(filename):
    os.chmod(filename, 0o444)  # Read-only
    print(f"File '{filename}' sekarang hanya bisa dibaca.")

def set_writeable(filename):
    os.chmod(filename, 0o666)  # Read-write
    print(f"File '{filename}' sekarang bisa dibaca dan ditulis.")

# Implementasi
set_read_only("secret.txt")
set_writeable("secret.txt")
\end{python}

\textbf{Output:}
\begin{python}
File 'secret.txt' sekarang hanya bisa dibaca.
File 'secret.txt' sekarang bisa dibaca dan ditulis.
\end{python}

\subsubsection{Group 9}
\subsubsection*{Soal 5: Navigating directories}
Anda sedang bekerja di dalam sebuah sub-folder untuk menyelesaikan beberapa tugas. Setelah menyelesaikan pekerjaan di sub-folder tersebut, Anda perlu kembali ke \textbf{folder induk (parent directory)} untuk mengatur file-file lain.

\textbf{Tugas:} Buatlah program Python yang dapat:
\begin{enumerate}
    \item Berpindah dari folder saat ini ke \textbf{parent directory}.
    \item Menampilkan isi dari \textit{parent directory} setelah berpindah.
\end{enumerate}

\textbf{Jawaban:}
\begin{python}
import os

def list_directory(path='.'):
    print(f"Isi direktori '{path}':")
    for item in os.listdir(path):
        print(item)

def change_to_parent_directory():
    parent_dir = os.path.abspath(os.path.join(os.getcwd(), os.pardir))
    os.chdir(parent_dir)
    print(f"Berpindah ke direktori parent: {parent_dir}")
    list_directory()

# Implementasi
change_to_parent_directory()
\end{python}

\textbf{Output:}
\begin{python}
Berpindah ke direktori parent: /path/to/parent/folder
Isi direktori '/path/to/parent/folder':
file1.txt
file2.txt
subfolder
\end{python}

\section{Conclusion}
The first half of the course introduced core operating system concepts, including process management, scheduling, multithreading, and file system access. These topics provided a foundation for more advanced topics to be covered in the second half of the course.

\begin{thebibliography}{9}
    \bibitem{gfg2024}
    TestingXperts. (2022, May 10). Performance testing metrics: A detailed guide for businesses. Digital Assurance. Retrieved October 8, 2024, from https://www.testingxperts.com
    \bibitem{gfg2024}
    GeeksforGeeks. (n.d.). Computer organization: Performance of computer. GeeksforGeeks. Retrieved October 1, 2024, from https://www.geeksforgeeks.org/computer-organization-performance-of-computer/
    \bibitem{gfg2024}
    GeeksforGeeks. (2024, August 20). Program for round robin scheduling for the same arrival time. GeeksforGeeks. Retrieved October 8, 2024, from https://www.geeksforgeeks.org/program-for-round-robin-scheduling-for-the-same-arrival-time/
\end{thebibliography}

\end{document}