\documentclass[12pt]{article}
\usepackage{amsmath}
\usepackage{graphicx}
\usepackage{hyperref}
\usepackage{listings}
\usepackage{color}
\usepackage{pythonhighlight}

\title{Operating System Course Report - First Half of the Semester}
\author{A class}
\date{\today}

\begin{document}

\maketitle
\newpage

\tableofcontents
\newpage

\section{Introduction}
This report summarizes the topics covered during the first half of the Operating System course. It includes theoretical concepts, practical implementations, and assignments. The course focuses on the fundamentals of operating systems, including system architecture, process management, CPU scheduling, and deadlock handling.

\section{Course Overview}
\subsection{Objectives}
The main objectives of this course are:
\begin{itemize}
    \item To understand the basic components and architecture of a computer system.
    \item To learn process management, scheduling, and inter-process communication.
    \item To explore file systems, input/output management, and virtualization.
    \item To study the prevention and handling of deadlocks in operating systems.
\end{itemize}

\subsection{Course Structure}
The course is divided into two halves. This report focuses on the first half, which covers:
\begin{itemize}
    \item Basic Concepts and Components of Computer Systems
    \item System Performance and Metrics
    \item System Architecture of Computer Systems
    \item Process Description and Control
    \item Scheduling Algorithms
    \item Process Creation and Termination
    \item Introduction to Threads
    \item File Systems
    \item Input and Output Management
    \item Deadlock Introduction and Prevention
    \item User Interface Management
    \item Virtualization in Operating Systems
\end{itemize}

\section{Topics Covered}

\subsection{Basic Concepts and Components of Computer Systems}
This section explains the fundamental components that make up a computer system, including the CPU, memory, storage, and input/output devices.

\subsection{System Performance and Metrics}
This section introduces various system performance metrics used to measure the efficiency of a computer system, including throughput, response time, and utilization.

\subsection{System Architecture of Computer Systems}
Describes the architecture of modern computer systems, focusing on the interaction between hardware and the operating system.

\subsection{Process Description and Control}
Processes are a central concept in operating systems. This section covers:
\begin{itemize}
    \item Process states and state transitions
    \item Process control block (PCB)
    \item Context switching
\end{itemize}
\subsubsection{\textit{Context switching}}
 \textit{Context switching} Context switching adalah mekanisme yang memungkinkan sistem operasi untuk \textbf{berpindah dari satu proses ke proses lain.} Setiap kali terjadi perpindahan, OS perlu menyimpan state dari proses yang sedang berjalan (misalnya nilai register, program counter, stack pointer, dll.) dan memuat state dari proses yang akan dijalankan. Proses penyimpanan dan pemuatan ini memerlukan waktu dan sumber daya, sehingga context switching dianggap sebagai overhead dalam sistem multitasking. 

 Dalam sistem preemptive multitasking, di mana proses-proses dihentikan secara paksa oleh OS untuk memberikan giliran pada proses lain, context switching menjadi sangat penting untuk menjaga ilusi bahwa semua proses berjalan secara bersamaan. Proses ini terjadi ketika CPU berpindah dari satu proses ke proses lain, atau dari proses ke kernel mode, misalnya saat menangani interrupts. 
 (\textit{Context switching}).

 \subsubsection{\textit{Inter-Process Communication}}
 \textit{Inter-Process Communication} IPC adalah serangkaian mekanisme yang disediakan oleh OS untuk \textbf{memungkinkan proses-proses yang berjalan secara bersamaan untuk saling berkomunikasi }atau bekerja sama. Karena proses-proses ini biasanya terisolasi satu sama lain, IPC memungkinkan pertukaran data dan sinkronisasi kerja mereka. 

Beberapa metode umum IPC termasuk:
\begin{itemize}
    \item \textbf{Message Passing: }Proses berkomunikasi dengan mengirim dan menerima pesan melalui saluran yang disediakan OS. 
    \item \textbf{Shared Memory:} Proses berbagi area memori bersama untuk bertukar data, memerlukan sinkronisasi agar tidak terjadi konflik. 
    \item \textbf{Signals:} Mekanisme sederhana untuk mengirim notifikasi antar-proses tentang terjadinya suatu kejadian. 
\end{itemize}
 (\textit{Inter-Process Communication}).

\subsection{Scheduling Algorithms}
This section covers:
\begin{itemize}
    \item First-Come, First-Served (FCFS)
    \item Shortest Job Next (SJN)
    \item Round Robin (RR)
\end{itemize}
It explains how these algorithms are used to allocate CPU time to processes.

\subsection{Process Creation and Termination}
Details how processes are created and terminated by the operating system, including:
\begin{itemize}
    \item Process spawning
    \item Process termination conditions
\end{itemize}

\subsection{Introduction to Threads}
This section introduces the concept of threads and their relation to processes, covering:
\begin{itemize}
    \item Single-threaded vs. multi-threaded processes
    \item Benefits of multithreading
\end{itemize}

\begin{figure}[h]
    \centering
    \includegraphics[width=0.5\textwidth]{/Users/khawaritzmi/Unhas/os_report_mid2024/a_class/asset/example.png}  % Sesuaikan nama file dan ukurannya
    \caption{Ini adalah gambar contoh dari multithreading.}
    \label{fig:contoh_gambar}
\end{figure}

Seperti yang terlihat pada Gambar \ref{fig:contoh_gambar}, inilah cara menambahkan gambar dengan keterangan.

\subsection{File Systems}
File systems provide a way for the operating system to store, retrieve, and manage data. This section explains:
\begin{itemize}
    \item File system structure
    \item File access methods
    \item Directory management
\end{itemize}

\subsection{Input and Output Management}
Input and output management is key for handling the interaction between the system and external devices. This section includes:
\begin{itemize}
    \item Device drivers
    \item I/O scheduling
\end{itemize}

\subsection{Deadlock Introduction and Prevention}
Explores the concept of deadlocks and methods for preventing them:
\begin{itemize}
    \item Deadlock conditions
    \item Deadlock prevention techniques
\end{itemize}

\subsection{User Interface Management}
This section discusses the role of the operating system in managing the user interface. Topics covered include:
\begin{itemize}
    \item Graphical User Interface (GUI)
    \item Command-Line Interface (CLI)
    \item Interaction between the user and the operating system
\end{itemize}

\subsection{Virtualization in Operating Systems}
Virtualization allows multiple operating systems to run concurrently on a single physical machine. This section explores:
\begin{itemize}
    \item Concept of virtualization
    \item Hypervisors and their types
    \item Benefits of virtualization in modern computing
\end{itemize}

\section{Assignments and Practical Work}
\subsection{Assignment 1: Process Scheduling}
Students were tasked with implementing various process scheduling algorithms (e.g., FCFS, SJN, and RR) and comparing their performance under different conditions.
\subsubsection{Group 1}
\begin{python}
    class Process:
    def __init__(self, pid, arrival_time, burst_time):
        self.pid = pid
        self.arrival_time = arrival_time
        self.burst_time = burst_time
        self.completion_time = 0
        self.turnaround_time = 0
        self.waiting_time = 0
\end{python}

\begin{table}[htbp] % Optional: For floating position
    \centering
    \begin{tabular}{|c|c|c|} % Defines number of columns and alignment (c = center, l = left, r = right). '|' creates vertical lines.
    \hline
    Header 1 & Header 2 & Header 3 \\ % Column headers
    \hline
    Row 1, Column 1 & Row 1, Column 2 & Row 1, Column 3 \\ % First row of data
    \hline
    Row 2, Column 1 & Row 2, Column 2 & Row 2, Column 3 \\ % Second row of data
    \hline
    \end{tabular}
    \caption{Your table caption} % Optional: For adding a caption
    \label{tab:your_label} % Optional: For cross-referencing the table
\end{table}
\subsection{Assignment 2: Deadlock Handling}
\subsubsection{Group 4}
\noindent
\textbf{Pertanyaan:}
Terdapat sebuah sistem yang memiliki 4 proses (P1, P2, P3, P4) dan 2 sumber daya (R1, R2). Proses-proses ini meminta dan melepaskan sumber daya sebagai berikut:
\begin{itemize}
    \item P1 memegang R1 dan membutuhkan R2
    \item P2 memegang R2 dan membutuhkan R1
    \item P3 memegang R2 dan membutuhkan R1
    \item P4 memegang R1 dan membutuhkan R2
\end{itemize}
Apakah sistem ini mengalami \textit{deadlock}? Jika ya, sebutkan proses mana saja yang terlibat dalam deadlock!

\noindent
\textbf{Jawaban:}
\[
P_1 \rightarrow R_1 \rightarrow P_2 \rightarrow R_2 \rightarrow P_1
\]
\begin{python}
# Representasi proses dan resource
processes = ['P1', 'P2', 'P3', 'P4']
resources = {'P1': ['R1'], 'P2': ['R2'], 'P3': ['R2'], 'P4': ['R1']}
needs = {'P1': ['R2'], 'P2': ['R1'], 'P3': ['R1'], 'P4': ['R2']}

# Deteksi circular wait untuk deadlock
def check_deadlock(processes, resources, needs):
    for process in processes:
        holding = resources[process][0]
        needed = needs[process][0]
        if holding != needed:
            print(f"{process} mengalami deadlock (memegang {holding}, menunggu {needed})")

check_deadlock(processes, resources, needs)
\end{python}

\noindent
\textbf{Penjelasan:}
Untuk menganalisis apakah sistem mengalami \textit{deadlock}, kita bisa menggunakan kondisi Coffman untuk deadlock:
\begin{itemize}
    \item \textit{Mutual Exclusion}: Setiap sumber daya hanya dapat dimiliki oleh satu proses pada satu waktu.
    \item \textit{Hold and Wait} : Proses yang memegang satu sumber daya dapat menunggu sumber daya lain.
    \item \textit{No Preemption} : Sumber daya yang sudah dialokasikan tidak dapat diambil paksa dari proses pemiliknya.
    \item \textit{Circular Wait} : Harus ada rangkaian siklus proses, di mana setiap proses menunggu sumber daya yang dimiliki oleh proses lain dalam siklus.
\end{itemize}

\subsection{Assignment 3: Multithreading and Amdahl's Law}
This assignment involved designing a multithreading scenario to solve a computationally intensive problem. Students then applied **Amdahl's Law** to calculate the theoretical speedup of the program as the number of threads increased.

\subsection{Assignment 4: Simple Command-Line Interface (CLI) for User Interface Management}
Students were tasked with creating a simple **CLI** for user interface management. The CLI should support basic commands such as file manipulation (creating, listing, and deleting files), process management, and system status reporting.

\subsection{Assignment 5: File System Access}
In this assignment, students implemented file system access routines, including:
\begin{itemize}
    \item File creation and deletion
    \item Reading from and writing to files
    \item Navigating directories and managing file permissions
\end{itemize}

\section{Conclusion}
The first half of the course introduced core operating system concepts, including process management, scheduling, multithreading, and file system access. These topics provided a foundation for more advanced topics to be covered in the second half of the course.

\end{document}