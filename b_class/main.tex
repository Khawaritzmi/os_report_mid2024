\documentclass[12pt]{article}
\usepackage{amsmath}
\usepackage{graphicx}
\usepackage{hyperref}
\usepackage{listings}
\usepackage{color}
\usepackage{pythonhighlight}

\title{Operating System Course Report - First Half of the Semester}
\author{B class}
\date{\today}

\begin{document}

\maketitle
\newpage

\tableofcontents
\newpage

\section{Introduction}
This report summarizes the topics covered during the first half of the
Operating System course. It includes theoretical concepts, practical
implementations, and assignments. The course focuses on the fundamentals of
operating systems, including system architecture, process management, CPU
scheduling, and deadlock handling.

\section{Course Overview}
\subsection{Objectives}
The main objectives of this course are:
\begin{itemize}
    \item To understand the basic components and architecture of a computer system.
    \item To learn process management, scheduling, and inter-process communication.
    \item To explore file systems, input/output management, and virtualization.
    \item To study the prevention and handling of deadlocks in operating systems.
\end{itemize}

\subsection{Course Structure}
The course is divided into two halves. This report focuses on the first half,
which covers:
\begin{itemize}
    \item Basic Concepts and Components of Computer Systems
    \item System Performance and Metrics
    \item System Architecture of Computer Systems
    \item Process Description and Control
    \item Scheduling Algorithms
    \item Process Creation and Termination
    \item Introduction to Threads
    \item File Systems
    \item Input and Output Management
    \item Deadlock Introduction and Prevention
    \item User Interface Management
    \item Virtualization in Operating Systems
\end{itemize}

\section{Topics Covered}

\subsection{Basic Concepts and Components of Computer Systems}
This section explains the fundamental components that make up a computer
system, including the CPU, memory, storage, and input/output devices.

\subsection{System Performance and Metrics}
This section introduces various system performance metrics used to measure the
efficiency of a computer system, including throughput, response time, and
utilization.

\subsection{System Architecture of Computer Systems}
Describes the architecture of modern computer systems, focusing on the
interaction between hardware and the operating system.

\subsection{Process Description and Control}
Processes are a central concept in operating systems. This section covers:
\begin{itemize}
    \item Process states and state transitions
    \item Process control block (PCB)
    \item Context switching
\end{itemize}

\subsection{Scheduling Algorithms}
This section covers:
\begin{itemize}
    \item First-Come, First-Served (FCFS)
    \item Shortest Job Next (SJN)
    \item Round Robin (RR)
\end{itemize}
It explains how these algorithms are used to allocate CPU time to processes.

\subsection{Process Creation and Termination}
Details how processes are created and terminated by the operating system,
including:
\begin{itemize}
    \item Process spawning
    \item Process termination conditions
\end{itemize}

\subsection{Introduction to Threads}
This section introduces the concept of threads and their relation to processes,
covering:
\begin{itemize}
    \item Single-threaded vs. multi-threaded processes
    \item Benefits of multithreading
\end{itemize}

% \begin{figure}[h]
%     \centering
%     \includegraphics[width=0.5\textwidth]{./b_class/asset/example.png}  % Sesuaikan nama file dan ukurannya
%     \caption{Ini adalah gambar contoh dari multithreading.}
%     \label{fig:contoh_gambar}
% \end{figure}

Seperti yang terlihat pada Gambar \ref{fig:contoh_gambar}, inilah cara
menambahkan gambar dengan keterangan.

\subsection{File Systems}
File systems provide a way for the operating system to store, retrieve, and
manage data. This section explains:
\begin{itemize}
    \item File system structure
    \item File access methods
    \item Directory management
\end{itemize}

\subsection{Input and Output Management}
Input and output management is key for handling the interaction between the
system and external devices. This section includes:
\begin{itemize}
    \item Device drivers
    \item I/O scheduling
\end{itemize}

\subsection{Deadlock Introduction and Prevention}
Explores the concept of deadlocks and methods for preventing them:
\begin{itemize}
    \item Deadlock conditions
\end{itemize}
\subsubsection{\textit{Deadlock prevention}}

\subsubsection{\textit{Deadlock prevention methods}}

\hspace{1cm} Pencegahan terhadap \textit{deadlock} adalah teknik yang digunakan untuk mencegah sistem memasuki keadaan \textit{deadlock}.
Berbeda dengan \textit{deadlock} avoidance yang di mana mengatasi masalah setelah \textit{deadlock} terjadi, pencegahan ada agar memperkecil
kemungkinan terjadinya \textit{deadlock}. Pencegahan \textit{deadlock} dapat dilakukan dengan cara menghindari keadaan-keadaan yang dapat menyebabkan \textit{deadlock}.
Beberapa metode pencegahan \textit{deadlock} adalah:

\begin{enumerate}
    \item \textit{Resources Ordering}

          \hspace{1cm}\textit{Resources Ordering} adalah pengurutan sumber daya di mana setiap sumber daya
          dalam sistem diberi nomor atau urutan unik. Proses harus meminta sumber daya
          dalam urutan yang konsisten dengan nomor tersebut. Artinya, jika suatu proses
          sudah memegang satu sumber daya dengan nomor lebih rendah, maka ia hanya boleh
          meminta sumber daya dengan nomor lebih tinggi.

          \hspace{1cm}Salah satu contoh kasus pencegahan \textit{deadlock} dengan metode \textit{resources
          ordering} yaitu jika ada dua sumber daya, R1 dan R2, dan proses P1 memegang R1 dan ingin
          meminta R2, maka metode ini memastikan bahwa proses P1 tidak dapat memegang R2
          tanpa mengikuti urutan yang telah ditentukan.

    \item \textit{Resource allocation denial}

          \hspace{1cm}Metode ini dikenal juga sebagai \textit{Banker's
          Algorithm} atau \textit{safety algorithm}, di mana sistem memeriksa apakah pemberian
          sumber daya tertentu akan menyebabkan kondisi \textit{unsafe} atau potensi \textit{deadlock}.
          Jika pemberian sumber daya dianggap tidak aman, maka permintaan sumber daya
          tersebut akan ditolak.

          \hspace{1cm}Metode \textit{resources allocation denial} adalah ketika
          proses P1 meminta sumber daya R1. Sebelum mengalokasikan sumber daya, sistem mengevaluasi
          apakah proses lain bisa menyelesaikan eksekusi dengan sumber daya yang tersisa.
          Jika tidak, permintaan P1 ditolak sementara waktu dan akan dilanjutkan ketika semua sumber daya yang dibutuhkan sudah lengkap.

    \item \textit{Timeouts}

          \hspace{1cm}Dalam metode ini, sistem menggunakan batas waktu (\textit{timeout})
          untuk menunggu sumber daya. Jika proses menunggu terlalu lama, ia akan dipaksa
          gagal atau \textit{restart}. \textit{Timeout} dapat digunakan untuk mencegah proses berlama-lama
          dalam kondisi menunggu, sehingga menghindari \textit{deadlock}.

          \hspace{1cm}Proses P1 sedang menunggu sumber daya R1 selama 5 menit. Jika setelah batas waktu ini R1 belum tersedia, proses P1 akan dibatalkan atau dijadwalkan ulang untuk mencoba lagi.

    \item \textit{Avoidance of hold-and-wait conditions}

          \hspace{1cm}Teknik \textit{avoidance of hold-and-wait
          conditions} adalah teknik pencegahan \textit{deadlock} dengan kondisi saling tunggu sumber daya antara suatu proses.
          Dalam teknik ini, proses tidak
          diperbolehkan menahan sumber daya sambil menunggu sumber daya lain. Proses
          harus meminta semua sumber daya yang dibutuhkan di awal (sebelum memulai
          eksekusi) atau melepaskan sumber daya yang sedang dipegang sebelum meminta
          sumber daya tambahan.

          \hspace{1cm}Dalam penerapannya, Proses P1 harus meminta sumber daya
          R1 dan R2 sekaligus. Jika R2 tidak tersedia, maka proses tidak diperbolehkan
          memegang R1 dan harus melepaskannya sampai kedua sumber daya tersedia.
          Setiap proses yang belum memiliki sumber daya yang cukup untuk berjalan maka
          akan terus menunggu hingga sumber daya yang dibutuhkan terpenuhi semua.

    \item \textit{Resource preemption}

          \hspace{1cm}Teknik ini memungkinkan sistem untuk merebut atau
          mengambil kembali sumber daya dari proses yang sedang berjalan jika diperlukan
          oleh proses lain yang lebih prioritas atau jika diperlukan untuk mencegah
          \textit{deadlock}\textit{deadlock}.

          \hspace{1cm}Dalam kasus seperti proses P1 harus meminta sumber daya
          R1 dan R2 sekaligus. Jika R2 tidak tersedia, proses tidak diperbolehkan
          memegang R1 dan harus melepaskannya sampai kedua sumber daya tersedia.
\end{enumerate}

\subsection{User Interface Management}
This section discusses the role of the operating system in managing the user
interface. Topics covered include:
\begin{itemize}
    \item Graphical User Interface (GUI)
    \item Command-Line Interface (CLI)
    \item Interaction between the user and the operating system
\end{itemize}

\subsection{Virtualization in Operating Systems}
Virtualization allows multiple operating systems to run concurrently on a
single physical machine. This section explores:
\begin{itemize}
    \item Concept of virtualization
    \item Hypervisors and their types
    \item Benefits of virtualization in modern computing
\end{itemize}

\section{Assignments and Practical Work}
\subsection{Assignment 1: Process Scheduling}
Students were tasked with implementing various process scheduling algorithms
(e.g., FCFS, SJN, and RR) and comparing their performance under different
conditions.
\subsubsection{Group 1}
\begin{python}
    class Process:
    def __init__(self, pid, arrival_time, burst_time):
    self.pid = pid
    self.arrival_time = arrival_time
    self.burst_time = burst_time
    self.completion_time = 0
    self.turnaround_time = 0
    self.waiting_time = 0
\end{python}

\begin{table}[htbp] % Optional: For floating position
    \centering
    \begin{tabular}{|c|c|c|} % Defines number of columns and alignment (c = center, l = left, r = right). '|' creates vertical lines.
        \hline
        Header 1        & Header 2        & Header 3        \\ % Column headers
        \hline
        Row 1, Column 1 & Row 1, Column 2 & Row 1, Column 3 \\ % First row of data
        \hline
        Row 2, Column 1 & Row 2, Column 2 & Row 2, Column 3 \\ % Second row of data
        \hline
    \end{tabular}
    \caption{Your table caption} % Optional: For adding a caption
    \label{tab:your_label} % Optional: For cross-referencing the table
\end{table}

\subsection{Assignment 2: Deadlock Handling}
In this assignment, students were asked to simulate different deadlock
scenarios and explore various prevention methods.

\subsection{Assignment 3: Multithreading and Amdahl's Law}
This assignment involved designing a multithreading scenario to solve a
computationally intensive problem. Students then applied **Amdahl's Law** to
calculate the theoretical speedup of the program as the number of threads
increased.

\subsection{Assignment 4: Simple Command-Line Interface (CLI) for User Interface Management}
Students were tasked with creating a simple **CLI** for user interface
management. The CLI should support basic commands such as file manipulation
(creating, listing, and deleting files), process management, and system status
reporting.

\subsection{Assignment 5: File System Access}
In this assignment, students implemented file system access routines,
including:
\begin{itemize}
    \item File creation and deletion
    \item Reading from and writing to files
    \item Navigating directories and managing file permissions
\end{itemize}

\section{Conclusion}
The first half of the course introduced core operating system concepts,
including process management, scheduling, multithreading, and file system
access. These topics provided a foundation for more advanced topics to be
covered in the second half of the course.

\end{document}