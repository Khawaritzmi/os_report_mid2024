\documentclass[12pt]{article}
\usepackage{amsmath}
\usepackage{graphicx}
\usepackage{hyperref}
\usepackage{listings}
\usepackage{color}
\usepackage{pythonhighlight}

\title{Operating System Course Report - First Half of the Semester}
\author{B class}
\date{\today}

\begin{document}

\maketitle
\newpage

\tableofcontents
\newpage

\section{Introduction}
This report summarizes the topics covered during the first half of the Operating System course. It includes theoretical concepts, practical implementations, and assignments. The course focuses on the fundamentals of operating systems, including system architecture, process management, CPU scheduling, and deadlock handling.

\section{Course Overview}
\subsection{Objectives}
The main objectives of this course are:
\begin{itemize}
    \item To understand the basic components and architecture of a computer system.
    \item To learn process management, scheduling, and inter-process communication.
    \item To explore file systems, input/output management, and virtualization.
    \item To study the prevention and handling of deadlocks in operating systems.
\end{itemize}

\subsection{Course Structure}
The course is divided into two halves. This report focuses on the first half, which covers:
\begin{itemize}
    \item Basic Concepts and Components of Computer Systems
    \item System Performance and Metrics
    \item System Architecture of Computer Systems
    \item Process Description and Control
    \item Scheduling Algorithms
    \item Process Creation and Termination
    \item Introduction to Threads
    \item File Systems
    \item Input and Output Management
    \item Deadlock Introduction and Prevention
    \item User Interface Management
    \item Virtualization in Operating Systems
\end{itemize}

\section{Topics Covered}

\subsection{Basic Concepts and Components of Computer Systems}
This section explains the fundamental components that make up a computer system, including the CPU, memory, storage, and input/output devices.

\subsection{System Performance and Metrics}
This section introduces various system performance metrics used to measure the efficiency of a computer system, including throughput, response time, and utilization.

\subsubsection{Definisi Performa Sistem}

Performa sistem komputer mengacu pada kemampuan suatu sistem untuk menjalankan tugas atau fungsinya secara efisien dan efektif. Terdapat beberapa aspek penting yang menjadi indikator performa sistem komputer, yaitu:

\textbf{1. Kecepatan Pemrosesan:} \\
Kecepatan pemrosesan adalah waktu yang dibutuhkan oleh sistem untuk mengeksekusi tugas tertentu. Hal ini biasanya diukur dalam satuan waktu (detik atau milidetik) yang diperlukan untuk menyelesaikan sebuah tugas. Rumus untuk kecepatan pemrosesan dalam hal frekuensi adalah:

\[
T = \frac{1}{F}
\]

dengan:
\begin{itemize}
    \item $T$ adalah waktu untuk satu siklus pemrosesan (detik per siklus).
    \item $F$ adalah frekuensi prosesor (siklus per detik atau Hz).
\end{itemize}

\textbf{2. Penggunaan Sumber Daya:} \\
Penggunaan sumber daya mengacu pada seberapa efisien sistem menggunakan CPU, memori, dan perangkat keras lainnya untuk menyelesaikan tugas. Pengelolaan sumber daya yang baik akan meminimalkan penggunaan yang berlebihan dan meningkatkan performa secara keseluruhan.

\textbf{3. Responsivitas:} \\
Responsivitas adalah kecepatan sistem dalam merespons input dari pengguna atau proses lainnya. Responsivitas yang tinggi sangat penting dalam aplikasi \textit{real-time} dan interaktif. Pengukuran responsivitas sering kali berkaitan dengan waktu tunda (\textit{latency}).

\textbf{4. Throughput:} \\
Throughput mengukur jumlah tugas atau pekerjaan yang dapat diselesaikan oleh sistem dalam waktu tertentu. Throughput yang lebih tinggi menunjukkan bahwa sistem dapat menangani lebih banyak pekerjaan dalam satu waktu. Rumus throughput dapat dinyatakan sebagai:

\[
Throughput = \frac{\text{Jumlah Pekerjaan}}{\text{Waktu}}
\]

\textbf{Faktor-Faktor yang Mempengaruhi Performa Sistem:} \\
Faktor-faktor yang memengaruhi performa sistem dapat dibagi menjadi dua kategori utama, yaitu faktor internal dan faktor eksternal.

\begin{itemize}
    \item \textbf{Faktor Internal:} Faktor internal melibatkan komponen yang memengaruhi performa sistem dari dalam, termasuk:
    \begin{itemize}
        \item \textbf{Kecepatan CPU:} Prosesor dengan kecepatan clock yang lebih tinggi dapat memproses instruksi lebih cepat. Kecepatan CPU diukur dalam Hertz (Hz), dan hubungan antara waktu eksekusi dan clock rate adalah:
        \[
        Waktu Eksekusi = \frac{\text{Jumlah Instruksi} \times \text{CPI}}{\text{Clock Rate}}
        \]
        \item \textbf{Arsitektur Memori:} Efisiensi memori sangat memengaruhi kinerja, terutama dalam hal kecepatan akses ke data. Struktur \textit{cache}, bandwidth memori, dan ukuran RAM berperan penting.
        \item \textbf{Algoritma dan Struktur Data:} Algoritma yang dioptimalkan dan pemilihan struktur data yang tepat akan meningkatkan kecepatan pemrosesan serta penggunaan memori secara efisien.
    \end{itemize}
    
    \item \textbf{Faktor Eksternal:} Faktor eksternal mencakup elemen di luar komponen internal yang juga dapat memengaruhi performa sistem, seperti:
    \begin{itemize}
        \item \textbf{Beban Kerja:} Semakin besar beban kerja yang ditangani oleh sistem, semakin tinggi kebutuhan akan sumber daya seperti CPU dan memori.
        \item \textbf{Kondisi Lingkungan:} Faktor-faktor seperti suhu, kelembapan, dan pasokan listrik yang stabil juga memengaruhi performa perangkat keras.
        \item \textbf{Jaringan:} Dalam sistem yang terhubung ke jaringan, kecepatan dan stabilitas jaringan sangat penting dalam menjaga throughput dan latensi yang rendah.
    \end{itemize}
\end{itemize}


\subsection{System Architecture of Computer Systems}
Describes the architecture of modern computer systems, focusing on the interaction between hardware and the operating system.

\subsection{Process Description and Control}
Processes are a central concept in operating systems. This section covers:
\begin{itemize}
    \item Process states and state transitions
    \item Process control block (PCB)
    \item Context switching
\end{itemize}

\subsection{Scheduling Algorithms}
This section covers:
\begin{itemize}
    \item First-Come, First-Served (FCFS)
    \item Shortest Job Next (SJN)
    \item Round Robin (RR)
\end{itemize}
It explains how these algorithms are used to allocate CPU time to processes.

\subsection{Process Creation and Termination}
Details how processes are created and terminated by the operating system, including:
\begin{itemize}
    \item Process spawning
    \item Process termination conditions
\end{itemize}

\subsection{Introduction to Threads}
This section introduces the concept of threads and their relation to processes, covering:
\begin{itemize}
    \item Single-threaded vs. multi-threaded processes
    \item Benefits of multithreading
\end{itemize}

\begin{figure}[h]
    \centering
    \includegraphics[width=0.5\textwidth]{/Users/khawaritzmi/Unhas/os_report_mid2024/b_class/asset/example.png}  % Sesuaikan nama file dan ukurannya
    \caption{Ini adalah gambar contoh dari multithreading.}
    \label{fig:contoh_gambar}
\end{figure}

Seperti yang terlihat pada Gambar \ref{fig:contoh_gambar}, inilah cara menambahkan gambar dengan keterangan.

\subsection{File Systems}
File systems provide a way for the operating system to store, retrieve, and manage data. This section explains:
\begin{itemize}
    \item File system structure
    \item File access methods
    \item Directory management
\end{itemize}

\subsection{Input and Output Management}
Input and output management is key for handling the interaction between the system and external devices. This section includes:
\begin{itemize}
    \item Device drivers
    \item I/O scheduling
\end{itemize}

\subsection{Deadlock Introduction and Prevention}
Explores the concept of deadlocks and methods for preventing them:
\begin{itemize}
    \item Deadlock conditions
    \item Deadlock prevention techniques
\end{itemize}

\subsection{User Interface Management}
This section discusses the role of the operating system in managing the user interface. Topics covered include:
\begin{itemize}
    \item Graphical User Interface (GUI)
    \item Command-Line Interface (CLI)
    \item Interaction between the user and the operating system
\end{itemize}

\subsection{Virtualization in Operating Systems}
Virtualization allows multiple operating systems to run concurrently on a single physical machine. This section explores:
\begin{itemize}
    \item Concept of virtualization
    \item Hypervisors and their types
    \item Benefits of virtualization in modern computing
\end{itemize}

\section{Assignments and Practical Work}
\subsection{Assignment 1: Process Scheduling}
Students were tasked with implementing various process scheduling algorithms (e.g., FCFS, SJN, and RR) and comparing their performance under different conditions.
\subsubsection{Group 1}
\begin{python}
    class Process:
    def __init__(self, pid, arrival_time, burst_time):
        self.pid = pid
        self.arrival_time = arrival_time
        self.burst_time = burst_time
        self.completion_time = 0
        self.turnaround_time = 0
        self.waiting_time = 0
\end{python}

\begin{table}[htbp] % Optional: For floating position
    \centering
    \begin{tabular}{|c|c|c|} % Defines number of columns and alignment (c = center, l = left, r = right). '|' creates vertical lines.
    \hline
    Header 1 & Header 2 & Header 3 \\ % Column headers
    \hline
    Row 1, Column 1 & Row 1, Column 2 & Row 1, Column 3 \\ % First row of data
    \hline
    Row 2, Column 1 & Row 2, Column 2 & Row 2, Column 3 \\ % Second row of data
    \hline
    \end{tabular}
    \caption{Your table caption} % Optional: For adding a caption
    \label{tab:your_label} % Optional: For cross-referencing the table
\end{table}

\subsection{Assignment 2: Deadlock Handling}
In this assignment, students were asked to simulate different deadlock scenarios and explore various prevention methods.

\subsection{Assignment 3: Multithreading and Amdahl's Law}
This assignment involved designing a multithreading scenario to solve a computationally intensive problem. Students then applied **Amdahl's Law** to calculate the theoretical speedup of the program as the number of threads increased.

\subsection{Assignment 4: Simple Command-Line Interface (CLI) for User Interface Management}
Students were tasked with creating a simple **CLI** for user interface management. The CLI should support basic commands such as file manipulation (creating, listing, and deleting files), process management, and system status reporting.

\subsubsection{Soal}

Jelaskan bagaimana Anda mengimplementasikan fungsi untuk membaca dan menulis file pada sistem berkas. Sebutkan langkah-langkah yang diperlukan untuk memastikan akses file dilakukan dengan aman dan efisien, serta bagaimana Anda menangani situasi ketika file tidak tersedia atau terjadi kesalahan dalam proses akses.

\subsubsection{Jawaban}

Untuk mengimplementasikan fungsi membaca dan menulis file pada sistem berkas, saya menggunakan Python dengan dua fungsi utama: \texttt{baca\_file} dan \texttt{tulis\_file}. Berikut adalah kode lengkapnya:

\begin{verbatim}
def baca_file(jalur_berkas):
    try:
        with open(jalur_berkas, 'r') as berkas:
            isi = berkas.read()
            return isi
    except FileNotFoundError:
        print(f"Berkas {jalur_berkas} tidak ditemukan.")
    except IOError:
        print("Terjadi kesalahan saat membaca berkas.")

def tulis_file(jalur_berkas, konten):
    try:
        with open(jalur_berkas, 'w') as berkas:
            berkas.write(konten)
            print(f"Berhasil menulis ke berkas {jalur_berkas}.")
    except IOError:
        print("Terjadi kesalahan saat menulis ke berkas.")

# Contoh penggunaan
jalur_berkas = 'contoh.txt'
konten = 'Ini adalah contoh isi berkas.'

# Menulis ke berkas
tulis_file(jalur_berkas, konten)

# Membaca dari berkas
isi_berkas = baca_file(jalur_berkas)
if isi_berkas:
    print("Isi berkas:", isi_berkas)
\end{verbatim}

Kode di atas terdiri dari dua fungsi utama: \texttt{baca\_file} dan \texttt{tulis\_file}. Fungsi \texttt{baca\_file} digunakan untuk membaca isi dari sebuah berkas yang ditentukan dalam variabel \texttt{jalur\_berkas}. Dengan membuka berkas dalam mode baca menggunakan \texttt{with open}, berkas akan secara otomatis ditutup setelah operasi selesai. Fungsi ini juga menangkap kesalahan jika berkas tidak ditemukan atau jika ada masalah I/O lainnya.

Di sisi lain, \texttt{tulis\_file} bertujuan untuk menulis teks yang terdapat dalam variabel \texttt{konten} ke berkas yang juga ditentukan dalam \texttt{jalur\_berkas}. Fungsi ini membuka berkas dalam mode tulis, yang secara otomatis akan membuat berkas baru jika berkas yang dituju belum ada. Seperti fungsi sebelumnya, \texttt{tulis\_file} juga menangkap dan menampilkan pesan jika terjadi kesalahan saat menulis.

Terakhir, terdapat contoh penggunaan fungsi-fungsi tersebut, di mana \texttt{tulis\_file} digunakan untuk menyimpan teks ke dalam berkas, kemudian \texttt{baca\_file} digunakan untuk membaca kembali isi berkas tersebut dan menampilkannya jika berhasil. Kode ini memberikan pengelolaan berkas yang aman serta responsif terhadap kesalahan.


\subsection{Assignment 5: File System Access}
In this assignment, students implemented file system access routines, including:
\begin{itemize}
    \item File creation and deletion
    \item Reading from and writing to files
    \item Navigating directories and managing file permissions
\end{itemize}

\section{Conclusion}
The first half of the course introduced core operating system concepts, including process management, scheduling, multithreading, and file system access. These topics provided a foundation for more advanced topics to be covered in the second half of the course.

\end{document}