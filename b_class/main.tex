\documentclass[12pt]{article}
\usepackage{amsmath}
\usepackage{graphicx}
\usepackage{hyperref}
\usepackage{listings}
\usepackage{color}
\usepackage{float}

\lstset{
    language=Python,                      % Set bahasa ke Python
    basicstyle=\ttfamily\footnotesize,    % Ukuran dan font kode
    keywordstyle=\color{blue},            % Warna keyword
    stringstyle=\color{red},              % Warna string
    commentstyle=\color{green},           % Warna komentar
    numbers=left,                         % Menampilkan nomor baris
    numberstyle=\tiny,                    % Ukuran nomor baris
    stepnumber=1,                         % Setiap baris diberi nomor
    breaklines=true,                      % Pemenggalan baris otomatis
    frame=single,                         % Bingkai di sekitar kode
    tabsize=4                             % Ukuran tab
}

\title{Operating System Course Report - First Half of the Semester}
\author{B class}
\date{\today}

\begin{document}

\maketitle
\newpage

\tableofcontents
\newpage

\section{Introduction}
This report summarizes the topics covered during the first half of the Operating System course. It includes theoretical concepts, practical implementations, and assignments. The course focuses on the fundamentals of operating systems, including system architecture, process management, CPU scheduling, and deadlock handling.

\section{Course Overview}
\subsection{Objectives}
The main objectives of this course are:
\begin{itemize}
    \item To understand the basic components and architecture of a computer system.
    \item To learn process management, scheduling, and inter-process communication.
    \item To explore file systems, input/output management, and virtualization.
    \item To study the prevention and handling of deadlocks in operating systems.
\end{itemize}

\subsection{Course Structure}
The course is divided into two halves. This report focuses on the first half, which covers:
\begin{itemize}
    \item Basic Concepts and Components of Computer Systems
    \item System Performance and Metrics
    \item System Architecture of Computer Systems
    \item Process Description and Control
    \item Scheduling Algorithms
    \item Process Creation and Termination
    \item Introduction to Threads
    \item File Systems
    \item Input and Output Management
    \item Deadlock Introduction and Prevention
    \item User Interface Management
    \item Virtualization in Operating Systems
\end{itemize}

\section{Topics Covered}

\subsection{Basic Concepts and Components of Computer Systems}
    \subsubsection{Definisi dan Konsep}
    \subsubsection{Komponen-Komponen Sistem Komputer}
    Sistem komputer merupakan kesatuan yang terbentuk dari beberapa komponen utama yang saling berhubungan untuk menjalankan fungsi komputasi. Ketiga komponen utama tersebut adalah \textit{hardware, software}, dan \textit{brainware}. Kombinasi ketiganya memungkinkan komputer untuk melakukan berbagai tugas, mulai dari pengolahan data, penyimpanan informasi, hingga eksekusi perintah.
    \begin{itemize}
        \item Hardware
        \par
        \textit{Hardware} adalah perangkat keras mengacu pada perangkat fisik yang dapat dilihat dan disentuh, seperti \textit{Central Processing Unit }(CPU), \textit{monitor, keyboard,} dan komponen internal lainnya.\textit{ Hardware} berperan sebagai fondasi yang menyediakan pl\textit{a}tform untuk berjalannya \textit{software}. Perangkat keras pada sistem komputer terbagi atas tiga bagian, yaitu:
        \begin{itemize}
            \item Input Unit
            \par 
            \textit{Input unit} merupakan bagian dari perangkat keras yang berfungsi sebagai alat untuk memasukkan data dan lainnya ke dalam komputer. Perangkat input unit antara lain, \textit{keyword, mouse, camera digital}, dan sebagainya.
            \item \textit{Processing Unit}
            \par
            \textit{Processing unit} biasanya disebut \textit{Central Processing Unit} (CPU) yang merupakan otak atau jantung dari komputer karena mengendalikan semua fungsi yang terjadi dalam sistem komputer. Perangkat utamanya berupa prosesor dan chipset yang biasanya terdapa pada \textit{motherboard}. CPU memiliki tiga komponen utama, yaitu:
            \begin{itemize}
                \item \textit{Arimatic dan Logical Unit} (ALU)
                \par
                ALU bertugas untuk melakukan perhitungan yang bersifat arimatik dan melakukan keputusan dari operasi logika dan manipulasi b\textit{i}t sesuai dengan instruksi program.
                \item \textit{Control Unit}
                \par
                \textit{Control unit} berfungsi sebagai pengatur dan pengendali semua peralatan yang ada pada sistem komputer dan mengatur kapan alat input menerima data dan kapan menampilkan output di monitor
                \item \textit{Main Memory}
                \par
               \textit{ Main memory} merupakan tempat atau media yang digunakan untuk menyimpan data yang akan atau sedang dikelola oleh sistem komputer. \textit{Main memory } terbagi menjadi dua, yaitu:
                \begin{itemize}
                    \item ROM (\textit{Read Only Memory})
                    \par
                    ROM merupakan memori permanen yang terdapat di dalam sistem komputer yang sudah disusun atau sudah dibuat dipabrik pembuatan dan biasanya tidak bisa diubah oleh \textit{user} komputer.
                    \item RAM  (\textit{Random Access Memory})
                    \par
                    RAM adalah memori yang memuat semua data yang dimasukkan melalui alat input pada setiap aplikasi yang akan dimasukkan terlebih dahulu ke dalam \textit{main memory} dan RAM ini biasanya bersifat sementara karena apabila komputer dimatikan maka semua data tersebut akan hilang.
                \end{itemize}
            \end{itemize}
        \item Output Unit 
        \par
        \textit{Output unit} merupakan perangkat keras yang berfungsi untuk menyajikan output dari proses yang sedang dikerjakan pada komputer. Bentuk peralatan output ini antara lain, monitor, \textit{printer, projector}, speaker, dan lain-lain.
        \end{itemize}
        
        \item Software 
        \par
        \textit{Software} adalah perangkat lunak atau program yang berjalan di atas \textit{hardware}. \textit{Software} bertugas mengendalikan dan mengatur \textit{hardware} agar dapat bekerja sesuai dengan fungsi yang diinginkan, seperti sistem operasi, aplikasi, dan program utilitas. Perangkat lunak menjembatani interaksi\textit{ user} dengan komputer yang hanya memahami bahasa komputer. Secara umum, perangkat lunak terbagi menjadi 2, yaitu:
        \begin{itemize}
            \item Operation System Software
            \par
            \textit{Operation System Software} merupakan perangkat lunak yang memiliki fungsi untuk mengonfigurasikan komputer agar menerima berbagai perintah dasar yang diberikan untuk dimasukkan. Contohnya, \textit{Linux, MS-DOS,} dan lain-lain
            \item Perangkat Lunak Aplikasi
            \par
            Perangkat Lunak Aplikasi merupakan program yang siap pakai yang digunakan di suatu bidang tertentu, seperti aplikasi pada bisnis dan perkantoran yang biasanya menggunakan\textit{ Microsoft Office, Koffice}, dan lain-lain
        \end{itemize}
        \item Brainware 
        \par
       \textit{ Brainware } mencakup pengguna atau orang yang berinteraksi dengan komputer, baik itu sebagai pengguna akhir, pengembang, maupun administrator. \textit{Brainware }berperan dalam merancang, mengoperasikan, dan memanfaatkan teknologi komputer secara optimal.
    \end{itemize}
    \begin{bibliography}
    Agus Tri Haryanto,M.CS., Taufiq Lilo Adi Sucipto, M.T. 2014. Sistem Komputer.Jakarta. Politeknik Negeri Media Kreatif.
\end{bibliography}

   

\subsection{System Performance and Metrics}
\subsubsection{\textit{Performance Metrics} Pada \textit{CPU}}
\hspace*{1cm} Jika kita membahas kinerja matriks pada komputer, maka kita tidak bisa lepas dari CPU. CPU atau \textit{Central Processing Unit} adalah komponen utama dari komputer yang bertanggung jawab mengeksekusi instruksi-instruksi yang diberikan kepada komputer. Kinerja suatu sistem sangat dipengaruhi oleh cara CPU mengeksekusi instruksi. Untuk memaksimalkan kinerja suatu sistem, kita tentu perlu meminimalkan waktu eksekusi, karena kinerja berbanding terbalik dengan waktu eksekusi.
\newline
\newline
\hspace*{1cm} Untuk menentukan waktu eksekusi CPU untuk suatu program, Anda dapat mencari tahu jumlah total \textit{\textit{clock} cycle} yang dibutuhkan program dan mengalikannya dengan waktu \textit{\textit{clock} cycle}. Setiap program terdiri dari sejumlah instruksi, dan setiap instruksi membutuhkan sejumlah \textit{\textit{clock} cycle} untuk dieksekusi. Jika Anda mengetahui jumlah total \textit{\textit{clock} cycle} per program serta mengetahui waktu \textit{\textit{clock} cycle} untuk setiap \textit{\textit{clock} cycle}, maka waktu eksekusi CPU dapat dihitung sebagai hasil perkalian jumlah total \textit{\textit{clock} cycle} CPU per program dengan waktu \textit{\textit{clock} cycle}. Karena waktu \textit{\textit{clock} cycle} dan \textit{\textit{clock} rate} saling terkait, hal ini juga dapat ditulis sebagai jumlah \textit{\textit{clock} cycle} CPU untuk suatu program dibagi dengan \textit{\textit{clock} rate}.

\begin{figure}
    \centering
    \includegraphics[width=\linewidth]{asset/image1.png}
    \caption{Rumus waktu CPU}
\end{figure}


\hspace*{1cm} Karena waktu eksekusi CPU merupakan hasil dari kedua faktor ini, kita dapat meningkatkan kinerja dengan mengurangi durasi waktu \textit{clock cycle} atau jumlah \textit{clock cycle} yang diperlukan untuk suatu program. Kecepatan \textit{clock} pada dasarnya bergantung pada organisasi CPU tertentu.

\subsubsection{Mengapa \textit{Performance Metrics} Penting}
\hspace*{1cm} Mengukur kinerja menggunakan data yang akurat sangat penting dalam sebuah sistem karena memberikan gambaran nyata tentang performa yang sedang berlangsung. Dengan data yang tepat, manajemen dapat membuat keputusan yang lebih terinformasi, sehingga memperkecil risiko kesalahan dalam strategi. Selain itu, pengukuran kinerja juga membantu mengidentifikasi area yang memerlukan perbaikan, sehingga organisasi dapat berfokus pada peningkatan yang signifikan. Memantau perkembangan dan kemajuan kinerja secara berkala meningkatkan akuntabilitas, baik di tingkat individu maupun tim, sehingga setiap anggota lebih bertanggung jawab atas hasil kerjanya. Pada akhirnya, penggunaan \textit{performance metrics} yang tepat dapat mendorong peningkatan berkelanjutan dengan memberikan dasar yang kuat untuk evaluasi dan inovasi secara sistematis dalam mencapai tujuan jangka panjang.

\subsubsection{Fungsi \textit{Performance Metrics} pada Sistem}
\hspace*{1cm} \textit{Performance metrics} berperan penting dalam berbagai aspek sistem operasi. Berikut adalah beberapa area di mana \textit{performance metrics} digunakan:
\begin{itemize}
    \item \textbf{\textit{Benchmark}} \newline
    \hspace*{1cm} \textit{Benchmark} biasanya menggunakan berbagai \textit{performance metrics} untuk mengukur dan melaporkan hasil kinerjanya. Misalnya, sebuah \textit{benchmark} untuk prosesor mungkin mengukur metrik seperti kecepatan \textit{clock}, jumlah instruksi per detik (\textit{IPS}), atau waktu komputasi untuk tugas tertentu.
    \item \textbf{\textit{Bandwidth}} \newline  
    \hspace*{1cm} Mengukur jumlah data yang dapat ditransmisikan melalui jaringan dalam jangka waktu tertentu. \textit{Bandwidth} yang terbatas membatasi kemampuan sistem untuk menangani banyak permintaan atau mentransfer data dalam jumlah besar, yang pada akhirnya memperlambat waktu muat atau respons. Pada sistem yang sangat bergantung pada transfer data, seperti aplikasi berbasis web atau layanan \textit{cloud}, \textit{bandwidth} adalah faktor penting yang memengaruhi performa keseluruhan.
    \item \textbf{\textit{Error Rate}} \newline  
    \hspace*{1cm} Menghitung persentase permintaan yang gagal diproses atau menghasilkan kesalahan. Metrik ini penting untuk menilai stabilitas dan keandalan sistem. Tingkat kesalahan yang tinggi menunjukkan masalah pada sistem, seperti bug dalam perangkat lunak, kesalahan konfigurasi, atau keterbatasan sumber daya. Tingkat kesalahan yang tinggi dapat mengganggu pengguna dan menurunkan kepercayaan mereka terhadap sistem.
    \item \textbf{Pengaturan Kualitas Layanan (\textit{QoS})} \newline  
    \hspace*{1cm} Dalam lingkungan yang membutuhkan kualitas layanan tertentu, \textit{performance metrics} digunakan untuk memastikan bahwa aplikasi dan layanan memenuhi \textit{SLA} (\textit{Service Level Agreement}). Metrik seperti latensi, \textit{throughput}, dan waktu respons digunakan untuk mengatur dan memantau \textit{QoS}.
    \item \textbf{Optimasi dan Tuning} \newline  
    \hspace*{1cm} \textit{Performance metrics} digunakan untuk mengidentifikasi peluang optimasi dan tuning. Misalnya, dengan
\end{itemize}

\hspace*{1cm} \textit{Performance metrics} berperan penting dalam berbagai aspek pengelolaan dan operasional sistem operasi. Dari monitoring dan manajemen sumber daya hingga optimasi, diagnosis, dan keamanan, metrik ini memberikan wawasan yang diperlukan untuk menjaga sistem tetap berjalan dengan efisien, aman, dan sesuai dengan kebutuhan kinerja yang diharapkan.

\subsection{System Architecture of Computer Systems}
Describes the architecture of modern computer systems, focusing on the interaction between hardware and the operating system.

\subsection{Process Description and Control}
Processes are a central concept in operating systems. This section covers:
\begin{itemize}
    \item Process states and state transitions
    \item Process control block (PCB)
    \item Context switching
\end{itemize}

\subsection{Scheduling Algorithms}
This section covers:
\begin{itemize}
    \item First-Come, First-Served (FCFS)
    \item Shortest Job Next (SJN)
    \item Round Robin (RR)
\end{itemize}
It explains how these algorithms are used to allocate CPU time to processes.

\subsection{Process Creation and Termination}
Details how processes are created and terminated by the operating system, including:
\begin{itemize}
    \item Process spawning
    \item Process termination conditions
\end{itemize}

\subsection{Introduction to Threads}
This section introduces the concept of threads and their relation to processes, covering:
\begin{itemize}
    \item Single-threaded vs. multi-threaded processes
    \item Benefits of multithreading
\end{itemize}

\subsection{File Systems}
File systems provide a way for the operating system to store, retrieve, and manage data. This section explains:
\begin{itemize}
    \item File system structure
    \item File access methods
    \item Directory management
\end{itemize}

\subsection{Input and Output Management}
Input and output management is key for handling the interaction between the system and external devices. This section includes:
\begin{itemize}
    \item Device drivers
    \item I/O scheduling
\end{itemize}

\subsection{Deadlock Introduction and Prevention}
Explores the concept of deadlocks and methods for preventing them:
\begin{itemize}
    \item Deadlock conditions
    \item Deadlock prevention techniques
\end{itemize}

\subsection{User Interface Management}
This section discusses the role of the operating system in managing the user interface. Topics covered include:
\begin{itemize}
    \item Graphical User Interface (GUI)
    \item Command-Line Interface (CLI)
    \item Interaction between the user and the operating system
\end{itemize}

\subsection{Virtualization in Operating Systems}
Virtualization allows multiple operating systems to run concurrently on a single physical machine. This section explores:
\begin{itemize}
    \item Concept of virtualization
    \item Hypervisors and their types
    \item Benefits of virtualization in modern computing
\end{itemize}


\section{Assignments and Practical Work}
\subsection{Assignment 1: Process Scheduling}
\hspace*{1cm} Implementasikan algoritma penjadwalan \textit{Shortest Job First} (SJF) dan \textit{Round Robin} (RR), lalu bandingkan rata-rata waktu tunggu (\textit{average waiting time}) dan rata-rata waktu penyelesaian (\textit{average turnaround time}) dari kedua algoritma tersebut. Manakah yang lebih baik dalam menangani proses? 
\newline
\newline

\begin{center}
    \underline{JAWABAN}
\end{center}

\textit{Short Job First (SJF) algorithms}
\begin{lstlisting}[language=Python]
    def sjf_non_preemptive(Jumlah_proses):
        Jumlah_proses.sort(key=lambda x: x[1])
    
        waktu_tunggu = 0
        total_waktu_tunggu = 0
    
        for pid, burst_time in Jumlah_proses:
            total_waktu_tunggu += waktu_tunggu
            waktu_tunggu += burst_time
    
        n = len(Jumlah_proses)
        avg_waktu_tunggu = total_waktu_tunggu / n
    
        print(total_waktu_tunggu)
        print(f"\nAverage Waiting Time: {avg_waktu_tunggu}")    
    
\end{lstlisting}

\textit{Round Robin (RR) algorithms}
\begin{lstlisting}[language=Python]
    def round_robin(jumlah_proses, quantum):
        n = len(jumlah_proses)
        remaining_time = [bt for _, bt in jumlah_proses]
        waktu_tunggu = [0] * n
        time = 0
    
        while True:
            done = True
            for i in range(n):
                if remaining_time[i] > 0:
                    done = False
                    if remaining_time[i] > quantum:
                        time += quantum
                        remaining_time[i] -= quantum
                    else:
                        time += remaining_time[i]
                        waktu_tunggu[i] = time - jumlah_proses[i][1]
                        remaining_time[i] = 0
            if done:
                break
    
        avg_waktu_tunggu = sum(waktu_tunggu) / n
        print(f"\nAverage Waiting Time: {avg_waktu_tunggu:.2f}")
    
\end{lstlisting}

\begin{figure}[H]
    \centering
    \includegraphics[width=1\linewidth]{asset/41.png}
    \caption{Output}
\end{figure}

\begin{figure}[H]
    \centering
    \includegraphics[width=1\linewidth]{asset/411.png}
    \caption{Perbandingan SJF dan RR(quantum = 2)}
\end{figure}

\paragraph{
    \hspace*{1cm} Dari hasil perbandingan rata-rata waktu tunggu pada kedua algoritma, dapat dilihat bahwa rata-rata waktu tunggu SJF lebih kecil dibandingkan dengan \textit{Round Robin}. Salah satu alasannya adalah karena sifat RR yang memberikan setiap proses sejumlah waktu yang sama (\textit{quantum}) dan memaksa semua proses menunggu giliran mereka secara bergantian. 
}

\paragraph{
    \hspace*{1cm} Kita dapat mengurangi kelemahan \textit{Round Robin} dengan memilih \textit{quantum} yang lebih besar. Pada visualisasi di atas terlihat bahwa waktu tunggu RR dapat sama dengan SJF. Namun, tetap saja, SJF sering kali lebih baik dalam kasus dengan banyak proses yang memiliki \textit{burst time} yang bervariasi.
}

\begin{figure}[H]
    \centering
    \includegraphics[width=1\linewidth]{asset/412.png}
    \caption{Perbandingan SJF dan RR(quantum = 8)}
\end{figure}


\subsection{Assignment 2: Deadlock Handling}
\hspace*{1cm} Simulasikan bagaimana \textit{algoritma safety} bekerja untuk mendeteksi \textit{deadlock}.
\newline
\newline

\begin{center}
    \underline{JAWABAN}
\end{center}

\begin{lstlisting}[language=Python]

    import numpy as np
    
    def is_safe(Jumlah_proses, avail, max_need, allocation):
        P, R = len(Jumlah_proses), len(avail) 
        need = max_need - allocation 
        finish = [False] * P 
        safe_sequence = []  
        work = avail.copy() 
    
        for _ in range(P):
            for p in range(P):
                if not finish[p] and all(need[p] <= work): 
                    work += allocation[p]
                    finish[p] = True
                    safe_sequence.append(p)
                    break
            else:
                print("Sistem tidak dalam keadaan aman.")
                return False, []
    
        print("Sistem dalam keadaan aman.")
        return True, safe_sequence
    
    
    Jumlah_proses = [0, 1, 2, 3, 4]
    avail = np.array([3, 3, 2]) 
    max_need = np.array([[7, 5, 3],
                         [3, 2, 2],
                         [9, 0, 2],
                         [2, 2, 2],
                         [4, 3, 3]]) 
    allocation = np.array([[0, 1, 0],
                           [2, 0, 0],
                           [3, 0, 2],
                           [2, 1, 1],
                           [0, 0, 2]]) 
    
    is_safe_state, safe_sequence = is_safe(Jumlah_proses, avail, max_need, allocation)
    
    if is_safe_state:
        print("Sistem dalam urutan aman:", safe_sequence)
    else:
        print("Tidak ada urutan aman, sistem dalam keadaan deadlock.")
    
\end{lstlisting}

\begin{figure}[H]
    \centering
    \includegraphics[width=1\linewidth]{asset/421.png}
    \caption{Output dan Penjelasannya}
\end{figure}

\paragraph{
    \hspace*{1cm} Dengan menggunakan metode \textit{Banker's} untuk algoritma \textit{Safety}, kita bisa memastikan bahwa sistem aman jika menjalankan proses-proses tertentu. Di sini, kita memeriksa setiap proses dan memastikan bahwa alokasi sumber daya untuk setiap proses dapat dilakukan dengan aman sehingga tidak akan menyebabkan \textit{deadlock}. Pada output, sistem akan menampilkan status apakah sistem aman atau tidak, dan menampilkan urutan proses yang aman.
}


\subsection{Assignment 3: Multithreading and Amdahl's Law}
\hspace*{1cm} Simulasikan skenario \textit{multithreading} untuk menghitung jumlah kuadrat dari rentang angka yang diberikan dan menjumlahkannya, dan terapkan \textit{Amdahl law} untuk menghitung prediksi kecepatan seiring penambahan \textit{thread} untuk mengetahui penggunaan jumlah \textit{thread} yang optimal.
\newline
\newline

\begin{center}
    \underline{JAWABAN}
\end{center}

\begin{lstlisting}[language=Python]

    import threading
    import time
    
    def compute_squares(start, end, results, index, execution_times):
        start_time = time.time()
        print(f"Thread {index} mulai menghitung dari {start} hingga {end}...")
        result = 0
        for number in range(start, end):
            time.sleep(0.0001) 
            result += number * number
        results[index] = result
        end_time = time.time()
        execution_times[index] = end_time - start_time
        print(f"Thread {index} selesai menghitung dalam {execution_times[index]:.4f} detik.")
    
    total_range = 10000
    num_threads = 4  
    step = total_range // num_threads
    
    
    results = [0] * num_threads
    execution_times = [0] * num_threads
    
    threads = []
    
    start_time = time.time()
    for i in range(num_threads):
        start = i * step
        end = (i + 1) * step if i != num_threads - 1 else total_range
        thread = threading.Thread(target=compute_squares, args=(start, end, results, i, execution_times))
        threads.append(thread)
        thread.start()
    
    for thread in threads:
        thread.join()
    end_time = time.time()
    
    total_execution_time = end_time - start_time
    total_result = sum(results)
    
    def amdahl_law(P, N):
        return 1 / ((1 - P) + (P / N))
    
    
    sequential_fraction = 0.1  
    P = 1 - sequential_fraction
    
    print("Semua thread selesai.")
    print(f"Hasil perhitungan: {total_result}")
    print(f"Waktu eksekusi total: {total_execution_time:.4f} detik")
    
    print("Speedup teoretis menurut Hukum Amdahl:")
    for N in range(1, num_threads + 1):  
        speedup = amdahl_law(P, N)
        print(f"Speedup teoretis dengan {N} thread: {speedup:.2f}x")
    
\end{lstlisting}

\begin{figure}[H]
    \centering
    \includegraphics[width=1\linewidth]{asset/31.png}
    \caption{Eksekusi dengan 4 threads (kiri) dan eksekusi dengan 8 threads (kanan)}
\end{figure}

\paragraph{
    \hspace*{1cm} Pada algoritma di atas, \textit{multithreading} berfungsi untuk membagi proses berdasarkan jumlah \textit{thread} yang digunakan, sehingga kita bisa membuat waktu eksekusi menjadi lebih singkat. Dengan menggunakan algoritma \textit{Amdahl's Law}, kita bisa melihat rasio peningkatan kecepatan seiring meningkatnya jumlah \textit{thread} yang digunakan. Dapat dilihat pada perbedaan output di atas, bahwa dengan meningkatkan jumlah \textit{thread} dari 4 ke 8, waktu eksekusi juga meningkat.
}



\subsection{Assignment 4: Simple Command-Line Interface (CLI) for User Interface Management}
Students were tasked with creating a simple **CLI** for user interface management. The CLI should support basic commands such as file manipulation (creating, listing, and deleting files), process management, and system status reporting.

\subsection{Assignment 5: File System Access}
In this assignment, students implemented file system access routines, including:
\begin{itemize}
    \item File creation and deletion
    \item Reading from and writing to files
    \item Navigating directories and managing file permissions
\end{itemize}
\subsubsection{Group 1}
Soal:
\par Buatlah program Python yang melakukan operasi sederhana pada file teks dan folder dengan fitur-fitur berikut:
\begin{itemize}
    \item Membuat folder baru : program meminta pengguna memasukkan nama folder yang ingin di buat.
    \item Menulis dan Membaca File Teks: program meminta pengguna memasukkan nama file teks (.txt) dan menampilkan isinya jika sudah ada. jika file belum ada, program meminta pengguna untuk memasukkan teks yang ingin ditulis ke dalam file tersebut.
    \item Menghitung Konten dalam File Teks: program meminta pengguna memasukkan nama file dan program harus membaca file dan menghitung jumlah baris, jumlah kata, jumlah karakter yang ada di dalam file.
    \item Menghapus File Teks: program meminta pengguna memasukkan file yang ingin dihapus.
    \item Menampilkan Daftar File Teks dalam Folder: program meminta pengguna memasukkan nama folder dan program menampilkan daftar semua file teks (.txt) yang ada di dalam folder tersebut.
\end{itemize}
Jawaban:
\lstset{ 
		language=Python, % Bahasa pemrograman
		backgroundcolor=\color{white}, % Warna latar belakang
		commentstyle=\color{gray}, % Warna komentar
		keywordstyle=\color{blue}, % Warna string
		basicstyle=\ttfamily\footnotesize, % Gaya dasar teks
		breaklines=true,  % Otomatis break baris
		frame=single, % Garis kotak di sekitar kode
		showstringspaces=false, % Hilangkan spasi dalam string
	}

\begin{python}
     import os

# Fungsi 1: Membuat Folder Baru

def create_folder():
    folder_name = input("Masukkan nama folder yang ingin dibuat: ")
    if os.path.exists(folder_name):
        print(f"Folder '{folder_name}' sudah ada.")
    else:
        os.makedirs(folder_name)
        print(f"Folder '{folder_name}' berhasil dibuat.")

# Fungsi 2: Menulis dan Membaca File Teks
def write_read_file():
    file_name = input("Masukkan nama file teks (misal: catatan.txt): ")
    if os.path.isfile(file_name):
        with open(file_name, 'r') as file:
            content = file.read()
            print(f"Isi file '{file_name}':\n{content}")
    else:
        content = input("File tidak ditemukan. Masukkan teks yang ingin disimpan ke dalam file: ")
        with open(file_name, 'w') as file:
            file.write(content)
            print(f"Isi file '{file_name}' berhasil disimpan.")

# Fungsi 3: Menghitung Jumlah Baris, Kata, dan Karakter dalam File Teks
def count_file_content():
    file_name = input("Masukkan nama file teks: ")
    if not os.path.isfile(file_name):
        print(f"File '{file_name}' tidak ditemukan.")
        return

    with open(file_name, 'r') as file:
        content = file.readlines()
        num_lines = len(content)
        num_words = sum(len(line.split()) for line in content)
        num_characters = sum(len(line) for line in content)

    print(f"Jumlah baris: {num_lines}")
    print(f"Jumlah kata: {num_words}")
    print(f"Jumlah karakter: {num_characters}")

# Fungsi 4: Menghapus File Teks
def delete_file():
    file_name = input("Masukkan nama file yang ingin dihapus: ")
    if os.path.isfile(file_name):
        os.remove(file_name)
        print(f"File '{file_name}' berhasil dihapus.")
    else:
        print(f"File '{file_name}' tidak ditemukan.")

# Fungsi 5: Menampilkan Daftar Semua File Teks dalam Folder
def list_text_files():
    folder_name = input("Masukkan nama folder: ")
    if not os.path.exists(folder_name):
        print(f"Folder '{folder_name}' tidak ditemukan.")
        return

    text_files = [f for f in os.listdir(folder_name) if f.endswith('.txt')]
    
    if text_files:
        print(f"Daftar file teks dalam folder '{folder_name}':")
        for file in text_files:
            print(file)
    else:
        print(f"Tidak ada file teks dalam folder '{folder_name}'.")

# Menu Interaktif
def main_menu():
    while True:
        print("\nPilih opsi yang ingin dijalankan:")
        print("1. Membuat folder baru")
        print("2. Menulis dan membaca file teks")
        print("3. Menghitung jumlah baris, kata, dan karakter dalam file teks")
        print("4. Menghapus file teks")
        print("5. Menampilkan daftar semua file teks dalam folder")
        print("0. Keluar")
        
        choice = input("Pilihan Anda: ")
        if choice == '1':
            create_folder()
        elif choice == '2':
            write_read_file()
        elif choice == '3':
            count_file_content()
        elif choice == '4':
            delete_file()
        elif choice == '5':
            list_text_files()
        elif choice == '0':
            print("Keluar dari program.")
            break
        else:
            print("Pilihan tidak valid! Silakan coba lagi.")

if __name__ == "__main__":
    main_menu()

\end{python}


\section{Conclusion}
The first half of the course introduced core operating system concepts, including process management, scheduling, multithreading, and file system access. These topics provided a foundation for more advanced topics to be covered in the second half of the course.

\begin{thebibliography}{9}
    \bibitem{gfg2024}
    TestingXperts. (2022, May 10). Performance testing metrics: A detailed guide for businesses. Digital Assurance. Retrieved October 8, 2024, from https://www.testingxperts.com
    \bibitem{gfg2024}
    GeeksforGeeks. (n.d.). Computer organization: Performance of computer. GeeksforGeeks. Retrieved October 1, 2024, from https://www.geeksforgeeks.org/computer-organization-performance-of-computer/
    \bibitem{gfg2024}
    GeeksforGeeks. (2024, August 20). Program for round robin scheduling for the same arrival time. GeeksforGeeks. Retrieved October 8, 2024, from https://www.geeksforgeeks.org/program-for-round-robin-scheduling-for-the-same-arrival-time/
\end{thebibliography}

\end{document}