\documentclass[12pt]{article}
\usepackage{amsmath}
\usepackage{graphicx}
\usepackage{hyperref}
\usepackage{listings}
\usepackage{color}
\usepackage{pythonhighlight}

\title{Operating System Course Report - First Half of the Semester}
\author{B class}
\date{\today}

\begin{document}

\maketitle
\newpage

\tableofcontents
\newpage

\section{Introduction}
This report summarizes the topics covered during the first half of the Operating System course. It includes theoretical concepts, practical implementations, and assignments. The course focuses on the fundamentals of operating systems, including system architecture, process management, CPU scheduling, and deadlock handling.

\section{Course Overview}
\subsection{Objectives}
The main objectives of this course are:
\begin{itemize}
    \item To understand the basic components and architecture of a computer system.
    \item To learn process management, scheduling, and inter-process communication.
    \item To explore file systems, input/output management, and virtualization.
    \item To study the prevention and handling of deadlocks in operating systems.
\end{itemize}

\subsection{Course Structure}
The course is divided into two halves. This report focuses on the first half, which covers:
\begin{itemize}
    \item Basic Concepts and Components of Computer Systems
    \item System Performance and Metrics
    \item System Architecture of Computer Systems
    \item Process Description and Control
    \item Scheduling Algorithms
    \item Process Creation and Termination
    \item Introduction to Threads
    \item File Systems
    \item Input and Output Management
    \item Deadlock Introduction and Prevention
    \item User Interface Management
    \item Virtualization in Operating Systems
\end{itemize}

\section{Topics Covered}

\subsection{Basic Concepts and Components of Computer Systems}
This section explains the fundamental components that make up a computer system, including the CPU, memory, storage, and input/output devices.

\subsection{System Performance and Metrics}
This section introduces various system performance metrics used to measure the efficiency of a computer system, including throughput, response time, and utilization.

\subsection{System Architecture of Computer Systems}
Describes the architecture of modern computer systems, focusing on the interaction between hardware and the operating system.

\subsection{Process Description and Control}
Processes are a central concept in operating systems. This section covers:
\begin{itemize}
    \item Process states and state transitions
    \item Process control block (PCB)
    \item Context switching
\end{itemize}

\subsection{Scheduling Algorithms}
This section covers:
\begin{itemize}
    \item First-Come, First-Served (FCFS)
    \item Shortest Job Next (SJN)
    \item Round Robin (RR)
\end{itemize}
It explains how these algorithms are used to allocate CPU time to processes.

\subsection{Process Creation and Termination}
Details how processes are created and terminated by the operating system, including:
\begin{itemize}
    \item Process spawning
    \item Process termination conditions
\end{itemize}

\subsection{Introduction to Threads}
This section introduces the concept of threads and their relation to processes, covering:
\begin{itemize}
    \item Single-threaded vs. multi-threaded processes
    \item Benefits of multithreading
\end{itemize}

\subsection{File Systems}
File systems provide a way for the operating system to store, retrieve, and manage data. This section explains:
\begin{itemize}
    \item File system structure
    \item File access methods
    \item Directory management
\end{itemize}

\subsection{Input and Output Management}
Input and output management is key for handling the interaction between the system and external devices. This section includes:
\begin{itemize}
    \item Device drivers
    \item I/O scheduling
\end{itemize}

\subsection{Deadlock Introduction and Prevention}
Explores the concept of deadlocks and methods for preventing them:
\begin{itemize}
    \item Deadlock conditions
    \item Deadlock prevention techniques
\end{itemize}

\subsection{User Interface Management}
		
		Manajemen antarmuka pengguna (User Interface Management) adalah komponen kritis dalam sistem operasi modern yang bertanggung jawab untuk mengatur interaksi antara pengguna dan komputer \cite{Lazar2017}. Fungsi utamanya meliputi pengelolaan input pengguna, rendering output visual, dan koordinasi berbagai elemen antarmuka untuk menciptakan pengalaman yang kohesif dan efisien.
		
		Ciri-ciri penting dari manajemen antarmuka pengguna meliputi:
		\begin{itemize}
			\item Konsistensi: Menjaga keseragaman dalam desain dan fungsi di seluruh aplikasi.
			\item Responsivitas: Memastikan sistem merespons input pengguna dengan cepat dan akurat.
			\item Aksesibilitas: Menyediakan fitur yang memungkinkan penggunaan oleh individu dengan berbagai kemampuan.
			\item Adaptabilitas: Kemampuan untuk menyesuaikan dengan preferensi pengguna dan konteks penggunaan.
			\item Keamanan: Melindungi data pengguna dan sistem dari akses yang tidak sah.
		\end{itemize}
		
		Sistem operasi modern mengimplementasikan manajemen antarmuka pengguna melalui berbagai komponen, termasuk window manager, display server, dan toolkit antarmuka pengguna \cite{Tanenbaum2015}. Integrasi yang mulus antara komponen-komponen ini memungkinkan sistem operasi untuk mendukung berbagai jenis antarmuka, yang paling umum adalah Graphical User Interface (GUI) dan Command-Line Interface (CLI). 


\subsection{Virtualization in Operating Systems}
Virtualization allows multiple operating systems to run concurrently on a single physical machine. This section explores:
\begin{itemize}
    \item Concept of virtualization
    \item Hypervisors and their types
    \item Benefits of virtualization in modern computing
\end{itemize}

\section{Assignments and Practical Work}
\subsection{Assignment 1: Process Scheduling}
Students were tasked with implementing various process scheduling algorithms (e.g., FCFS, SJN, and RR) and comparing their performance under different conditions.
\subsubsection{Group 1}
\begin{python}
    class Process:
    def __init__(self, pid, arrival_time, burst_time):
        self.pid = pid
        self.arrival_time = arrival_time
        self.burst_time = burst_time
        self.completion_time = 0
        self.turnaround_time = 0
        self.waiting_time = 0
\end{python}

\begin{table}[htbp] % Optional: For floating position
    \centering
    \begin{tabular}{|c|c|c|} % Defines number of columns and alignment (c = center, l = left, r = right). '|' creates vertical lines.
    \hline
    Header 1 & Header 2 & Header 3 \\ % Column headers
    \hline
    Row 1, Column 1 & Row 1, Column 2 & Row 1, Column 3 \\ % First row of data
    \hline
    Row 2, Column 1 & Row 2, Column 2 & Row 2, Column 3 \\ % Second row of data
    \hline
    \end{tabular}
    \caption{Your table caption} % Optional: For adding a caption
    \label{tab:your_label} % Optional: For cross-referencing the table
\end{table}

\subsection{Assignment 2: Deadlock Handling}
In this assignment, students were asked to simulate different deadlock scenarios and explore various prevention methods.

\subsection{Assignment 3: Multithreading and Amdahl's Law}
This assignment involved designing a multithreading scenario to solve a computationally intensive problem. Students then applied **Amdahl's Law** to calculate the theoretical speedup of the program as the number of threads increased.

\subsection{Assignment 4: Simple Command-Line Interface (CLI) for User Interface Management}
Students were tasked with creating a simple **CLI** for user interface management. The CLI should support basic commands such as file manipulation (creating, listing, and deleting files), process management, and system status reporting.

\subsection{Assignment 5: File System Access}
In this assignment, students implemented file system access routines, including:
\begin{itemize}
    \item File creation and deletion
    \item Reading from and writing to files
    \item Navigating directories and managing file permissions
\end{itemize}

\subsubsection{Group 12}
{Soal 1: Pembuatan dan Penghapusan File}
Buatlah sebuah program yang akan membuat file bernama \texttt{example.txt} dan menulis teks "Ini adalah contoh file" ke dalamnya. Setelah itu, tampilkan apakah file tersebut berhasil dibuat. Kemudian, hapus file tersebut dan periksa apakah file berhasil dihapus.

\paragraph{Jawaban:}
\begin{verbatim}
import os

# Membuat dan menulis ke file
with open("example.txt", "w") as file:
    file.write("Ini adalah contoh file")

# Mengecek apakah file berhasil dibuat
if os.path.exists("example.txt"):
    print("File 'example.txt' berhasil dibuat.")
else:
    print("File 'example.txt' gagal dibuat.")

# Menghapus file
os.remove("example.txt")

# Mengecek apakah file berhasil dihapus
if not os.path.exists("example.txt"):
    print("File 'example.txt' berhasil dihapus.")
else:
    print("File 'example.txt' gagal dihapus.")
\end{verbatim}

\paragraph{Output:}
\begin{verbatim}
File 'example.txt' berhasil dibuat.
File 'example.txt' berhasil dihapus.
\end{verbatim}

\subsubsection{Group 12}
{Soal 2: Membaca dan Menulis ke File}
Buatlah sebuah program yang akan menambahkan teks "Tambahan teks baru" ke dalam file \texttt{log.txt} tanpa menghapus konten yang sudah ada. Kemudian, baca dan tampilkan isi file \texttt{log.txt} setelah penambahan teks.

\paragraph{Jawaban:}
\begin{verbatim}
# Menambahkan teks baru ke file tanpa menghapus konten sebelumnya
with open("log.txt", "a") as file:
    file.write("Tambahan teks baru\n")

# Membaca dan menampilkan isi file
with open("log.txt", "r") as file:
    content = file.read()

print("Isi file 'log.txt':\n", content)
\end{verbatim}

\paragraph{Output:}
\begin{verbatim}
Isi file 'log.txt':
Tambahan teks baru
\end{verbatim}

\subsubsection{Group 12}
{Soal 3: Menavigasi Direktori}
Buatlah sebuah program yang akan membuat folder baru bernama \texttt{new\_folder}. Setelah itu, program harus berpindah ke direktori tersebut dan menampilkan direktori saat ini. Terakhir, kembali ke direktori awal dan hapus folder \texttt{new\_folder}.

\paragraph{Jawaban:}
\begin{verbatim}
import os

# Membuat folder baru
os.mkdir("new_folder")
print("Folder 'new_folder' berhasil dibuat.")

# Pindah ke dalam folder yang baru dibuat
os.chdir("new_folder")
print("Direktori saat ini:", os.getcwd())

# Kembali ke direktori awal
os.chdir("..")
print("Kembali ke direktori awal:", os.getcwd())

# Menghapus folder yang baru dibuat
os.rmdir("new_folder")
print("Folder 'new_folder' berhasil dihapus.")
\end{verbatim}

\paragraph{Output:}
\begin{verbatim}
Folder 'new_folder' berhasil dibuat.
Direktori saat ini: /path/to/current/directory/new_folder
Kembali ke direktori awal: /path/to/current/directory
Folder 'new_folder' berhasil dihapus.
\end{verbatim}

\subsubsection{Group 12}
{Soal 4: Manajemen Izin File}
Buatlah sebuah program yang akan mengubah izin file \texttt{contoh.txt} sehingga hanya pemilik file yang memiliki akses baca dan tulis. Setelah itu, tampilkan izin file yang baru. Jika file \texttt{contoh.txt} tidak ada, buat file terlebih dahulu.

\paragraph{Jawaban:}
\begin{verbatim}
import os
import stat

# Membuat file jika tidak ada
if not os.path.exists("contoh.txt"):
    with open("contoh.txt", "w") as file:
        file.write("Ini adalah file contoh.")

# Mengubah izin file sehingga hanya pemilik yang dapat membaca dan menulis
os.chmod("contoh.txt", stat.S_IRUSR | stat.S_IWUSR)

# Mengecek izin file yang telah diubah
file_stat = os.stat("contoh.txt")
print("Izin file 'contoh.txt':", oct(file_stat.st_mode)[-3:])
\end{verbatim}

\paragraph{Output:}
\begin{verbatim}
Izin file 'contoh.txt': 600
\end{verbatim}

\subsubsection{Group 12}
{Soal 5: Mengecek Izin File}
Buatlah sebuah program yang memeriksa apakah file \texttt{data.txt} memiliki izin tulis untuk pemilik. Jika tidak, berikan izin tulis kepada pemilik dan tampilkan perubahan yang terjadi.

\paragraph{Jawaban:}
\begin{verbatim}
import os
import stat

# Mengecek apakah file memiliki izin tulis untuk pemilik
file_stat = os.stat("data.txt")
if not bool(file_stat.st_mode & stat.S_IWUSR):
    print("File 'data.txt' tidak memiliki izin tulis untuk pemilik.")
    os.chmod("data.txt", file_stat.st_mode | stat.S_IWUSR)
    print("Izin tulis telah diberikan kepada pemilik.")
else:
    print("File 'data.txt' sudah memiliki izin tulis untuk pemilik.")

# Mengecek izin file setelah perubahan
file_stat = os.stat("data.txt")
print("Izin file 'data.txt':", oct(file_stat.st_mode)[-3:])
\end{verbatim}

\paragraph{Output:}
\begin{verbatim}
File 'data.txt' tidak memiliki izin tulis untuk pemilik.
Izin tulis telah diberikan kepada pemilik.
Izin file 'data.txt': 600
\end{verbatim}



\section{Conclusion}
The first half of the course introduced core operating system concepts, including process management, scheduling, multithreading, and file system access. These topics provided a foundation for more advanced topics to be covered in the second half of the course.

\begin{thebibliography}{9}

\bibitem{Lazar2017} 
    Lazar, J., Feng, J. H., \& Hochheiser, H. (2017). \textit{Research methods in human-computer interaction} (2nd ed.). Morgan Kaufmann.

\bibitem{Tanenbaum2015} 
    Tanenbaum, A. S., \& Bos, H. (2015). \textit{Modern operating systems} (4th ed.). Pearson.

\end{thebibliography}



\end{document}
